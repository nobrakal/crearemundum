\documentclass[a4paper, 11pt]{article}
\usepackage[utf8]{inputenc}
\usepackage[T1]{fontenc}
\usepackage[francais]{babel}
\usepackage[linkcolor=blue, colorlinks=true,]{hyperref}
\usepackage{graphicx}
\usepackage{fancyhdr}
\pagestyle{fancy}
\fancyhead[C]{\rightmark}
\fancyhead[L]{}
\fancyhead[R]{}
\fancyfoot[C]{\hyperlink {participation} {Envie de participer?}}

\makeatletter
\let\insertdate\@date
\makeatother

% Creare Mundum est sous licence CC-BY-SA, présente dans le dossier d'origine, merci de la respecter!

\title{La Silencieuse \\ Scénario pour Creare Mundum \\ Devel}
\author{The Creare Mundum Project}
\date{\oldstylenums{\insertdate}}
\begin{document}
\maketitle
\setcounter{tocdepth}{1} %Génération du Sommaire.
\renewcommand{\contentsname}{Sommaire} 
\tableofcontents
\newpage
\section{Synopsis et présentation}
Prenant place dans le monde de {\hyperlink {participation} {Creare Mundum}}, ce scénario met les joueurs dans une situation peu commune. 
\part{À la découverte de la Silencieuse}
\section{Une bien mystérieuse mission}
Le groupe se trouve dans la région des monts mineurs, où il a déjà rempli quelques missions pour un baron local, qui leur fait désormais plutôt confiance. Il les contacte une fois de plus pour leur confier une tâche qui promet de ne pas s'avérer des plus faciles, trouver une cité disparue. La missive qu'il leur envoie comporte ces mots :
\begin{quotation}
Mes amis. 
\newline
Je vous envoie à nouveau à la recherche d'une chimère. il s'agit cette fois de \textit{La Silencieuse}, une cité réputée apparaître à ceux qui se perdent sur les rives du Lac Al-Adren. Vous le savez, je suis féru de mythes, et cette cité est souvent vue par les pêcheurs comme un conte pour les enfants. C'est la raison principale de mon intérêt pour cette dernière, mais j'ai peur que cela soit aussi un frein à votre enquête. Je vous demande donc de faire preuve de discernement, et d'éviter de trop utiliser mon nom sous peine de le discréditer.
\newline
Amicalement votre, le Baron Faenor d'Ermusse.
\end{quotation}
Il est probable que les PJ se lancent aussitôt à la recherche de leur objectif. S'ils essaient de se renseigner auprès des autochtones, ils seront tournés en dérision et aucune bribe d'information ne leur sera donnée, tous les pêcheurs étant bien trop terre-à-terre pour prêter attention à ces ragots de bonne femme. Cependant, un personnage sera disposé à leur donner quelques pistes. il peut s'agir d'un aubergiste pressé de les voir déguerpir, d'un enfant rêveur ou encore autre-chose. Il leur signalera la demeure d'un vieil excentrique, qui a soit-disant consacré des années à faire des recherches sur la Silencieuse. Il habite aux abords de Montien-Les-Berges, le plus gros village de la région, et le seul qui ne soit pas uniquement tourné vers la pêche. 
\section{Le but se précise}
Les PJ prennent donc la route qui leur a été indiquée. Le temps est clément, le paysage splendide, et il fait bon se promener le long des berges du lac. Le groupe n'en est que plus surpris quand une bande de bandits crasseux sort des fourrés en maugréant. Avant de poser la moindre question, les malfrats se jettent à l'attaque. Mal équipés, sans aucune expérience du combat, seul leur léger surnombre est un avantage. Ils se font rapidement tailler en pièces, sans infliger le moindre dégât aux PJ. Dès les premières victimes, les bandits fuient sans demander leur reste. Ils ne s'attendaient visiblement pas à une telle résistance (il s'agit simplement de dégourdir et de distraire les joueurs).
\newline
Dès leur arrivée à Montien, les PJ aperçoivent une demeure qui semble appartenir à l'excentrique dont ils ont entendu parler. C'est une bicoque biscornue, à l'écart du reste du village. Le terrain aux alentours est en friche, seul un petit potager semble échapper à la négligence évidente du maître des lieux. Si les PJ se renseignent plus avant, il apparaît en effet qu'un vieux solitaire habite dans cette maison. Il est assez rare de le voir aux alentours, mais la plupart des villageois l'évitent soigneusement. Cependant, dès qu'un personnage toque à sa porte, l'ermite se précipite pour ouvrir. Il est vêtu simplement, de manière négligée mais propre. Après quelques formules de courtoisie bafouillées à la hâte, il s'enquit de la raison de cette visite impromptue. Dès qu'il est fait mention de la Silencieuse, son visage s'illumine. Il fait entrer le groupe, et commence à parler tellement vite que ses paroles en sont presque incompréhensibles. Finalement, il leur tend une carte des environs. Cette dernière est d'aspect usé, et une myriade de marques à la plume la constelle. Selon l'homme, chaque marque représente une apparition de la cité. A ce moment de l'aventure, sur un jet d'intelligence (ou autre, à voir selon le système) ou encore si les joueurs s'en aperçoivent d'eux-mêmes, il apparaît au groupe que le vieillard ne leur a jamais donné son nom. Si jamais ils décident de lui demander, sa réponse sera évasive, comme si il rechignait à leur donner. A chaque fois que la question est posée, trouvez un nouveau prétexte, une nouvelle information pour ne pas leur donner de réponse.
\section{Les problèmes commencent}
Le groupe partira donc probablement vers la zone du lac où toutes les marques sont concentrées sur la carte. Il fait beau, la promenade est agréable et il est facile d'oublier qu'ils sont en mission pour un noble local. La végétation, au départ assez peu fournie, se densifie de plus en plus. Rapidement, une sorte de forêt d'ajoncs couvre les berges du lac. Les PJ s'aperçoivent qu'ils sont arrivés à leur but. Il leur semble entendre au loin une musique douce, assez ténue. Pour l'instant, seules quelques notes leur parviennent, égrenées dans la brise. \emph{Il est conseillé à ce stade de passer à vos joueurs une musique douce, assez régulière et entêtante. Pensez par exemple à une boîte à musique.} Plus ils entendent la musique, plus les personnages se sentent absorbés, comme des papillons par une flamme. Ils commencent alors presque inconsciemment à avancer dans les ajoncs, se laissant porter par la musique. Après quelques tests de volonté réussis, ou simplement après un laps de temps plus ou moins long, ils se 'réveillent' brusquement. Ils n'entendent plus une seule note, rien que quelques oiseaux et le vent sur le lac. Leur champ de vision est réduit au minimum, il sont complètement perdus.
\newline
Brusquement, les tiges aux alentours frémissent. Soudain, un combat s'engage.
\begin{flushright}
\texttt{Les Gardiens de la Silencieuse sont des monstres complètement inconnus aux joueurs. Ils mesurent approximativement 1.5 mètre de haut. Les Gardiens se déplacent sur deux bras musculeux, terminés par des mains à quatre doigts. Leurs épaules sont larges, leurs hanches plutôt étroites et leurs jambes atrophiées pendent le plus souvent inutilement. Leur tête semble plutôt ronde, attachée au corps par un coup presque inexistant. Ils ne possèdent qu'un œil, immense, jaune et injecté de sang, à la pupille verticale. Leur immense gueule est garnie de centaines de crocs ressemblant à des aiguilles, presque translucides. Pour finir, leur peau est un cuir épais, brun-gris, sans aucun poil. Ils poussent de petits glapissements à chaque bond.}
\end{flushright} 
Les Gardiens possèdent une force musculaire surprenante. Au nombre de trois, ils tentent de surprendre leurs adversaires en surgissant des ajoncs en en les plaquant au sol, espérant infliger une blessure sérieuse avant de repartir. \emph{Utilisez pour les Gardiens des profils de Gargouille ou équivalent. Le but est d'infliger plusieurs blessures aux joueurs, mais pas de les mettre en danger mortel.}
\section{L'arrivée à la Silencieuse}
A la fin du combat, les joueurs se retrouvent au bord même du lac. Pourtant, la végétation alentour n'a pas été piétinée, ni même dérangée, comme si ils avaient toujours été là. Cependant, ce qui attire leur attention, c'est le ponton qui s'éloigne de la berge, se dirigeant vers une cité sur pilotis. Celle-ci semble modeste d'apparence, quelques cabanes reliées par des ponts en bois. Cependant, il est difficile de se faire une idée précise à cette distance. 
\newline
A ce stade-là, il serait improbable que le groupe décide de faire demi-tour ! Les PJ avancent donc sur le ponton. Rapidement, ils s'aperçoivent qu'ils n'ont pas l'impression de s'approcher ni de l'éloigner de la cité. Cette réalisation est rapidement suivie d'une seconde. Les PJ ne marchent plus sur un ponton en bois, mais dans une rue en pierre. De plus, ils sont entourés de bâtiments en pierre, construits dans un style architectural très différent de ce qu'ils connaissent. Les bâtiments sont légers et fins, les arcades se succèdent sur plusieurs niveaux. Aucune construction ne semble habitable, les quelques entrées n'ouvrant que sur des pièces vides. Au bout de quelques heures de déambulations, les PJ se rendent compte d'un fait étrange. Qu'importe leur direction, ils sont attirés inlassablement vers le même manoir, construit au milieu d'une place vide.
\section{Le Manoir}
L'intérieur du Manoir, contrairement au reste de la cité, semble plutôt normale. Chaque salle est meublée de façon sommaire et un peu désuète. Le Manoir est construit de la façon suivante :
\begin{flushright}
\texttt{En entrant dans le manoir, trois choix s'offrent au visiteur. Il peut rester au rez de chaussée, auquel cas il pénétrera dans une série de salles normalement meublées. Cet ensemble comporte un salon et une salle à manger. Les deux semblent confortables, mais aucune fenêtre n'est percée dans les murs. 
\newline
La deuxième possibilité est le premier étage. Ce dernier est formé d'un couloir, de deux chambres et d'un bureau. Si un des PJ se penche par une fenêtre, il aperçoit des silhouettes noires dans les rues. Ces silhouettes portent une grande cape, qui leur couvre le corps et leur cache le visage. Ces silhouettes se déplacent doucement dans les rues, sans but apparent. Si un PJ redescend dans la rue, il la voit vide. Si il remonte et regarde par la fenêtre, il voit les silhouettes.
\newline
Enfin, et c'est l'option qui nous intéresse, un couloir est percé en face de l'entrée du manoir. Ne mettez pas trop l'accent dessus dans un premier temps, il serait intéressant que le groupe visite le premier étage, mais c'est bien ce couloir qui est la clé pour la suite de cette aventure.}
\end{flushright}
\section{Adieu, Silencieuse, ou au revoir}
Après avoir descendu un escalier de quelques marches, le groupe pénètre dans le couloir lui-même. Ce dernier est étrangement vide, fait de pierre nue. Aucune source de lumière n'est présente, pourtant tout est visible comme en plein jour. Au bout de quelque temps -quelques secondes, quelques minutes, ou quelques années, il est difficile de se faire une idée précise...- le couloir change brusquement de nature. Des fenêtres sont percées sur les murs, et elles donnent un aperçu de la cité. Cependant, les fenêtres surplombent les rues de plusieurs dizaines de mètres. Si les PJ se penchent par une fenêtre, ils voient la cité à perte de vue, et le couloir, qui se prolonge à l'infini, sans soutien apparent. Les fenêtres se referment enfin, plus rien n'est visible à par l'intérieur du couloir. Le couloir débouche enfin sur une pièce carrée, avec un plafond un peu plus haut que celui du couloir. Soudain, quatre des silhouettes aperçues dans la rue surgissent des murs.
\begin{flushright}
\texttt{Les Spectres ne sont guère plus que des silhouettes dissimulées sous des capes d'un noir d'encre. Sous leur capuche, il est parfois possible d'entrevoir un visage émacié, presque squelettique, et d'une blancheur de cadavre. Les yeux des Spectres se résument à des billes noires. Pour représenter ces monstres en jeu, utilisez un profil de fantôme ou équivalent. Ils se battent en fonçant littéralement sur les joueurs pour les traverser. Lorsqu'un Spectre touche un PJ, celui-ci ressent un froid mortel et subit d'importants malus. Une fois tués, ils se dissolvent en une sorte de fumée noirâtre. Les Spectres peuvent être des adversaires sérieux, étant les derniers monstres du scénario. Attention toutefois à ne pas tuer de PJ, leur survie est importante pour la suite de l'aventure.}
\end{flushright}
A la fin du combat, seul une des entrées de la salle est restée ouverte. Après quelques mètres, le couloir se termine en cul-de-sac et un parchemin est posé au sol. Sur ce parchemin, ces mots sont inscrits :
\begin{center}
\emph{Franchis le seuil en invité,
\newline
Ou rebrousse chemin, étranger.
\newline
Entre si tu t'en sens digne,
\newline
Encore faut-il savoir lire entre les lignes.}
\end{center}
La solution de l'énigme est la suivante : en lisant une ligne sur deux, on obtient "Franchis le seuil en invité, entre si tu t'en sens digne." Quand les joueurs trouvent, ils passent à travers le mur, et la séance est ainsi clôturée...
\part{Un employeur discret}
\section{Une lettre sur la table}
Les héros arrivent dans une grande pièce, totalement nue, avec en son centre une table. Sur cette table, une lettre. Voici ce qu'elle dit.
\begin{quotation}
Bonjour,
\newline
Vous avez eu une démonstration de mes pouvoirs avec l'ordre que vous a donné le Baron d'Ermusse d’enquêter sur cette étrange maison, ma demeure, et vous aurez l'occasion d'en voir encore plus. Mais je ne puis tout faire, et c'est là que vous entrez en jeu. J'ai besoin de personnes telles que vous pour quelques missions. Acceptez-vous de vous mettre à mon service?
\newline
Je saurais vous récompenser, bien sûr. Considérez comme une avance ces petites choses. Je crois qu'elles ont grande valeur en ce monde.
\newline
M.G.
\end{quotation}
En effet, sur la table se trouve un nombre de petites pierres noires (il s'agit de pierres d'âmes) égal à celui de héros. Elles sont pour l'instant encastrée dans la table. Cela ressemble à de la puissante magie. Si les héros acceptent, des nouvelles lignes apparaissent sur la lettre:
\begin{quotation}
Vous  avez fait le bon choix. Votre première tâche sera d'aller déloger un groupe de gobelin qui sévissent dans les environs de Dren. Ces derniers prévoient une attaque d'un des avants poste de la ville, Kurok, alors que les gardes chassent un autre petit groupe de l'autre côté de la montagne. Dépêchez-vous.
\end{quotation}
Les pierres sont alors libérées de leur socle. Voici leurs effets respectifs:
\begin{enumerate}
\item La force du personnage augmente d'un quart
\item Aucun nouvel effet en apparence, mais le personnage sais lire les messages codés/cachés
\item Le personnage sais parler une nouvelle langue de son choix
\item Le personnage augmente ses chances de toucher d'un quart
\item Le personnage augmente ses points de vie d'un quart
\end{enumerate}
Une porte se dresse devant les héros, gravées de mystérieux symboles.
\chapter{L'arrivée dans la cité blanche}
\section{Wwwaaaahhh}
Nos héros, ouvrent la mystérieuse porte. Ils se retrouvent pleine ville. 
\begin{flushright}
\texttt{Vous vous retrouvez en pleine ville. À droite, une marchande de légume vous salut, comme si elle vous connaissait depuis toujours, des enfants courent et jouent.
\newline
En levant les yeux, vous voyez une magnifique maison blanche qui domine la cité.
\newline
Bienvenue à Transition.}
\end{flushright}
\emph{Nous conseillons au maitre de jeu d'utiliser une musique très particulière (ex: A. Piazolla: Las cuatro estaciones porteñas) qui permet une identification profonde de la ville}
Ils peuvent déambuler dans la ville, acheter équipements et autres potions. Si ils se renseignent sur leurs pierres d'âmes
\subsection{Pour vendre des pierres d'âmes}
Transition centralise tout le commerce de pierres d'âmes. "Chez Thierry", "l'Animus" ou encore "La Pierre noire" sont autant de boutiques réputées. Une pierre d'âme vaut très cher (environ le prix de deux rubis de la même taille), et les marchands tenteront évidement d'arnaquer nos pauvres héros.
\subsection{Pour assimiler des pierres d'âmes}
Se reporter à la section correspondante de CreareMundum
\subsection{Se rendre à Kurok}
Les nains communiquent assez peu sur leur géographie, et personne à Transition ne sait où se trouve Kurok.

%---------------------------------------------------------------------------------------------

\newpage
\section{Participation au projet}
\subsection{Comment?}
\hypertarget{participation}{}
Creare Mundum vous a plu? 
Envie de partager vos scénarios, vos suggestions ou vos idées?
\newline
Contactez-nous sur la mailing list: \href {mailto:crearemundum@lists.tuxfamily.org}{crearemundum@lists.tuxfamily.org}
\newline
Ou rendez vous sur notre site: \href {http://creare-mundum.tuxfamily.org/} {http://creare-mundum.tuxfamily.org/}
\subsection{Les créateurs}
Voici la liste de ceux qui ont participé au projet Creare Mundum. Leur aide fut, est et sera toujours très précieuse au projet. Merci encore!  
\begin{itemize}
\item Alexandre ’Nobrakal’ Moine 
\item Philippe ’Tymophil’ Aubé 
\item Eliott ’Sulf’ Filippi
\item Émiland ’Derec’ Garrabé
\item Luc H.
\end{itemize}
\subsection{Licence}
Creare Mundum est un projet libre de droit, publié sous la licence Creative Commons BY-SA. C'est à dire que quiconque a la possibilité d'utiliser ce document (ainsi que toute les autres parties du projet), de le redistribuer et de le modifier. La seule obligation est de redistribuer le contenu (modifié ou non) sous les mêmes conditions.
\end{document}