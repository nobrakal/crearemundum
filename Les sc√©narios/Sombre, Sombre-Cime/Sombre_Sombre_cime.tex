\documentclass[a4paper, 11pt]{article}
\usepackage[utf8]{inputenc}
\usepackage[T1]{fontenc}
\usepackage[french]{babel}
\usepackage[linkcolor=blue, colorlinks=true,]{hyperref}
\usepackage{graphicx}
\usepackage{fancyhdr}
\pagestyle{fancy}
\fancyhead[C]{\rightmark}
\fancyhead[L]{}
\fancyhead[R]{}
\fancyfoot[C]{\hyperlink {participation} {Envie de participer?}}

\makeatletter
\let\insertdate\@date
\makeatother

% Creare Mundum est sous licence CC-BY-SA, présente dans le dossier d'origine, merci de la respecter!

\title{Sombre, Sombre-Cime\\ Scénario pour Creare Mundum \\ Devel}
\author{The Creare Mundum Project}
\date{\oldstylenums{\insertdate}}
\begin{document}
\maketitle
\setcounter{tocdepth}{1} %Génération du Sommaire.
\renewcommand{\contentsname}{Sommaire} 
\tableofcontents
\newpage

%----------------------------------------------------------------------------
\section{Introduction pour le maitre de jeu}
"Sombre, Sombre-Cime" est un court scénario de trois à quatre heures. Il met en scène la mort des personnages, qu'un nécromancien utilise alors à ses fins. Ils seront sauvés par leur courage et l'aide d'un grand prêtre de l'ordre de la guérison, Monseigneur Torwar.
Cette intrigue prend place dans l'univers de Creare Mundum, durant l'ère médiévale. Pour plus de précisions concernant le monde, consultez le fichier principal.
\newline
Creare Mundum n'a pas de système de règle propre, il reste donc au maitre de jeu le travail d'adapter ce scénario au système qu'il souhaite utiliser, notamment e ce qui concerne la création des monstres.
\newline
Les paragraphes commençant par "Partie supplémentaire: " sont comme leur nom l'indique, hors de l'intrigue principale, mais permettent une plus grande immersion et un allongement du temps de jeu.
\newline
Les instructions et le déroulement qui suivent sont bien sûr donné uniquement à titre indicatif, n'hésitez pas à modifier cette aventure comme bon vous semble!

\section{Du vieil aulne au vieux chêne}
\subsection{Un départ}
L'action commence à la périphérie d'Anksfall. Les personnages se trouvent sur la place de l’aulne, lieu de départ de la plupart des voyages vers le nord. Nos héros ne se connaissent pas forcément, et leur arrivée un par un peut être un effet de style intéressant. C'est de cette place qu'une carriole doit vernir les prendre pour aller à Sombre-Cime. En effet, la "Grande Foire de Sombre-Cime" approche et les personnages y vont pour des motifs différents. Les personnages  se rencontrent donc sur cette petite place, qui n'as de place que le nom. Un grand aulne se dresse a coté de la grande route. Un petit abris a été dressé pour protéger les voyageurs. La scène du voyage est dédiée à la rencontre des personnages. Le voyage comporte peut de danger et dure une dizaine d'heure. Dans la carriole se trouve aussi trois chevaliers qui partent à la foire: Revir, Pegrus et Larsus. Étrangement, ils n'ont pas l'air festif, bien qu'ils démentiront quoi que ce soit si les héros leur posent des questions.

\subsection{Partie supplémentaire: L'attaque des gobelins} 
La compagnie est attaquée peu avant l'arrivée, sur une petite route qui coupe à travers un petit bois des plaines de sang, que le cocher assurait "sans danger". Un groupe composé d'autant de gobelins que de personnages plus 3 entre en combat. Ils ne sont pas particulièrement fort, ils manque des membres à certaines d'entre eux, et à leur mort, ils semblent comme retrouver le sourire. Il s'agit en fait de cobayes du nécromancien dont les héros seront les victimes. 

\subsection{Une arrivée}
Les héros arrivent en soirée ou au petit matin, la foire commence de toute façon le lendemain. Ils font ce qu'ils veulent, flâner dans les boutiques (la plupart on déjà leurs étalages de près), prendre une chambre d'hôtel, visiter la ville... S'ils approchent de l'église de la ville, dédiée à l'ordre du soin, ils y voient de l'agitation. Si les personnages cherchent à savoir pourquoi, en posant quelques questions, il apprendront vite qu'un des grands prêtres de l'ordre, Torwar, est ici pour célébrer une messe pour la foire. Faites durer cette partie autant que vous voulez, mais il faut faire attention à ne pas ennuyer les personnages.

\subsection{La mort est douce}
Cette partie est plus intéressante jouée dans une semi-obscurité. Les personnages vont y mourir, de la main d'un mystérieux spectre. Ils passent l'arme à gauche un par un, toutes les 5 minutes environ, quoiqu'ils fassent. Aidez-les dans leur fuite, conseillez-les, jouez sur leur ressentis: Suivez la voie qu'il tracent pour vous. Voici des exemples de ce qu'ils peuvent ressentir:
\begin{itemize}
\item "Tu sens une piqure dans le cou. Ton cœur ralentit, puis s’arrête"
\item  "Tu sens une piqure sur la jambe. Après avoir fait trois pas, elle se dérobe sous ton poids. Tu tombes, perdant connaissance. "
\item  "Une douleur atroce se propage dans ton corps, mais ta voix refuse de produire un son. Tu tombes par terre en gémissant. "
\end{itemize}

\section{Le château grotesque}
\subsection{Illusion}
Nos héros sont réveiller par des cris. Ils sont dans une chambre de princesse, sans doute en haut d'un donjon. La princesse, justement, leur cri de se préparer, qu' "Ils arrivent". En effet, des bruits se font entendre dans les escaliers.

%----------------------------------------------------------------------------

\newpage
\section{Participation au projet}
\subsection{Comment?}
\hypertarget{participation}{}
Creare Mundum vous a plu? 
Envie de partager vos scénarios, vos suggestions ou vos idées?
\newline
Contactez-nous sur la mailing list: \href {mailto:crearemundum@lists.tuxfamily.org}{crearemundum@lists.tuxfamily.org}
\newline
Ou rendez vous sur notre site: \href {http://creare-mundum.tuxfamily.org/} {http://creare-mundum.tuxfamily.org/}
\subsection{Les créateurs}
Voici la liste de ceux qui ont participé au projet Creare Mundum. Leur aide fut, est et sera toujours très précieuse au projet. Merci encore!  
\begin{itemize}
\item Alexandre ’Nobrakal’ Moine 
\item Philippe ’Tymophil’ Aubé 
\item Eliott ’Sulf’ Filippi
\item Émiland ’Derec’ Garrabé
\item Luc H.
\end{itemize}
\subsection{Licence}
Creare Mundum est un projet libre de droit, publié sous la licence Creative Commons BY-SA. C'est à dire que quiconque a la possibilité d'utiliser ce document (ainsi que toute les autres parties du projet), de le redistribuer et de le modifier. La seule obligation est de redistribuer le contenu (modifié ou non) sous les mêmes conditions.
\end{document}