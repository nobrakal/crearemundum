\section{Note 58 : Transition se dépeuple}
\subsection{Les Elfes ont perdu}
La guerre contre les Nalfein est terminée. Ils sont vaincus, si l'on peut dire. Les Elfes et les Nains aussi ont été victimes de cette guerre. Ils sont presque anéantis, et prendront des décennies, voir des siècles à se remettre. En attendant, les humains vont prendre leur essor, comme je m'en étais douté. Ils commencent déjà à clamer des domaines aux abords des forêts Elfes, mais pour l'instant ma curiosité est braquée sur autre chose. Je pense que Transition va bientôt changer de nature et de population. Déjà quelques Elfes sont retournés à Thorneye, et la cité est plus calme qu'elle ne l'a été depuis longtemps.
\subsection{Nous allons partir}
Depuis la défaite de nos armée contre les Nalfein, je pense sans arrêt à retourner au palais. Elleanol a perdu ses saveurs depuis quelques temps. Nombreux ont été ceux parmi mes amis qui ont fait de même. De plus, aucun des poèmes que j'ai écrits récemment n'a la moindre valeur. Je ne suis capable que d'aligner des mots sans queue ni tête depuis des jours. Il faut à tout prix que je retrouve un cadre plus propice à l'écriture et à la détente. Oui, c'est décidé, je pars. Elleanol va connaître des changements, et j'ai la ferme intention de les observer de loin, depuis ma forêt natale.
\subsection{Transition se repeuple}
Ça y est, les derniers Elfes sont partis de la cité. Elle est maintenant presque vide. Je crois que quelques humains en ont entendu parler. L'un d'entre eux, Thédétren, s'est entretenu par hasard avec un soldat Elfe. Ce dernier a attisé sa curiosité, et l'homme a posé la question à son chaman. Après quelques temps, une assemblée de mages humains à réussi à localiser la ville et à créer un passage magique vers l'endroit. Un groupe d'humains y a été envoyé, mené par Ermudor. Je dois dire que cet homme ne manque pas de courage. Le corps expéditionnaire a découvert une cité vide, et prête à être colonisée. Une fois un lien permanent créé, les premières auberges du monde ont été installées, la ville ne pouvant abriter de source permanente de nourriture. Comme d'habitude, personne ne s'est posé la moindre question à propos de la demeure de ce cher Démiurge.