\section{Note n\degre1568: Le pouvoir d'Arym}
\subsection{Introduction}
Cette note a pour but de présenter la vie d'Arym, un des derniers sorciers de ce monde, qui fut le premier dirigeant de la nouvelle société.
\subsection{La naissance}
Il est exactement 3 trois heures et quarante et une minutes. Un cri retentit dans la vieille."C'est un garçon" s'écrit la vieille femme. "Et ça sera 20 pièces". Le père, trop heureux pour réfléchir, tend la petite fortune que demande l'accoucheuse, qui disparait en quelques instants.
\newline
"Regardes Ana, comme il est beau". Aucune réponse. "Ana ?". Toujours rien. Il se retourne. Sa femme sourit. Elle semble dormir d'un doux sommeil. Mais quelque chose a changé. Son expression est trop calme, beaucoup trop calme. Aaron, grand homme au visage carré, n'a pas besoin de plus d'explications. Il pose délicatement son enfant dans son lit, et va fermer les yeux de la morte. Sans une larme.
\subsection{L'enfance}
"Papa, je peux aller voir Ray ? S'il te plaaaiiiitttt"
\newline
"Oui, vas-y Arym, mais rentre avant le couché du soleil."
\newline
"Oui papa, ne t'inquiète pas , je ne veux pas avoir à faire aux gobelins. Merci!"
\newline
Le petit Arym sortit de la vieille carcasse de bateau que constituait sa maison, gravit la dune qui le séparait de l'intérieure des terres, et courut vers les caravanes qui se profilaient à l'horizon. Le petit village de Dolokin était plus un agrégat de tôle et de parpaings qu'autre chose, mais une bonne dizaine de familles vivaient là. Elles trouvaient refuge près des côtes, loin des restes de la guerre.
\newline
Arym se précipita vers une caravane qui était autrefois jaune. La porte grinça, et il fut immédiatement acceuilli par deux "Salut Ary!", l'un provenant d'une vieille femme, sans doute la grand-mère de Ray, et l'autre de Ray lui même. 
\subsection{À la sortie}
Plus besoin de l'autorisation parentale pour aller voir Ray maintenant qu'il avait ses 17 ans. Il suivit comme d'habitude le petit sentier qui contournait la dune pour se rendre sur leur terrain de jeux favoris, les ruines d'une grande ville. Oui, Grahyrst fut son nom chez les humains, il y a fort longtemps. Sa carabine, cadeau là encore paternel, collait à son dos suant.
\newline
"Plus vite, plus vite". Plus vite, il fallait aller plus vite se disait Kurk, chaman d'une bande de gobelin. Je suis sûr qu'il y a de l'or là dessous! Nous pourrions vivre un an avec ce pactole! "AAAaaaarrrrkkk". Ils l'ont trouvé? Kurk bouscule ses petits protégés, en écrase deux, pour se précipiter vers l'origine de ces cris.