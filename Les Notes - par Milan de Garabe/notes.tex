\documentclass{book}
\usepackage[utf8]{inputenc}
\usepackage[francais]{babel}
\usepackage[babel=true,kerning=true]{microtype}
\usepackage[linkcolor=blue, colorlinks=true,]{hyperref}
\usepackage{graphicx}
\usepackage{fancyhdr}
\pagestyle{fancy}
\fancyhead[C]{\rightmark}
\fancyhead[L]{}
\fancyhead[R]{}
\fancyfoot[LO]{\hyperlink {participation} {Envie de participer?}}
\fancyfoot[RE]{\hyperlink {participation} {Envie de participer?}}

% Creare Mundum est sous licence CC-BY-SA, présente dans le dossier d'origine, merci de la respecter!
%
% Page mère regroupant les autres pour une meilleure coordination des participants (communiquez uniquement le fichier modifié)
% Pour tout nouveau participant, il est conseillé d'aller voir le tutoriel sur le LaTex du site du zero.

\title{Notes}
\author{Milan de Garabe}
\begin{document}
\maketitle
\setcounter{tocdepth}{2} %Génération du Sommaire.
\renewcommand{\contentsname}{Sommaire} 
\tableofcontents
%%%%%%%%%

\chapter{L'ère médiévale}
\section{Note n\degre236: La chute de Dolgarur}
\subsection{Je suis le Roi}
Pouah! Mauvaise nuit, comme toujours, je dors de plus en plus mal. Je crois que c'est cette histoire de filon d'or. Les prospecteurs repartent ce soir, ils devraient bien ramener quelque chose cette fois. C'est vrai que le manque d'or se fait de plus en plus sentir. Certains d’entre nous commencent à trouver le temps long... J'ai appris que Dwalan était partit dans les montagnes du sud. D'autre suivront, je le crains. Attendons le retour des prospecteurs, nous improviserons alors.
\newline
Toujours pas d'or. Je n'en peux plus. Il me manque. Il me réveille. Je ne sais plus quoi faire. La cité est au bord de la crise. Hier encore, plusieurs de mes sujets ont rejoins les rangs de Ketelundr. Ils le payeront.
\newline
Je ne dors plus. Mais ce n'ai pas grave. Je sais où est l'or! Ces maudits elfes! Ça ne peut être qu'eux! Je le savais! Ces immondes bestioles nous ont pris notre or! Mais ils ne s'en tireront pas comme ça... Cela fait trop longtemps que nous n'avons pas montré nos forces. Le sang vas couler. Je suis Dolgarur, roi sous les montagnes de Kazadren, et mon peuple entre en guerre.

\subsection{Je suis aussi Roi}
Les nains s'agitent en ce moment. Les éclaireurs rapportent de petits groupes d'entre eux se dirigeant vers le sud. Ce n'est pas bien. Quelque chose se passe.
\newline
La montagne gronde. Il y a bien longtemps que l'on avait pas entendu ça. Sous le règne même de mon père, cela ne s'était pas produit. Ainsi, pauvre fou, vous venez à notre rencontre.
\textit{Il faut savoir que si la "montagne gronde", selon les propres dire du roi Othond de la ligné de Néril, c'est sous l'effet du pas cadencé des armés naines se répercutant contre les parois des montagnes abruptes.} "Elenel, sonnez l'alarme, fermez les frontières, préparez-vous à la bataille" ordonnais-je sur le champs. Personne n'attaque la vieille forêt sans conséquences. Vieux nain, tu vas souffrir.

\subsection{Quand Throneye s'embrase}
Othond, roi des elfes, mis son casque millénaire, pris son arc d'if, ses flèches, sortit de sa loge qui dominait la forêt. Il observa. Les armées naines étaient bien là, en ligne, près à lancer l’attaque. Il ne les voyait pas, dans le noir de cette nuit sans lune, mais sentait leurs présences. Il leva le bras: les arcs se bandèrent.
\newline
Il le baissa, les arcs tirèrent.
\newline
Khori, sergent du 5ème bataillon, entendit plusieurs bruits secs. Une masse s'effondra à côté de lui. Étrange, il n'avait entendu aucun clairon sonnant le départ de la bataille. Il tourna la tête, lentement. Il mis du temps avant de se rendre compte que le corps qui gisait devant lui, une flèche entre les deux yeux, était celui de son ami. Il ne s'entendit pas hurler de rage. Un clairon nain, cette fois-ci retentit dans les hauteurs. L'attaque était sonnée. C'est avec plaisir qu'il arma la baliste, mis un carreau, l’enflamma, et tira, ainsi qu'une grande partie de l'armée. Le bois a un grand défaut: il brûle. Contrairement à l'or.
\newline
Depuis la lisière de la forêt, Othond vit plusieurs point rouge se dessiner dans le ciel. Des cris commencèrent à se faire entendre. Ils osaient mettre le feu. C'était inconcevable. Pas à cette forêt. Pas ici. Il pris une sorte de petit médaillon en bois qui pendait à son coup et le brisa. Un autre grondement se fit entendre, bien plus sourd que le pas des nains. D'étranges arbres se mirent à pousser autour du cœur de la forêt. Cette barrière là ne prendrai pas feu.
\newline
La bataille faisait rage. Les armées naines attaquaient, les armées elfes sortaient de leur bois en flamme. Seul l'observateur aguerrit pouvait voir depuis ce qui serrait bien plus tard Sombre-Cime, que le cœur de la forêt semblait étrangement résister au feu. Le mage verrait bien, quant à lui, et où qu'il soit, la profonde modification de l'ordre magique.

\subsection{Le roi est mort}
Les soldats elfes combattaient avec acharnement. Seul les quelques fuyards comprirent que le cœur de la forêt était fermé, mais ils étaient trop affolés pour avertir leurs congénères restés sur le front.
\newline
Taril arma pour une énième fois son arc, le banda, visa une masse informe, tira.
\newline
Je suis le roi! Vous avez notre or! Tiens mange de mon fer immonde elfe! Haha, tu ne m'auras pas avec cette technique. J'aime le son de tes os brisés. Quel est ce bruit sourd? Ma vision se perd, se couvre de rouge. Mon casque heurte le sol. Je ne tiens plus. Mon ventre...
\newline
"Le roi Dolgarur est touché, à l'aide, à l'aide" hurla un jeune aide de camps. Le meilleurs combattant nains entourèrent firent barrière de leur corps pour protéger leur souverain. Mais la nouvelle se répandit bien vite: le roi avait été touché.
\newline
Le sang du roi toucha le sol. La ligné de Dolgarur s'arrêta sur ces terres, semie-elfiques. Son sang abreuva les arbres, abreuva la nature, étancha la soif de ces lieux. 
\newline
"Le roi est mort, le roi est mort". Ce cri retentit sur tout le champs de bataille. Les nains redoublèrent leurs attaques. Ces elfes ne s'en tireraient pas.
\newline
Le mage observateur n'est pas déçu. Le sang d'un roi nain est puissant. Les arbres pourrirent en quelques minutes, le feu se stoppa net, des pierres fendirent le sol. Les elfes ne purent lutter. Il se firent massacrer sans merci.

\subsection{Épilogue}
Je reprends ma plume pour finir cette note qui m'a demandé beaucoup d'efforts. Tout d'abord, un mot sur Othond et sa garde. Leur cœur de la forêt étant fermé pour longtemps et constatant leur sous-nombres, les survivants partirent vers le sud, vers les plaines d'Aegnord.
\newline
Les nains, quant à eux, après avoir pleuré leur roi, rasèrent l'ancienne forêt, et creusèrent avidement les décombres à la recherche de l'or promit. Rien ne s'y trouvais, à part quelques pierres. Les nains rentrèrent donc chez eux déçut mais victorieux. Le nouveau roi nain prit quand même le soin de vendre cette terre devenue inutile aux humains, en plein expansionnisme.
\newline
Peu de gens savent ce qui s'est passé durant les cents ans que durèrent la prison du cœur de Thorneye, mais on raconte d'étranges choses à ce sujet. 

\chapter{L'Apocalypse}
\section{Note n\degre1375: L'apocalypse: récit général}
\subsection{Introduction}
Nous sommes en 3021, le monde est à son apogée technologique, mais de fortes tensions diplomatiques se font de plus en plus ressentir .
\subsection{La goutte d'eau}
Après une course à l'armement entre les différentes nations du monde, l'Empereur d'Anksfall, Eson Yrcert, décide de stopper les raids des pirates Elfes Marins et Orcs qui sévissent sur les côtes, et s'enhardissent de plus en plus. Les bateaux de guerre fraîchement construits s'ébranlent et des divisions entières sont mobilisées. Cependant, au lieu de la victoire rapide escomptée, les troupes humaines échouent dans leur entreprise de forcer les pirates à livrer bataille, et la guérilla meurtrière que choisissent de mener les pirates s'avère fatale aux humains. Les assauts sont menés là où les troupes sont les plus vulnérables, puis les assaillants disparaissent avant que les renforts n'aient le temps de s'organiser. Plusieurs frégates et autres navires lourds sont ainsi capturés ou coulés, les affrontements s'éternisent. C'est à ce moment que les nains de Kazadren, après des décennies passées à se préparer au fin fond de leurs montagnes, choisissent de sortir de leurs cités pour attaquer Anksfall, enfin plus précisément les banques de la cité. Les humains, affaiblis par la guerre qu'ils mènent contre les pirates, sont balayés par les armes de destruction massive déployées par les troupes naines. Le roi sous les montagnes de Thargelon, allié des humains, est forcé de réagir face à cette attaque, et de marcher contre les représentants de sa propre race. 
\subsection{Le monde s'embrase}
Les missiles sol-sol pleuvent sur la cité de Dren. Les armes surpuissantes utilisées par les belligérants causent rapidement un raz-de-marée de destruction qui déferle sur le continent. Les Orques, engagés comme mercenaires par les deux camps, ravagent, pillent et détruisent tout sur leur passage. Les elfes, pourtant neutres, subissent des attaques face auxquelles ils sont impuissants, ayant négligé de se préparer militairement. Les Elfes Gris sont rapidement exterminés, les derniers survivants des bombardements, condamnés à l'exil, se font massacrer aveuglément dès qu'ils croisent soldats ou pillards. Au contraire, les Elfes de la Terre opposent une résistance farouche, leurs mages de combat partent en campagne afin de défendre leurs terres. Cependant, même leur puissante magie est balayée par la technologie avancée des autres races, et ils sont massacrés par l'artillerie naine, rejoignant leurs cousins Gris dans la tombe. 
\newline
La cité de Transition est rapidement isolée du reste du monde, les portails magiques étant pour la plupart détruits lors d'affrontements au sein des cités qui les abritent. L'existence d'une ville située sur la Mer de Cristal et d'une entité magique supérieure deviennent rapidement des mythes.
\newline
Très vite, chaque humain et chaque nain est mobilisé, des cités entière sont rasées. Rinam et l'île entière sur laquelle la cité se trouve sont rayées de la carte par un bombardement intensif qui dure près de trois jours, ce qui entraîne la perte de la race des Elfes Marins. 
\subsection{Trahison ?}
Un évènement notable de cette guerre est la trahison d'une petite armée de mercenaires orques, se battant aux côtés des troupes de Kazadren. En effet, leur leader, Morkhar le puissant, renonce à combattre sous les ordres de ses employeurs et lance ses troupes dans une campagne de pillages et de massacres qui laisse une balafre sanglante sur le continent. Cette trahison impromptue désorganise Kazadren tout entier, ce qui permet aux alliés humains et nains d'enfoncer la ligne de front de plusieurs kilomètres.
\newline
Inquiets de la chute progressive de la race humaine, les dirigeants de la province au sud de Darkhaven décident de faire sécession et déclarent leur indépendance vis-à-vis de toute forme de pouvoir. Les difficultés rencontrées par leur ancienne patrie empêche cette dernière de mener une campagne de représailles, mais les raids orques subis par l'état naissant le réduisent presque à néant dans les semaines qui suivent son apparition.
\subsection{Explosion}
Enfin, les services de renseignements de Thargelon localisent les stocks de géranium grâce au désordre qui règne chez leurs ennemis et bombardent intensivement la zone, ce qui entraîne un cataclysme sans précédent. Le royaume de Kazadren est dévasté et sa population exterminée, à l'exception de quelques survivants qui subissent rapidement les radiations dégagées par la zone désormais inhabitable. Les armées naines, parties en campagne lors du cataclysme, se lancent dans une campagne de destruction aveugle et ravagent la moitié du royaume humain avant qu'une contre-offensive désespérée ne les stoppe. La guerre de positions qui s'ensuit saigne chaque armée aux quatre veines, mais les nains de Kazadren commencent à déserter en masse, et leur armée s'effiloche. 
\newline
Alors que la guerre se termine, la conjonction des dépenses de guerre, des ravages et des retombées radioactives cause une famine terrible qui achève de précipiter la civilisation vers son déclin. Les cultures pourrissent sur pied, les survivants de chaque race s'entredéchirent pour quelques miettes de nourriture, et le continent se retrouve de nouveau à feu et à sang. La période qui s'ensuit voit la culture des survivants se déliter et chaque population retomber dans un nouvel âge de pierre.

\chapter{L'ère post-apocalyptique}
\section{Note n\degre1568: Le pouvoir d'Arym}
\subsection{Introduction}
Cette note a pour but de présenter la vie d'Arym, un des derniers sorciers de ce monde, qui fut le premier dirigeant de la nouvelle société.
\subsection{La naissance}
Il est exactement 3 trois heures et quarante et une minutes. Un cri retentit dans la maison."C'est un garçon" s'écrit la vieille femme. "Et ça sera 20 pièces". Le père, trop heureux pour réfléchir, tend la petite fortune que demande l'accoucheuse, qui disparait en quelques instants.
\newline
"Regardes Ana, comme il est beau". Aucune réponse. "Ana ?". Toujours rien. Il se retourne. Sa femme sourit. Elle semble dormir d'un doux sommeil. Mais quelque chose a changé. Son expression est trop calme, beaucoup trop calme. Aaron, grand homme au visage carré, n'a pas besoin de plus d'explications. Il pose délicatement son enfant dans son lit, et va fermer les yeux de la morte. Sans une larme.
\subsection{L'enfance}
"Papa, je peux aller voir Ray ? S'il te plaaaiiiitttt"
\newline
"Oui, vas-y Arym, mais rentre avant le couché du soleil."
\newline
"Oui papa, ne t'inquiète pas , je ne veux pas avoir à faire aux skhos. Merci!"
\newline
Le petit Arym sortit de la vieille carcasse de bateau que constituait sa maison, gravit la dune qui le séparait de l'intérieure des terres, et courut vers les caravanes qui se profilaient à l'horizon. Le petit village de Dolokin était plus un agrégat de tôle et de parpaings qu'autre chose, mais une bonne dizaine de familles vivaient là. Elles trouvaient refuge près des côtes, loin des restes de la guerre.
\newline
Arym se précipita vers une caravane qui était autrefois jaune. La porte grinça, et il fut immédiatement acceuilli par deux "Salut Ary!", l'un provenant d'une vieille femme, sans doute la grand-mère de Ray, et l'autre de Ray lui même. 
\subsection{À la sortie}
Plus besoin de l'autorisation parentale pour aller voir Ray maintenant qu'il avait ses 17 ans. Il suivit comme d'habitude le petit sentier qui contournait la dune pour se rendre sur leur terrain de jeux favoris, les ruines d'une grande ville. Oui, Grahyrst fut son nom chez les humains, il y a fort longtemps. Sa carabine, cadeau là encore paternel, collait à son dos suant.
\newline
"Plus vite, plus vite". Plus vite, il fallait aller plus vite se disait Kurk, chaman d'une bande de skhos. Je suis sûr qu'il y a de l'or là dessous! Nous pourrions vivre un an avec ce pactole! "AAAaaaarrrrkkk". Ils l'ont trouvé? Kurk bouscule ses petits protégés, en écrase deux, pour se précipiter vers l'origine de ces cris.
\newline
Ray hurlait. Une bande de skhos venait de se jeter sur lui. Déjà, l'un d'entre eux avait sorti une dague (ou plutôt un boût de tôle mal affuté) et le menaçait.
\newline
Arym entendit les cris de Ray, il se précipita vers leur source, enjambant deux carcasses d'objets métalliques tout rouillés. Il se retrouva devant ce qui fut un grand batiment. Une pencarte indiquait "Supermarché", mais il ne savait pas ce que ça voulait dire. Vite. Il enjambât une barrière tombée, et contemplât son meilleur ami, ou e qu'il en restait. Son corps était réduit à l'état de charpie. Seule la tête était préservée. Et une expression d'horreur était à jamais figée sur son visage.
\subsection{Au retour}
Arym hurlât, lui aussi, mais pas de terreur. De rage, si ce mot convient encore. Son être tout entier était habité de colère. Pourquoi avoir tué Ray? Pourquoi? "POURQUOI ?". 
\newline
"Pourquoi quoi, Ary ?". Arym se retourna, et vit son ami tout sourire aux lèvres. "Regarde ce que j'ai trouvé !". Et il lui montra une superbe montre de l'ancien temps, qui marchait toujours. 

\end{document}