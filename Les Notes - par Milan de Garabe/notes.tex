\documentclass{book}
\usepackage[utf8]{inputenc}
\usepackage[francais]{babel}
\usepackage[babel=true,kerning=true]{microtype}
\usepackage[linkcolor=blue, colorlinks=true,]{hyperref}
\usepackage{graphicx}
\usepackage{fancyhdr}
\pagestyle{fancy}
\fancyhead[C]{\rightmark}
\fancyhead[L]{}
\fancyhead[R]{}
\fancyfoot[LO]{\hyperlink {participation} {Envie de participer?}}
\fancyfoot[RE]{\hyperlink {participation} {Envie de participer?}}

% Creare Mundum est sous licence CC-BY-SA, présente dans le dossier d'origine, merci de la respecter!
%
% Page mère regroupant les autres pour une meilleure coordination des participants (communiquez uniquement le fichier modifié)
% Pour tout nouveau participant, il est conseillé d'aller voir le tutoriel sur le LaTex du site du zero.

\title{Notes}
\author{Milan de Garabe}
\begin{document}
\maketitle
\setcounter{tocdepth}{2} %Génération du Sommaire.
\renewcommand{\contentsname}{Sommaire} 
\tableofcontents
%%%%%%%%%

\chapter{L'ère médiévale}
\section{Note n\degre236: La chute de Dolgarur}
\subsection{Je suis le Roi}
Pouah! Mauvaise nuit, comme toujours, je dors de plus en plus mal. Je crois que c'est cette histoire de filon d'or. Les prospecteurs repartent ce soir, ils devraient bien ramener quelque chose cette fois. C'est vrai que le manque d'or se fait de plus en plus sentir. Certains d’entre nous commencent à trouver le temps long... J'ai appris que Dwalan était partit dans les montagnes du sud. D'autre suivront, je le crains. Attendons le retour des prospecteurs, nous improviserons alors.
\newline
Toujours pas d'or. Je n'en peux plus. Il me manque. Il me réveille. Je ne sais plus quoi faire. La cité est au bord de la crise. Hier encore, plusieurs de mes sujets ont rejoins les rangs de Ketelundr. Ils le payeront.
\newline
Je ne dors plus. Mais ce n'ai pas grave. Je sais où est l'or! Ces maudits elfes! Ça ne peut être qu'eux! Je le savais! Ces immondes bestioles nous ont pris notre or! Mais ils ne s'en tireront pas comme ça... Cela fait trop longtemps que nous n'avons pas montré nos forces. Le sang vas couler. Je suis Dolgarur, roi sous les montagnes de Kazadren, et mon peuple entre en guerre.

\subsection{Je suis aussi Roi}
Les nains s'agitent en ce moment. Les éclaireurs rapportent de petits groupes d'entre eux se dirigeant vers le sud. Ce n'est pas bien. Quelque chose se passe.
\newline
La montagne gronde. Il y a bien longtemps que l'on avait pas entendu ça. Sous le règne même de mon père, cela ne s'était pas produit. Ainsi, pauvre fou, vous venez à notre rencontre.
\textit{Il faut savoir que si la "montagne gronde", selon les propres dire du roi Othond de la ligné de Néril, c'est sous l'effet du pas cadencé des armés naines se répercutant contre les parois des montagnes abruptes.} "Elenel, sonnez l'alarme, fermez les frontières, préparez-vous à la bataille" ordonnais-je sur le champs. Personne n'attaque la vieille forêt sans conséquences. Vieux nain, tu vas souffrir.

\subsection{Quand Throneye s'embrase}
Othond, roi des elfes, mis son casque millénaire, pris son arc d'if, ses flèches, sortit de sa loge qui dominait la forêt. Il observa. Les armées naines étaient bien là, en ligne, près à lancer l’attaque. Il ne les voyait pas, dans le noir de cette nuit sans lune, mais sentait leurs présences. Il leva le bras: les arcs se bandèrent.
\newline
Il le baissa, les arcs tirèrent.
\newline
Khori, sergent du 5ème bataillon, entendit plusieurs bruits secs. Une masse s'effondra à côté de lui. Étrange, il n'avait entendu aucun clairon sonnant le départ de la bataille. Il tourna la tête, lentement. Il mis du temps avant de se rendre compte que le corps qui gisait devant lui, une flèche entre les deux yeux, était celui de son ami. Il ne s'entendit pas hurler de rage. Un clairon nain, cette fois-ci retentit dans les hauteurs. L'attaque était sonnée. C'est avec plaisir qu'il arma la baliste, mis un carreau, l’enflamma, et tira, ainsi qu'une grande partie de l'armée. Le bois a un grand défaut: il brûle. Contrairement à l'or.
\newline
Depuis la lisière de la forêt, Othond vit plusieurs point rouge se dessiner dans le ciel. Des cris commencèrent à se faire entendre. Ils osaient mettre le feu. C'était inconcevable. Pas à cette forêt. Pas ici. Il pris une sorte de petit médaillon en bois qui pendait à son coup et le brisa. Un autre grondement se fit entendre, bien plus sourd que le pas des nains. D'étranges arbres se mirent à pousser autour du cœur de la forêt. Cette barrière là ne prendrai pas feu.
\newline
La bataille faisait rage. Les armées naines attaquaient, les armées elfes sortaient de leur bois en flamme. Seul l'observateur aguerrit pouvait voir depuis ce qui serrait bien plus tard Sombre-Cime, que le cœur de la forêt semblait étrangement résister au feu. Le mage verrait bien, quant à lui, et où qu'il soit, la profonde modification de l'ordre magique.

\subsection{Le roi est mort}
Les soldats elfes combattaient avec acharnement. Seul les quelques fuyards comprirent que le cœur de la forêt était fermé, mais ils étaient trop affolés pour avertir leurs congénères restés sur le front.
\newline
Taril arma pour une énième fois son arc, le banda, visa une masse informe, tira.
\newline
Je suis le roi! Vous avez notre or! Tiens mange de mon fer immonde elfe! Haha, tu ne m'auras pas avec cette technique. J'aime le son de tes os brisés. Quel est ce bruit sourd? Ma vision se perd, se couvre de rouge. Mon casque heurte le sol. Je ne tiens plus. Mon ventre...
\newline
"Le roi Dolgarur est touché, à l'aide, à l'aide" hurla un jeune aide de camps. Le meilleurs combattant nains entourèrent firent barrière de leur corps pour protéger leur souverain. Mais la nouvelle se répandit bien vite: le roi avait été touché.
\newline
Le sang du roi toucha le sol. La ligné de Dolgarur s'arrêta sur ces terres, semie-elfiques. Son sang abreuva les arbres, abreuva la nature, étancha la soif de ces lieux. 
\newline
"Le roi est mort, le roi est mort". Ce cri retentit sur tout le champs de bataille. Les nains redoublèrent leurs attaques. Ces elfes ne s'en tireraient pas.
\newline
Le mage observateur n'est pas déçu. Le sang d'un roi nain est puissant. Les arbres pourrirent en quelques minutes, le feu se stoppa net, des pierres fendirent le sol. Les elfes ne purent lutter. Il se firent massacrer sans merci.

\subsection{Épilogue}
Je reprends ma plume pour finir cette note qui m'a demandé beaucoup d'efforts. Tout d'abord, un mot sur Othond et sa garde. Leur cœur de la forêt étant fermé pour longtemps et constatant leur sous-nombres, les survivants partirent vers le sud, vers les plaines d'Aegnord.
\newline
Les nains, quant à eux, après avoir pleuré leur roi, rasèrent l'ancienne forêt, et creusèrent avidement les décombres à la recherche de l'or promit. Rien ne s'y trouvais, à part quelques pierres. Les nains rentrèrent donc chez eux déçut mais victorieux. Le nouveau roi nain prit quand même le soin de vendre cette terre devenue inutile aux humains, en plein expansionnisme.
\newline
Peu de gens savent ce qui s'est passé durant les cents ans que durèrent la prison du cœur de Thorneye, mais on raconte d'étranges choses à ce sujet. 

\chapter{L'Apocalypse}
\input{./src/1375}

\chapter{L'ère post-apocalyptique}
\section{Note n\degre1568: Le pouvoir d'Arym}
\subsection{Introduction}
Cette note a pour but de présenter la vie d'Arym, un des derniers sorciers de ce monde, qui fut le premier dirigeant de la nouvelle société.
\subsection{La naissance}
Il est exactement 3 trois heures et quarante et une minutes. Un cri retentit dans la vieille."C'est un garçon" s'écrit la vieille femme. "Et ça sera 20 pièces". Le père, trop heureux pour réfléchir, tend la petite fortune que demande l'accoucheuse, qui disparait en quelques instants.
\newline
"Regardes Ana, comme il est beau". Aucune réponse. "Ana ?". Toujours rien. Il se retourne. Sa femme sourit. Elle semble dormir d'un doux sommeil. Mais quelque chose a changé. Son expression est trop calme, beaucoup trop calme. Aaron, grand homme au visage carré, n'a pas besoin de plus d'explications. Il pose délicatement son enfant dans son lit, et va fermer les yeux de la morte. Sans une larme.
\subsection{L'enfance}
"Papa, je peux aller voir Ray ? S'il te plaaaiiiitttt"
\newline
"Oui, vas-y Arym, mais rentre avant le couché du soleil."
\newline
"Oui papa, ne t'inquiète pas papa, je ne veux pas avoir à faire aux gobelins. Merci!"
\newline
Le petit Arym sortit de la vieille carcasse de bateau que constituait sa maison, gravit la dune qui le séparait de l'intérieure des terres, et courut vers les caravanes qui se profilaient à l'horizon.

\end{document}