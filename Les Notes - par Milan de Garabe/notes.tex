\documentclass{book}
\usepackage[utf8]{inputenc}
\usepackage[T1]{fontenc}
\usepackage[francais]{babel}
\usepackage[linkcolor=blue, colorlinks=true,]{hyperref}
\usepackage{graphicx}
\usepackage{fancyhdr}
\pagestyle{fancy}
\fancyhead[C]{\rightmark}
\fancyhead[L]{}
\fancyhead[R]{}
\fancyfoot[LO]{\hyperlink {participation} {Envie de participer?}}
\fancyfoot[RE]{\hyperlink {participation} {Envie de participer?}}

\makeatletter
\let\insertdate\@date
\makeatother

% Creare Mundum est sous licence CC-BY-SA, présente dans le dossier d'origine, merci de la respecter!
%
% Page mère regroupant les autres pour une meilleure coordination des participants (communiquez uniquement le fichier modifié)
% Pour tout nouveau participant, il est conseillé d'aller voir le tutoriel sur le LaTex du site du zero.

\title{Notes}
\author{Milan de Garabe}
\date{\oldstylenums{\insertdate}}
\begin{document}
\maketitle
\setcounter{tocdepth}{2} %Génération du Sommaire.
\renewcommand{\contentsname}{Sommaire} 
\tableofcontents
%%%%%%%%%
\chapter{Introduction}
\section{Milan de Garabe, le créateur}
Milan de Garabe a créé le monde. Le but de ce dieu est de raconter une histoire, tout simplement d'écrire des épopées fantastiques. En voici quelques-unes qu'il fit parvenir dans son monde, et que l'on peut trouver dans la grande Bibliothèque.

\chapter{L'ère raciale}
\section{Note 52 : l'enfance d'Ermudor le Grand}
\subsection{Il est né}
Dieu, que cette époque était ennuyeuse. Heureusement que les Elfes se sont enfin décidés à réagir. Il m'aura fallu attendre toutes ces années sans rien à voir, ou à raconter... Enfin. Il est né. Ermudor. Cet enfant me semble promis à de grandes choses. Je sens même qu'il pourrait brusquement me donner des milliers d'histoires à raconter, des milliers de pages à écrire. Et tout ça dans un si petit être. Il respire depuis quelques heures seulement, et déjà j'espère de lui qu'il changera la face du monde. Mais il le fera, je le sais.

\subsection{Il progresse}
Ermudor est un enfant précoce. Il est jeune, et déjà il chasse mieux que beaucoup. Il attire la haine et la jalousie, mais aussi l'amour et le respect. Je pense que tout sera bientôt en place. Il me suffit d'attendre quelques années de plus. Oui, dans quelques années, tout sera prêt...

\subsection{Ce n'est pas juste}
Aujourd'hui, lors d'une partie de chasse, je me suis aventuré plus loin que jamais du village. J'ai vu quelque chose de surprenant. Je crois que c'étaient des Elfes. Ils étaient grands, agiles, et leurs manteaux brillaient. Leurs armes aussi avaient l'air très belles, et très tranchantes. Je me demande comment ils les font. Pourquoi ne nous l'ont-ils pas dit ? Pourquoi nous laissent-ils nous battre pour survivre, alors qu'ils on l'air si bien nourris et en bonne santé ? Peut-être qu'ils on peur de nous. Après tout, peut-être qu'il y a beaucoup d'hommes dans le monde. Il faut que je demande à Déothain, il doit le savoir. Je pense que les Elfes sont contents de nous voir souffrir. Peut-être qu'ils nous haïssent. Peut-être qu'un jour ils voudront nous battre pour toujours. Nous devons être préparés. J'en parlerai en rentrant.

\subsection{Ça y est}
Tout s'enclenche, Ermudor est en âge de se battre. Bien entendu, il est le chef du clan, l'inverse m'aurait surpris. il a préparé son peuple à se battre, il les a armés du mieux qu'il a pu et ils sont motivés. Je pense qu'ils se feront tailler en pièces à la moindre escarmouche. J'espère simplement qu'il apprendra assez vite. L'autre question que je me pose est le choix de sa première cible. Les Elfes, qu'il haït et jalouse, ou les Nalfein, le principal danger qu'a à craindre son peuple ? Je verrai bien.
\section{Note 58 : Transition se dépeuple}
\subsection{Les Elfes ont perdu}
La guerre contre les Nalfein est terminée. Ils sont vaincus, si l'on peut dire. Les Elfes et les Nains aussi ont été victimes de cette guerre. Ils sont presque anéantis, et prendront des décennies, voir des siècles à se remettre. En attendant, les humains vont prendre leur essor, comme je m'en étais douté. Ils commencent déjà à clamer des domaines aux abords des forêts Elfes, mais pour l'instant ma curiosité est braquée sur autre chose. Je pense que Transition va bientôt changer de nature et de population. Déjà quelques Elfes sont retournés à Thorneye, et la cité est plus calme qu'elle ne l'a été depuis longtemps.
\subsection{Nous allons partir}
Depuis la défaite de nos armée contre les Nalfein, je pense sans arrêt à retourner au palais. Elleanol a perdu ses saveurs depuis quelques temps. Nombreux ont été ceux parmi mes amis qui ont fait de même. De plus, aucun des poèmes que j'ai écrits récemment n'a la moindre valeur. Je ne suis capable que d'aligner des mots sans queue ni tête depuis des jours. Il faut à tout prix que je retrouve un cadre plus propice à l'écriture et à la détente. Oui, c'est décidé, je pars. Elleanol va connaître des changements, et j'ai la ferme intention de les observer de loin, depuis ma forêt natale.
\subsection{Transition se repeuple}
Ça y est, les derniers Elfes sont partis de la cité. Elle est maintenant presque vide. Je crois que quelques humains en ont entendu parler. L'un d'entre eux, Thédétren, s'est entretenu par hasard avec un soldat Elfe. Ce dernier a attisé sa curiosité, et l'homme a posé la question à son chaman. Après quelques temps, une assemblée de mages humains à réussi à localiser la ville et à créer un passage magique vers l'endroit. Un groupe d'humains y a été envoyé, mené par Ermudor. Je dois dire que cet homme ne manque pas de courage. Le corps expéditionnaire a découvert une cité vide, et prête à être colonisée. Une fois un lien permanent créé, les premières auberges du monde ont été installées, la ville ne pouvant abriter de source permanente de nourriture. Comme d'habitude, personne ne s'est posé la moindre question à propos de la demeure de ce cher Démiurge.

\chapter{L'ère médiévale}
\section{Note n\degre236: La chute de Dolgarur}
\subsection{Je suis le Roi}
Pouah! Mauvaise nuit, comme toujours, je dors de plus en plus mal. Je crois que c'est cette histoire de filon d'or. Les prospecteurs repartent ce soir, ils devraient bien ramener quelque chose cette fois. C'est vrai que le manque d'or se fait de plus en plus sentir. Certains d’entre nous commencent à trouver le temps long... J'ai appris que Dwalan était partit dans les montagnes du sud. D'autre suivront, je le crains. Attendons le retour des prospecteurs, nous improviserons alors.
\newline
Toujours pas d'or. Je n'en peux plus. Il me manque. Il me réveille. Je ne sais plus quoi faire. La cité est au bord de la crise. Hier encore, plusieurs de mes sujets ont rejoins les rangs de Ketelundr. Ils le payeront.
\newline
Je ne dors plus. Mais ce n'ai pas grave. Je sais où est l'or! Ces maudits elfes! Ça ne peut être qu'eux! Je le savais! Ces immondes bestioles nous ont pris notre or! Mais ils ne s'en tireront pas comme ça... Cela fait trop longtemps que nous n'avons pas montré nos forces. Le sang vas couler. Je suis Dolgarur, roi sous les montagnes de Kazadren, et mon peuple entre en guerre.

\subsection{Je suis aussi Roi}
Les nains s'agitent en ce moment. Les éclaireurs rapportent de petits groupes d'entre eux se dirigeant vers le sud. Ce n'est pas bien. Quelque chose se passe.
\newline
La montagne gronde. Il y a bien longtemps que l'on avait pas entendu ça. Sous le règne même de mon père, cela ne s'était pas produit. Ainsi, pauvre fou, vous venez à notre rencontre.
\textit{Il faut savoir que si la "montagne gronde", selon les propres dire du roi Othond de la ligné de Néril, c'est sous l'effet du pas cadencé des armés naines se répercutant contre les parois des montagnes abruptes.} "Elenel, sonnez l'alarme, fermez les frontières, préparez-vous à la bataille" ordonnais-je sur le champs. Personne n'attaque la vieille forêt sans conséquences. Vieux nain, tu vas souffrir.

\subsection{Quand Throneye s'embrase}
Othond, roi des elfes, mis son casque millénaire, pris son arc d'if, ses flèches, sortit de sa loge qui dominait la forêt. Il observa. Les armées naines étaient bien là, en ligne, près à lancer l’attaque. Il ne les voyait pas, dans le noir de cette nuit sans lune, mais sentait leurs présences. Il leva le bras: les arcs se bandèrent.
\newline
Il le baissa, les arcs tirèrent.
\newline
Khori, sergent du 5ème bataillon, entendit plusieurs bruits secs. Une masse s'effondra à côté de lui. Étrange, il n'avait entendu aucun clairon sonnant le départ de la bataille. Il tourna la tête, lentement. Il mis du temps avant de se rendre compte que le corps qui gisait devant lui, une flèche entre les deux yeux, était celui de son ami. Il ne s'entendit pas hurler de rage. Un clairon nain, cette fois-ci retentit dans les hauteurs. L'attaque était sonnée. C'est avec plaisir qu'il arma la baliste, mis un carreau, l’enflamma, et tira, ainsi qu'une grande partie de l'armée. Le bois a un grand défaut: il brûle. Contrairement à l'or.
\newline
Depuis la lisière de la forêt, Othond vit plusieurs point rouge se dessiner dans le ciel. Des cris commencèrent à se faire entendre. Ils osaient mettre le feu. C'était inconcevable. Pas à cette forêt. Pas ici. Il pris une sorte de petit médaillon en bois qui pendait à son coup et le brisa. Un autre grondement se fit entendre, bien plus sourd que le pas des nains. D'étranges arbres se mirent à pousser autour du cœur de la forêt. Cette barrière là ne prendrai pas feu.
\newline
La bataille faisait rage. Les armées naines attaquaient, les armées elfes sortaient de leur bois en flamme. Seul l'observateur aguerrit pouvait voir depuis ce qui serrait bien plus tard Sombre-Cime, que le cœur de la forêt semblait étrangement résister au feu. Le mage verrait bien, quant à lui, et où qu'il soit, la profonde modification de l'ordre magique.

\subsection{Le roi est mort}
Les soldats elfes combattaient avec acharnement. Seul les quelques fuyards comprirent que le cœur de la forêt était fermé, mais ils étaient trop affolés pour avertir leurs congénères restés sur le front.
\newline
Taril arma pour une énième fois son arc, le banda, visa une masse informe, tira.
\newline
Je suis le roi! Vous avez notre or! Tiens mange de mon fer immonde elfe! Haha, tu ne m'auras pas avec cette technique. J'aime le son de tes os brisés. Quel est ce bruit sourd? Ma vision se perd, se couvre de rouge. Mon casque heurte le sol. Je ne tiens plus. Mon ventre...
\newline
"Le roi Dolgarur est touché, à l'aide, à l'aide" hurla un jeune aide de camps. Le meilleurs combattant nains entourèrent firent barrière de leur corps pour protéger leur souverain. Mais la nouvelle se répandit bien vite: le roi avait été touché.
\newline
Le sang du roi toucha le sol. La ligné de Dolgarur s'arrêta sur ces terres, semie-elfiques. Son sang abreuva les arbres, abreuva la nature, étancha la soif de ces lieux. 
\newline
"Le roi est mort, le roi est mort". Ce cri retentit sur tout le champs de bataille. Les nains redoublèrent leurs attaques. Ces elfes ne s'en tireraient pas.
\newline
Le mage observateur n'est pas déçu. Le sang d'un roi nain est puissant. Les arbres pourrirent en quelques minutes, le feu se stoppa net, des pierres fendirent le sol. Les elfes ne purent lutter. Il se firent massacrer sans merci.

\subsection{Épilogue}
Je reprends ma plume pour finir cette note qui m'a demandé beaucoup d'efforts. Tout d'abord, un mot sur Othond et sa garde. Leur cœur de la forêt étant fermé pour longtemps et constatant leur sous-nombres, les survivants partirent vers le sud, vers les plaines d'Aegnord.
\newline
Les nains, quant à eux, après avoir pleuré leur roi, rasèrent l'ancienne forêt, et creusèrent avidement les décombres à la recherche de l'or promit. Rien ne s'y trouvais, à part quelques pierres. Les nains rentrèrent donc chez eux déçut mais victorieux. Le nouveau roi nain prit quand même le soin de vendre cette terre devenue inutile aux humains, en plein expansionnisme.
\newline
Peu de gens savent ce qui s'est passé durant les cents ans que durèrent la prison du cœur de Thorneye, mais on raconte d'étranges choses à ce sujet. 

\chapter{L'ère post-apocalyptique}
\section{Note n\degre1568: Le pouvoir d'Arym}
\subsection{Introduction}
Cette note a pour but de présenter la vie d'Arym, un des derniers sorciers de ce monde, qui fut le premier dirigeant de la nouvelle société.
\subsection{La naissance}
Il est exactement 3 trois heures et quarante et une minutes. Un cri retentit dans la maison."C'est un garçon" s'écrit la vieille femme. "Et ça sera 20 pièces". Le père, trop heureux pour réfléchir, tend la petite fortune que demande l'accoucheuse, qui disparait en quelques instants.
\newline
"Regardes Ana, comme il est beau". Aucune réponse. "Ana ?". Toujours rien. Il se retourne. Sa femme sourit. Elle semble dormir d'un doux sommeil. Mais quelque chose a changé. Son expression est trop calme, beaucoup trop calme. Aaron, grand homme au visage carré, n'a pas besoin de plus d'explications. Il pose délicatement son enfant dans son lit, et va fermer les yeux de la morte. Sans une larme.
\subsection{L'enfance}
"Papa, je peux aller voir Ray ? S'il te plaaaiiiitttt"
\newline
"Oui, vas-y Arym, mais rentre avant le couché du soleil."
\newline
"Oui papa, ne t'inquiète pas , je ne veux pas avoir à faire aux skhos. Merci!"
\newline
Le petit Arym sortit de la vieille carcasse de bateau que constituait sa maison, gravit la dune qui le séparait de l'intérieure des terres, et courut vers les caravanes qui se profilaient à l'horizon. Le petit village de Dolokin était plus un agrégat de tôle et de parpaings qu'autre chose, mais une bonne dizaine de familles vivaient là. Elles trouvaient refuge près des côtes, loin des restes de la guerre.
\newline
Arym se précipita vers une caravane qui était autrefois jaune. La porte grinça, et il fut immédiatement acceuilli par deux "Salut Ary!", l'un provenant d'une vieille femme, sans doute la grand-mère de Ray, et l'autre de Ray lui même. 
\subsection{À la sortie}
Plus besoin de l'autorisation parentale pour aller voir Ray maintenant qu'il avait ses 17 ans. Il suivit comme d'habitude le petit sentier qui contournait la dune pour se rendre sur leur terrain de jeux favoris, les ruines d'une grande ville. Oui, Grahyrst fut son nom chez les humains, il y a fort longtemps. Sa carabine, cadeau là encore paternel, collait à son dos suant.
\newline
"Plus vite, plus vite". Plus vite, il fallait aller plus vite se disait Kurk, chaman d'une bande de skhos. Je suis sûr qu'il y a de l'or là dessous! Nous pourrions vivre un an avec ce pactole! "AAAaaaarrrrkkk". Ils l'ont trouvé? Kurk bouscule ses petits protégés, en écrase deux, pour se précipiter vers l'origine de ces cris.
\newline
Ray hurlait. Une bande de skhos venait de se jeter sur lui. Déjà, l'un d'entre eux avait sorti une dague (ou plutôt un boût de tôle mal affuté) et le menaçait.
\newline
Arym entendit les cris de Ray, il se précipita vers leur source, enjambant deux carcasses d'objets métalliques tout rouillés. Il se retrouva devant ce qui fut un grand batiment. Une pencarte indiquait "Supermarché", mais il ne savait pas ce que ça voulait dire. Vite. Il enjambât une barrière tombée, et contemplât son meilleur ami, ou e qu'il en restait. Son corps était réduit à l'état de charpie. Seule la tête était préservée. Et une expression d'horreur était à jamais figée sur son visage.
\subsection{Au retour}
Arym hurlât, lui aussi, mais pas de terreur. De rage, si ce mot convient encore. Son être tout entier était habité de colère. Pourquoi avoir tué Ray? Pourquoi? "POURQUOI ?". 
\newline
"Pourquoi quoi, Ary ?". Arym se retourna, et vit son ami tout sourire aux lèvres. "Regarde ce que j'ai trouvé !". Et il lui montra une superbe montre de l'ancien temps, qui marchait toujours. 

\newpage
\section{Participation au projet}
\subsection{Comment?}
\hypertarget{participation}{}
Creare Mundum vous a plu? 
Envie de partager vos scénarios, vos suggestions ou vos idées?
\newline
Contactez-nous sur la mailing list: \href {mailto:crearemundum@lists.tuxfamily.org}{crearemundum@lists.tuxfamily.org}
\newline
Ou rendez vous sur notre site: \href {http://creare-mundum.tuxfamily.org/} {http://creare-mundum.tuxfamily.org/}
\subsection{Les créateurs}
Voici la liste de ceux qui ont participé au projet Creare Mundum. Leur aide fut, est et sera toujours très précieuse au projet. Merci encore!  
\begin{itemize}
\item Alexandre ’Nobrakal’ Moine 
\item Philippe ’Tymophil’ Aubé 
\item Eliott ’Sulf’ Filippi
\item Émiland ’Derec’ Garrabé
\item Luc H.
\end{itemize}
\subsection{Licence}
Creare Mundum est un projet libre de droit, publié sous la licence Creative Commons BY-SA. C'est à dire que quiconque a la possibilité d'utiliser ce document (ainsi que toute les autres parties du projet), de le redistribuer et de le modifier. La seule obligation est de redistribuer le contenu (modifié ou non) sous les mêmes conditions.
\end{document}