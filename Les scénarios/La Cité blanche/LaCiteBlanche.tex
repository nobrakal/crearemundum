\documentclass[a4paper, 11pt]{article}
\usepackage[utf8]{inputenc}
\usepackage[T1]{fontenc}
\usepackage[francais]{babel}
\usepackage[linkcolor=blue, colorlinks=true,]{hyperref}
\usepackage{graphicx}
\usepackage{fancyhdr}
\pagestyle{fancy}
\fancyhead[C]{\rightmark}
\fancyhead[L]{}
\fancyhead[R]{}
\fancyfoot[C]{\hyperlink {participation} {Envie de participer?}}

\makeatletter
\let\insertdate\@date
\makeatother

% Creare Mundum est sous licence CC-BY-SA, présente dans le dossier d'origine, merci de la respecter!

\title{La Cité blanche \\ Scénario pour Creare Mundum \\ Devel}
\author{The Creare Mundum Project}
\date{\oldstylenums{\insertdate}}
\begin{document}
\maketitle
\setcounter{tocdepth}{1} %Génération du Sommaire.
\renewcommand{\contentsname}{Sommaire} 
\tableofcontents
\newpage

\section{Une lettre sur la table}
Les héros arrivent dans une grande pièce, totalement nue, avec en son centre une table. Sur cette table, une lettre. Voici ce qu'elle dit.
\begin{quotation}
Bonjour,
\newline
Vous avez eu une démonstration de mes pouvoirs avec l'ordre que vous a donné le duc de Leheath d’enquêter sur cette étrange maison, ma demeure, et vous aurez l'occasion d'en voir encore plus. Mais je ne puis tout faire, et c'est là que vous entrez en jeu. J'ai besoin de personnes telles que vous pour quelques missions. Acceptez-vous de vous mettre à mon service?
\newline
Je saurais vous récompenser, bien sûr. Considérez comme une avance ces petites choses. Je crois qu'elles ont grande valeur en ce monde.
\newline
M.G.
\end{quotation}
En effet, sur la table se trouve un nombre de petites pierres noires (il s'agit de pierres d'âmes) égal à celui de héros. Elles sont pour l'instant encastrée dans la table. Cela ressemble à de la puissante magie. Si les héros acceptent, des nouvelles lignes apparaissent sur la lettre:
\begin{quotation}
Vous  avez fait le bon choix. Votre première tâche sera d'aller déloger un groupe de gobelin qui sévissent dans les environs de Dren. Ces derniers prévoient une attaque d'un des avants poste de la ville, Kurok, alors que les gardes chassent un autre petit groupe de l'autre côté de la montagne. Vous avez une journée pour vous rendre sur les lieux.
\end{quotation}
Les pierres sont alors libérées de leur socle. Voici leurs effets respectifs:
\begin{enumerate}
\item La force du personnage augmente d'un quart
\item Aucun nouvel effet en apparence, mais le personnage sais lire les messages codés/cachés
\item Le personnage sais parler une nouvelle langue de son choix
\item Le personnage augmente ses chances de toucher d'un quart
\item Le personnage augmente ses points de vie d'un quart
\end{enumerate}
Une porte se dresse devant les héros, gravées de mystérieux symboles.
\chapter{L'arrivée dans la cité blanche}
\section{Wwwaaaahhh}
Nos héros, ouvrent la mystérieuse porte. Ils se retrouvent pleine ville. \emph{Nous conseillons au maitre de jeu d'utiliser une musique très particulière (ex: A. Piazolla: Las cuatro estaciones porteñas) qui permet une identification profonde de la ville}
Ils peuvent déambuler dans la ville, acheter équipements et autres potions. Si ils se renseignent sur leurs pierres d'âmes
\subsection{Pour vendre des pierres d'âmes}
Transition centralise tout le commerce de pierres d'âmes. "Chez Thierry", "l'Animus" ou encore "La Pierre noire" sont autant de boutiques réputées. Une pierre d'âme vaut très cher (environ le prix de deux rubis de la même taille), et les marchands tenteront évidement d'arnaquer nos pauvres héros.
\subsection{Pour assimiler des pierres d'âmes}

\newpage
\section{Participation au projet}
\subsection{Comment?}
\hypertarget{participation}{}
Creare Mundum vous a plu? 
Envie de partager vos scénarios, vos suggestions ou vos idées?
\newline
Contactez-nous sur la mailing list: \href {mailto:crearemundum@lists.tuxfamily.org}{crearemundum@lists.tuxfamily.org}
\newline
Ou rendez vous sur notre site: \href {http://creare-mundum.tuxfamily.org/} {http://creare-mundum.tuxfamily.org/}
\subsection{Les créateurs}
Voici la liste de ceux qui ont participé au projet Creare Mundum. Leur aide fut, est et sera toujours très précieuse au projet. Merci encore!  
\begin{itemize}
\item Alexandre ’Nobrakal’ Moine 
\item Philippe ’Tymophil’ Aubé 
\item Eliott ’Sulf’ Filippi
\item Émiland ’Derec’ Garrabé
\item Luc H.
\end{itemize}
\subsection{Licence}
Creare Mundum est un projet libre de droit, publié sous la licence Creative Commons BY-SA. C'est à dire que quiconque a la possibilité d'utiliser ce document (ainsi que toute les autres parties du projet), de le redistribuer et de le modifier. La seule obligation est de redistribuer le contenu (modifié ou non) sous les mêmes conditions.
\end{document}