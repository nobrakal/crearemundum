\documentclass[a4paper]{article}
\usepackage[utf8]{inputenc}
\usepackage[T1]{fontenc}
\usepackage[french]{babel}
\usepackage[linkcolor=blue, colorlinks=true,]{hyperref}
\usepackage{graphicx}
\usepackage{fancyhdr}
\pagestyle{fancy}
\fancyhead[C]{\rightmark}
\fancyhead[L]{}
\fancyhead[R]{}
\fancyfoot[C]{\hyperlink {participation} {Envie de participer?}}

\makeatletter
\let\insertdate\@date
\makeatother

% Creare Mundum est sous licence CC-BY-SA, présente dans le dossier d'origine, merci de la respecter!

\title{Élémentaire, mon cher\\ Scénario pour Creare Mundum \\ Devel}
\author{The Creare Mundum Project}
\date{\oldstylenums{\insertdate}}
\begin{document}
\maketitle
\setcounter{tocdepth}{1} %Génération du Sommaire.
\renewcommand{\contentsname}{Sommaire} 
\tableofcontents
\newpage

%----------------------------------------------------------------------------
\section*{TRAME GÉNÉRALE}
Idée principale: Présence d'élémentaires d'eau
\newline
Lieu: Autour de la mer intérieure.

Synopsis: Un sorcier (Ator ? elfe marin) itinérant fait le tour de la mer intérieure, invoque des élémentaires d'eau, et pille les villages alentours. -> Trafic d'esclaves !!!!!?????
Pourquoi ? Que pour l'argent ? Commanditaire ? Seul ?
-> Religion, hérétique -> Nie le Démiurge, prie un autre Dieu, "Itar"?

TRAFIC D'ESCLAVE: les persos sont enlevés et trainés comme esclaves, vendus aux orcs -> échappent après combat ou ruse.
\newline
Les persos vont devoir ensuite vaincre ces élémentaires, reçoivent une demande du duc de Leheath -> elfes

Début ? Les persos sont attaqués, alors qu'ils se reposent? -> simple mais efficace

Pourquoi les élémentaires se battent ? Ne prennent part à aucun conflit humain normalement...
%----------------------------------------------------------------------------


\section{Introduction pour le maitre de jeu}
"Élémentaire, mon cher" est [TODO: description et synopsis]
\newline
Cette intrigue prend place dans l'univers de Creare Mundum, durant l'ère médiévale. Pour plus de précisions concernant le monde, consultez le fichier principal.
\newline
Creare Mundum n'a pas de système de règle propre, il reste donc au maitre de jeu le travail d'adapter ce scénario au système qu'il souhaite utiliser, notamment e ce qui concerne la création des monstres.
\newline
Les paragraphes commençant par "Partie supplémentaire: " sont comme leur nom l'indique, hors de l'intrigue principale, mais permettent une plus grande immersion et un allongement du temps de jeu.
\newline
Les instructions et le déroulement qui suivent sont bien sûr donné uniquement à titre indicatif, n'hésitez pas à modifier cette aventure comme bon vous semble!

\section{L'Orage gronde}
\subsection{Petites vacances à Leheath}
Nos héros prennent des vacances bien méritées après leurs grandes aventures dans la cité balnéaire de Leheath (ou y vivent). 
Il fait vraiment bon y vivre, surtout en cet été 734. La ville n'a connu aucun conflit depuis la fin de l'empire humain, 200 ans plus tôt.
\newline
Les personnages vivent dans un petite location, au sud de la ville, non loin du centre ville, et surtout à quelques mètres de la plage (chose fort pratique pour les 30\degre saisonniers).

\subsection{Des rumeurs courent}
Faite vivre les personnages quelques instants dans un monde insouciant. Par une telle chaleur, il fait bon aller sur la plage, ou se rafraichir au bar... Laissez-les gambader à leur aise. Quoiqu'il en soit, au bout d'un certain temps, l'idéal étant un début de soirée, l'orage se met à rugir au loin. 
\newline
Le personnage le plus fin remarque une anomalie assez vite: Bien qu'il fasse lourd, il n'y a aucun nuage, le ciel est pur, et l'on voit les étoiles. Quelques minutes se passent. Puis le tocsin se met à sonner. Des murmures emplissent l'endroit dans lequel sont les personnages:
\begin{itemize}
\item "Cette fois-ci, c'est notre tour, il le fallait bien..."
\item "Nnnnoooonnn, pas ça ! "
\item "Que voulez-vous faire quand ce sont les éléments eux-mêmes qui se rebellent ? "
\end{itemize}
Pourtant, rien ne se profile à l'horizon...

\subsection{Enlèvement}
Des bruits d'eaux, des éclaboussements se font entendre de plus en plus près des joueurs. Au coin d'une rue apparait une sorte d'énorme siphon. Des yeux exercés perçoivent une forme humanoïde à l'intérieure, et n'importe quel elfe un peu éduqué sais directement ce qui s’abat sur la ville: des élémentaires d'eau. Quelques secondes après, les siphon se stoppent, rendant ainsi clairement une forme humanoïde aux élémentaires, qui se mettent à entrer dans les maisons, et à en vider les richesses.
\newline
Que les personnages essayent de s'interposer ou non, les élémentaires les saisissent, et les emportent sur un bateau situé dans le port de la ville.

%----------------------------------------------------------------------------

\newpage
\section{Participation au projet}
\subsection{Comment?}
\hypertarget{participation}{}
Creare Mundum vous a plu? 
Envie de partager vos scénarios, vos suggestions ou vos idées?
\newline
Contactez-nous sur la mailing list: \href {mailto:crearemundum@lists.tuxfamily.org}{crearemundum@lists.tuxfamily.org}
\newline
Ou rendez vous sur notre site: \href {http://creare-mundum.tuxfamily.org/} {http://creare-mundum.tuxfamily.org/}
\subsection{Les créateurs}
Voici la liste de ceux qui ont participé au projet Creare Mundum. Leur aide fut, est et sera toujours très précieuse au projet. Merci encore!  
\begin{itemize}
\item Alexandre ’Nobrakal’ Moine 
\item Philippe ’Tymophil’ Aubé 
\item Eliott ’Sulf’ Filippi
\item Émiland ’Derec’ Garrabé
\item Luc H.
\end{itemize}
\subsection{Licence}
Creare Mundum est un projet libre de droit, publié sous la licence Creative Commons BY-SA. C'est à dire que quiconque a la possibilité d'utiliser ce document (ainsi que toute les autres parties du projet), de le redistribuer et de le modifier. La seule obligation est de redistribuer le contenu (modifié ou non) sous les mêmes conditions.
\end{document}
