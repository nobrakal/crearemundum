\section{La magie}
La magie est une partie intégrante de nombreux jeu de rôle, et il convient de la traiter d'une façon appropriée.
Nous allons nous baser sur la magie présente dans CreareMundum, mais il est très simple de changer de "base magique".

\subsection{La magie élémentaire}
La magie la plus primaire, et qui conviendra à tout apprenti magicien. C'est dur, c'est brutal, et ça fait mal.
\subsubsection{L'attaque}
Globalement, la magie élémentaire suit les règles de dommage des armes à distance. Voici un tableau d'équivalence entre les deux.
\newline
\newline
\begin{tabular}{l|l|l}
   Rang & Sort & Flèche correspondante \\
   \hline
   4 & Boule d'énergie & Flèche de fer \\
   3 & Flash énergétique & Flèche d'acier \\
   2 & Rayon énergétique & Flèche orc \\
   1 & Supra-rayon énergétique & Flèche elfique \\
\end{tabular}
\newline
\subsubsection{La défense}
La magie élémentaire sert aussi à se défendre. Il existe 4 rang de défense, annulant les attaques de même rang.
\newline
\newline
\begin{tabular}{l|l|l}
   Rang & Sort & Sort annulé \\
   \hline
   4 & Protection énergétique  & Boule d'énergie \\
   3 & Bouclier énergétique  & Flash énergétique \\
   2 & Mur d'énergie & Rayon énergétique \\
   1 & Rempart d'énergie & Supra-rayon énergétique \\
\end{tabular}
\newline
\newline
Les sorts de rangs supérieurs annulent les dégâts des sorts inférieurs. La réciproque est par contre totalement fausse. Se protéger avec un sort de rang inférieur à l'attaque est simplement inutile. La protection vole en éclat sans enlever le moindre dégât. 