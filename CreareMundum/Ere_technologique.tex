\documentclass{book}
\usepackage[utf8]{inputenc}
\usepackage[T1]{fontenc}
\usepackage[francais]{babel}
\usepackage[linkcolor=blue, colorlinks=true,]{hyperref}
\usepackage{xcolor}
\usepackage{graphicx}
\usepackage{fancyhdr}
\pagestyle{fancy}
\fancyhead[C]{\rightmark}
\fancyhead[L]{}
\fancyhead[R]{}
\fancyfoot[LO]{\hyperlink {participation} {Envie de participer?}}
\fancyfoot[RE]{\hyperlink {participation} {Envie de participer?}}
\newcommand{\cadre}[1]{\newline\newline\colorbox{lightgray}{\begin{minipage}{\linewidth}#1\end{minipage}}}

\makeatletter
\let\insertdate\@date
\makeatother

% Creare Mundum est sous licence CC-BY-SA, présente dans le dossier d'origine, merci de la respecter!
%
% Page mère regroupant les autres pour une meilleure coordination des participants (communiquez uniquement le fichier modifié)
% Pour tout nouveau participant, il est conseillé d'aller voir le tutoriel sur le LaTex du site du zero.

\title{Creare Mundum \\ L'Ère technologique\\Devel}
\author{The Creare Mundum Project}
\date{\oldstylenums{\insertdate}}
\begin{document}
\maketitle
\setcounter{tocdepth}{2} %Génération du Sommaire.
\renewcommand{\contentsname}{Sommaire} 
\tableofcontents

\setcounter{part}{3}
\part{L'ère technologique}
\chapter{Histoire et Géographie technologique}
\section{Introduction}
An 2500. Les différents peuples sont arrivés à l'apogée de leurs savoirs-faire. Le monde est globalement en paix, sous l'égide humaine.
\section{Les Terres Humaines}
\subsection{Sombre-Cime}
\subsubsection{Description}
La population de la cité s'est de plus en plus rapprochée des mauriacs, la plupart des humains ayant choisi l'exode face au racisme croissant exprimé par les habitants du royaume voisin. La technologie de la cité est donc très proche de celle des mauriacs. La cité s'éloigne de plus en plus des autres cités humaines, et a maintenant des relations plus que froide avec Anksfall. On peut par exemple citer le conflit des 30 jours, durant lequel Sombre-Cime fut sous blocus de la part de l'imposante cité humaine d'Anksfall.
\subsection{Anksfall}
\subsubsection{Description}
La ville est assez avancée technologiquement, principalement dans le domaine des armes et de la production d'énergie. La découverte d'explosifs bon marché et faciles à produire a permis de créer des armes à feu très modernes et puissantes. C'est le fleuron de la technologie humaine.
\subsubsection{Direction politique}
Anksfall se réclame encore et toujours capitale de l'empire humain, et un empereur y vis toujours. Ce dernier na cependant plus aucun pouvoir en dehors des frontières de la cité et des plaines avoisinantes. Seuls les cités de Grahyrst en en moinde mesure Leheath conservent des liens puissants avec Anksfall, notamment économiques.
\subsection{Grahysrt}
\subsubsection{Description}
Les luttes de pouvoir incessantes qui ont pris place dans la cité ont lourdement freiné son économie, mais cette dernière commence à se relever et ses chantiers navals se remettent en marche, l'industrie de la cité étant de nouveau entièrement tournée vers la production de navires. On murmure que les chantiers navals tournent à plein régime et que des navires de guerre dotés des dernières innovations techniques commencent à sortir de la cité. Cette dernière est très proche du pouvoir central d'Anksfall, et la reconnaissance de l'empereur d'Anksfall par la cité-état revient régulièrement sur la table des dirigeants de la ville.
\subsubsection{Direction politique}
La direction de la ville est assurée par un conseil de marchands, qui reignent ici en rois. À tour de rôle, ces derniers se répartissent l'exécutif.
\subsection{Leheath}
\subsubsection{Description}
La cité est restée relativement en marge du developpement scientifique, et n'a pour ainsi dire fait aucune découverte majeure récemment. Cependant, elle reste la capitale des plaisirs, et ne se prive pas de profiter des innovations techniques de ses voisins... À sa manière.
\subsubsection{Direction politique}
Ici, c'est le prince qui dirige. Dans une vie de débauche et de luxures délicieuses, ce dernier assure l'exécutif de la cité-état. Il est heureusement secondé par nombre de conseillers, qui assurent le reste de la politique de l'état.
\subsubsection{Le solarium}
Il s'agit du joyau de la cité. Une immense baie vitrée, où reigne en permanence 35\degre. Transats, bar et autres piscines se succèdent sur quelques kilomètres, assurant des vacances inégallées sur tout le monde humain.
\section{Thargelon}
\subsection{Ketelundr}
\subsubsection{Description}
Progressivement, les mauriacs de ce royaume se sont ouverts au monde et leurs relations avec les humains se sont améliorées. Ce regain de l'entente locale a avantagé les mauriacs de Thargelon puisque la cité est devenue plus puissante que jamais et connaît son âge d'or.
\subsubsection{Technologie}
La technologie de ce royaume nain est assez orthodoxe, bien que très avancée. Les ingénieurs de Thargelon sont passés maîtres dans l'art de la miniaturisation, si bien que la plupart des mauriacs du royaume délaissent les taches manuelles au profit d'une grande compétence en ingénierie. Cela ne se fait pas sans problème, et les nains ont parfois recours à l'utilisation de procédés douteux pour subvenir à leur besoin en main d’œuvre. L'armée, quant à elle, préfère les affrontements à distance utilisant drones et missiles à un combat rapproché. Les générateurs de la cité, merveilles technologiques, utilisent les forces naturelles et les convertissent en énergie utilisable au quotidien.
\section{Kazadren}
\subsection{Dren}
\subsubsection{Description}
Au fur et à mesure du temps, les mauriacs de Kazadren se sont repliés sur eux-mêmes, et un racisme croissant envers toutes les autres races les a entraînés à haïr les mauriacs de Thargelon qui acceptent de commercer avec le reste du monde.
\subsubsection{Le géranium}
Un nouveau minerai découvert dans les montagnes voisines de la cité est désormais utilisé par les ingénieurs mauriacs: ils s'agit du géranium (Certains lui préfèreront sa dénomination scientifique de "Môr-ladron"). Ce métal est utilisé pour produire des armes dévastatrices mais à la manipulation très risquée. Ce Môr-ladron peut également servir de source d'énergie car il dégage une chaleur intense. Cette énergie peut être utilisée pour alimenter des véhicules par exemple. Des décennies d'expérimentation, ainsi que de nombreux décès, ont été nécessaires afin de mettre au point ces armes, et de nombreux mauriacs ont trouvé la mort en testant ce genre d'engins. Récemment, les mines de géranium ont commencé à fourmiller d'activité, et les usines tournent en permanence à produire un débit ahurissant de technologie létale.
\section{Le désert Orque}
\subsection{Mauhagr}
\subsubsection{Description}
Les Orques possèdent peu de technologie, se contentant de piller celle des humains lorsque l'occasion se présente. Ils sont depuis longtemps repliés sur eux-mêmes, organisant des raids de temps en temps.
\section{La faille Skhos}
\subsection{Vrag}
\subsubsection{Description}
Les Skhos étant de grands marchands, Vrag a connu un important développement et est progressivement devenue une plaque tournante du commerce mondial. Cette situation satisfait grandement la mafia locale, car elle n'hésite pas à taxer à volonté le commerce international qui transite par Vrag. Correctement avancés technologiquement, les skhos privilégient le commerce pour se fournir ce dont ils ont besoin. Ils sont peu orientés vers les armes, et seront sans doute les premières victimes collatérales si une guerre éclate.
\subsubsection{Direction politique}
Vrag, la capitale skhossienne, est dirigée par sa mafia, du moins officieusement. Il existe bien un grand roi skhos, mais ce dernier ne fait pas un pas sans l'assentiment des barons de la mafias locale.
\section{L'eau et la terre}
\subsection{Rinam}
\subsubsection{Description}
Les Dalcés, de plus en plus tournés vers la piraterie et la violence, augmentent la fréquence et l'intensité de leurs attaques pour alimenter leurs velléités expansionnistes. Ils méprisent le combat à distance, privilégiant une approche furtive suivie d'un assaut foudroyant. C'est pour cette raison que leurs véhicules légers, leurs armes de poing et de corps-à-corps sont aussi réputées que convoitées.
\subsubsection{Direction politique}
Rinam demeure une démocratie, et en est très fière ! Le président ne cesse d'ailleurs de mettre en oeuvre un "soft-power" redoutable, et on peut observer dans toutes les villes possédant un régime autortaire, notamment les villes humaines, la naissance de cellules républicaines.
\subsection{Thorneye}
\subsubsection{Description}
Les Alvénis bruns renient et méprisent toute technologie. Un prétendu guide spirituel affermit son emprise sur la société et la race entière périclite, s'accrochant désespérément aux derniers restes de sa civilisation.
\subsection{Aegnord} 
\subsubsection{Description}
Les Alvénis Gris ont vite compris qu'ils pourraient tirer profit de la technologie. Alors que certains d'entre eux restent méfiants par nature, d'autre se vouent corps et âme à la recherche sur les Intelligences Artificielles (IA). Certains d'entre eux devinrent même addict à l'utilisation de gadgets technologiques et basèrent leur vie sur leur utilisation. Leur civilisation commença à dépérir lentement mais sûrement et la technologie sur laquelle ils se reposaient signa leur perte, après des années de prospérité et de leadership dans le domaine de la technologie et de l'IA.


\newpage
\section{Participation au projet}
\subsection{Comment?}
\hypertarget{participation}{}
Creare Mundum vous a plu? 
Envie de partager vos scénarios, vos suggestions ou vos idées?
\newline
Contactez-nous sur la mailing list: \href {mailto:crearemundum@lists.tuxfamily.org}{crearemundum@lists.tuxfamily.org}
\newline
Ou rendez vous sur notre site: \href {http://creare-mundum.tuxfamily.org/} {http://creare-mundum.tuxfamily.org/}
\subsection{Les créateurs}
Voici la liste de ceux qui ont participé au projet Creare Mundum. Leur aide fut, est et sera toujours très précieuse au projet. Merci encore!  
\begin{itemize}
\item Alexandre ’Nobrakal’ Moine 
\item Philippe ’Tymophil’ Aubé 
\item Eliott ’Sulf’ Filippi
\item Émiland ’Derec’ Garrabé
\item Luc H.
\end{itemize}
\subsection{Remerciements}
Voilà maintenant la liste de ceux qui ont contribué à plus petite mesure, et souvent sans le savoir. 
\begin{itemize}
\item Dan Scott, pour l’image d’arodef.
\item Les artistes du MMORPG Ryzom, pour l'habitation de Thorneye ainsi que la dague d'Aegnord (modifiée). 
\end{itemize}
\subsection{Licence}
Creare Mundum est un projet libre de droit, publié sous la licence Creative Commons BY-SA. C'est à dire que quiconque a la possibilité d'utiliser ce document (ainsi que toute les autres parties du projet), de le redistribuer et de le modifier. La seule obligation est de redistribuer le contenu (modifié ou non) sous les mêmes conditions.
\end{document}

