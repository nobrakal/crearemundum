\section{L'Empire}
\subsection{Sombre-Cime}
\subsubsection{Description}
La population de la cité s'est de plus en plus rapprochée des nains, la plupart des humains ayant choisi l'exode face au racisme croissant exprimé par les habitants du royaume nain voisin. La technologie de la cité est donc très proche de celle des nains. Il serait même question d'une sécession avec l'empire.
\subsection{Anksfall}
\subsubsection{Description}
La ville est assez avancée technologiquement, principalement dans le domaine des armes et de la production d'énergie. La découverte d'explosifs bon marché et faciles à produire a permis de créer des armes à feu très modernes et puissantes.
\subsection{Grahysrt}
\subsubsection{Description}
Les luttes de pouvoir incessantes qui ont pris place dans la cité on lourdement freiné son économie, mais elle commence à se relever et ses chantiers navals se remettent en marche, l'industrie de la cité étant de nouveau entièrement tournée vers la production de navires. On murmure que les chantiers navals tournent à plein régime et que des navires de guerre dotés des dernières innovations techniques commencent à sortir de la cité...
\section{Thargelon}
\subsection{Ketelundr}
\subsubsection{Description}
Progressivement, les nains de ce royaume se sont ouverts au monde et leurs relations avec les humains se sont améliorées. Cette amélioration de l'entente locale a avantagé les nains de Thargelon puisque la cité est devenue plus puissante que jamais et connaît son âge d'or.
\newline
La technologie de ce royaume nain est assez orthodoxe, bien que très avancée. Les ingénieurs de Thargelon sont passés maîtres dans l'art de la miniaturisation, si bien que la plupart des nains du royaume délaissent les taches manuelles au profit d'une grande compétence en ingénierie. L'armée, quant à elle, préfère les affrontements à distance utilisant drones et missiles à un combat rapproché. Les générateurs de la cité, merveilles technologiques, utilisent les forces naturelles et les convertissent en énergie utilisable au quotidien.
\section{Kazadren}
\subsection{Dren}
\subsubsection{Description}
Au fur et à mesure du temps, les nains de Kazadren se sont repliés sur eux-mêmes, et un racisme croissant envers toutes les autres races les a entraînés à haïr les nains de Thargelon qui acceptent de commercer avec le reste du monde.
\subsubsection{Le géranium}
Un nouveau minerai découvert dans les montagnes voisines de la cité est désormais utilisé par les ingénieurs nains: ils s'agit du géranium (Certains lui préfèreront sa dénomination scientifique de "Môr-ladron"). Ce métal est utilisé pour produire des armes dévastatrices mais à la manipulation très risquée et peut également servir de source d'énergie car il dégage une chaleur intense. Cette énergie peut être utilisée pour alimenter des véhicules par exemple. Des décennies d'expérimentation ont été nécessaires afin de mettre au point ces armes, et de nombreux nains ont trouvé la mort en testant ce genre d'engins. Récemment, les mines de géranium ont commencé à fourmiller d'activité, et les usines tournent en permanence à produire un débit ahurissant de technologie létale.
\section{Le désert Orque}
\subsection{Mauhagr}
\subsubsection{Description}
Les Orques possèdent peu de technologie, se contentant de piller celle des humains lorsque l'occasion se présente. Ils sont depuis longtemps repliés sur eux-mêmes, organisant des raids à l'occasion.
\section{La faille Gobeline}
\subsection{Vrag}
\subsubsection{Description}
Vrag, la capitale gobeline, est dirigée par sa mafia, du moins officieusement. Les Gobelins étant de grands marchands, Vrag a connu un important développement et est progressivement devenue une plaque tournante du commerce mondial. Cette situation satisfait grandement la mafia locale, car elle n'hésite pas à taxer à volonté le commerce international. Correctement avancés technologiquement, les gobelins privilégient le commerce pour se fournir ce dont ils ont besoin. Ils sont peu orientés vers les armes, et seront sans doute les premières victime collatérales si une guerre éclate.
\section{L'eau et la terre}
\subsection{Rinam}
\subsubsection{Description}
Les Dalcés, de plus en plus tournés vers la piraterie et la violence, augmentent la fréquence et l'intensité de leurs attaques pour alimenter leurs velléités expansionnistes. Ils méprisent le combat à distance, privilégiant une approche furtive suivie d'un assaut foudroyant. C'est pour cette raison que leurs véhicules légers, leurs armes de poing et de corps-à-corps sont aussi réputées que convoitées.
\subsection{Thorneye}
\subsubsection{Description}
Les Alvénis de la Terre renient et méprisent toute technologie. Un prétendu guide spirituel affermit son emprise sur la société et la race entière périclite, s'accrochant désespérément aux derniers restes de sa civilisation.
\subsection{Aegnord} 
\subsubsection{Description}
Les Alvénis Gris ont vite compris qu'ils pourraient tirer profit de la technologie. Alors que certains d'entre eux restent méfiants par nature, d'autre se vouent corps et âme à la recherche sur les Intelligences Artificielles (IA). Certains d'entre eux devinrent même accro à l'utilisation de gadgets technologiques et basèrent leur vie sur leur utilisation. Leur civilisation commença à dépérir lentement mais sûrement et la technologie sur laquelle ils se reposaient signa leur perte, après des années de prospérité et de leadership dans le domaine de la technologie et de l'IA.
