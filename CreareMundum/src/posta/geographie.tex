\section{Territoires à influences humaines}
\subsection{Anksfall}
\subsubsection{Description}
La guerre n'a pas épargné l'ancienne capitale humaine. Elle n'est plus qu'un monceau de débris, sans forme. La nature reprends petit à petits le dessus: plusieurs quartiers ont été inondés, d'autres devenus forêts.
\subsubsection{Vie quotidienne}
La haute-ville est elle bien habitée par des humains, retournés au clanisme. Ils vivent dans les anciens buildings, qui leur assurent une certaine protection. Le verritable problème est l'accès à l'eau potable, et c'est une lutte quotidienne que de ne serais-ce que satisfaire ses besoins vitaux.

\subsection{Tanok (ancienne Sombre-Cime)}
\subsubsection{Description}
La ville a été totalement rasée dans le conflit. Cependant, dans les débris, les barbares du nord ont commencés à s'établir, à reconstruire presque un semblant de cité. Il s'agit désormais d'un endroit assez violent, mais qui bénificie d'une certaine stabilité, si l'on oublie les querelles internes.

\subsection{Les territoires dévastés}
\subsubsection{Description}
Les anciennes plaines de sang sont devenues stériles suite à une sur-exposition au géranium. Pire, aucun être vivant ne réussi à s'y implanter durablement. La terre y est ici véritablement nocive, et on rcconte que boire de l'eau de ces terres condamne a une mort lente et douloureuse.
\newline
Tout ceci a pour conséquence une certaine division des zones humaines: le nord est durablement séparé du sud, et il n'existe aucune route qui relie désormais la zone d'Anksfall à celle des montagnes de Kazadren.

\subsection{Grahyrst}
\subsubsection{Description}
La ville a été relativement épargnée par les bombardements. Actuellement, les chefs de guerre, décendents des généraux de l'armée humaine, reignent en maitres sur la ville. Ces derniers se sont rendus maitre de l'eau, et l'utilisent pour faire des raids sur certains villages de pécheurs de la mer intérieure.

\section{La faille Skhos}
\subsection{Vrag}
\subsubsection{Description}
Les skhos ont particulièrement bien résisté à l'apocalypse. Vrag aurait presque prospéré vu la manne financière liée à l'armement qui s'y était developpée. Maintenant, la vie est quand même nettement plus difficile, car plus aucune matière première n'est disponible. Les skhos locaux se sont donc reconvertis dans le troc de récoltes, et constituent la plaque tournante des quelques denrées alimentaires qui sont échangées sur ce monde.

\section{L'Eau et la Terre}
\subsection{Leheath}
\subsubsection{Description}
La ville est désormais inhabitée. Seuls subsistent les ruines de la décadence humaine.

\subsection{Natskhaz}
\subsubsection{Histoire}
\hypertarget{natskhaz}{}L'ancienne forêt de Thorneye a été rasé par une gigantesque explosion.
Rien n'en est restée.
Maintenant, une grande plaine totalement stérile a prit la place.
Seules une faune et une flore très rustiques réussissent à survivre.

\subsection{Darkhaven}
\subsubsection{Description}
Drakhaven est devenu le refuge d'une communauté humaine que l'on connait sous le nom de "charbonniers". Ils vivent exclusivement des ressources de la forêt, et n'ont que peu de contact avec l'exterieur.
\subsubsection{Organisation politique}
Les charbonniers vivent en anarchie. Une démocratie directe s'y exerce, masculine, pour les plus de 20 ans.
