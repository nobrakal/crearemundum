\section{Récit historique}
\subsection{Introduction}
Nous sommes en 3021, le monde est à son apogée technologique, mais de fortes tensions diplomatiques se font de plus en plus ressentir .
\subsection{La goutte d'eau}
Après une course à l'armement entre les différentes nations du monde, l'Empereur d'Anksfall, Eson Yrcert, décide de stopper les raids des pirates Dalcés et Orcs qui sévissent sur les côtes, et s'enhardissent de plus en plus. Les bateaux de guerre fraîchement construits s'ébranlent et des divisions entières sont mobilisées. Cependant, au lieu de la victoire rapide escomptée, les troupes humaines échouent dans leur entreprise de forcer les pirates à livrer bataille, et la guérilla meurtrière que choisissent de mener les pirates s'avère fatale aux humains. Les assauts sont menés là où les troupes sont les plus vulnérables, puis les assaillants disparaissent avant que les renforts n'aient le temps de s'organiser. Plusieurs frégates et autres navires lourds sont ainsi capturés ou coulés, les affrontements s'éternisent. C'est à ce moment que les mauriacs de Kazadren, après des décennies passées à se préparer au fin fond de leurs montagnes, choisissent de sortir de leurs cités pour attaquer Anksfall, enfin plus précisément les banques de la cité. Les humains, affaiblis par la guerre qu'ils mènent contre les pirates, sont balayés par les armes de destruction massive déployées par les troupes naines. Le roi sous les montagnes de Thargelon, allié des humains, est forcé de réagir face à cette attaque, et de marcher contre les représentants de sa propre race. 
\subsection{Le monde s'embrase}
Les missiles sol-sol pleuvent sur la cité de Dren. Les armes surpuissantes utilisées par les belligérants causent rapidement un raz-de-marée de destruction qui déferle sur le continent. Les Orques, engagés comme mercenaires par les deux camps, ravagent, pillent et détruisent tout sur leur passage. Les alvénis, pourtant neutres, subissent des attaques face auxquelles ils sont impuissants, ayant négligé de se préparer militairement. Les alvénis Gris sont rapidement exterminés, les derniers survivants des bombardements, condamnés à l'exil, se font massacrer aveuglément dès qu'ils croisent soldats ou pillards. Au contraire, les alvénis de la Terre opposent une résistance farouche, leurs mages de combat partent en campagne afin de défendre leurs terres. Cependant, même leur puissante magie est balayée par la technologie avancée des autres races, et ils sont massacrés par l'artillerie naine, rejoignant leurs cousins Gris dans la tombe. 
\newline
La cité de Transition est rapidement isolée du reste du monde, les portails magiques étant pour la plupart détruits lors d'affrontements au sein des cités qui les abritent. L'existence d'une ville située sur la Mer de Cristal et d'une entité magique supérieure deviennent rapidement des mythes.
\newline
Très vite, chaque humain et chaque nain est mobilisé, des cités entière sont rasées. Rinam et l'île entière sur laquelle la cité se trouve sont rayées de la carte par un bombardement intensif qui dure près de trois jours, ce qui entraîne la perte de la race des Dalcés. 
\subsection{Trahison ?}
Un évènement notable de cette guerre est la trahison d'une petite armée de mercenaires orques, se battant aux côtés des troupes de Kazadren. En effet, leur leader, Morkhar le puissant, renonce à combattre sous les ordres de ses employeurs et lance ses troupes dans une campagne de pillages et de massacres qui laisse une balafre sanglante sur le continent. Cette trahison impromptue désorganise Kazadren tout entier, ce qui permet aux alliés humains et mauriacs d'enfoncer la ligne de front de plusieurs kilomètres.
\newline
Inquiets de la chute progressive de la race humaine, les dirigeants de la province au sud de Darkhaven décident de faire sécession et déclarent leur indépendance vis-à-vis de toute forme de pouvoir. Les difficultés rencontrées par leur ancienne patrie empêche cette dernière de mener une campagne de représailles, mais les raids orques subis par l'état naissant le réduisent presque à néant dans les semaines qui suivent son apparition.
\subsection{Explosion}
Enfin, les services de renseignements de Thargelon localisent les stocks de géranium grâce au désordre qui règne chez leurs ennemis et bombardent intensivement la zone, ce qui entraîne un cataclysme sans précédent. Le royaume de Kazadren est dévasté et sa population exterminée, à l'exception de quelques survivants qui subissent rapidement les radiations dégagées par la zone désormais inhabitable. Les armées naines, parties en campagne lors du cataclysme, se lancent dans une campagne de destruction aveugle et ravagent la moitié du royaume humain avant qu'une contre-offensive désespérée ne les stoppe. La guerre de positions qui s'ensuit saigne chaque armée aux quatre veines, mais les mauriacs de Kazadren commencent à déserter en masse, et leur armée s'effiloche. 
\newline
Alors que la guerre se termine, la conjonction des dépenses de guerre, des ravages et des retombées radioactives cause une famine terrible qui achève de précipiter la civilisation vers son déclin. Les cultures pourrissent sur pied, les survivants de chaque race s'entredéchirent pour quelques miettes de nourriture, et le continent se retrouve de nouveau à feu et à sang. La période qui s'ensuit voit la culture des survivants se déliter et chaque population retomber dans un nouvel âge de pierre.
