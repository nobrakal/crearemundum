%Caractéristiques types d'un paragraphe:
% \subsection{TitreDuParagrape}
% \hypertarget {titreduparagraphe} (en minuscule et sans accent)
% Votre texte
% Merci de retourner à la ligne à chaque phrase. Cela n'influencera pas la présentation de votre texte et permettra une meilleure lisibilité
%Un lien intra document se fait ainsi: \hyperlink {lenomdepuaragraphe (tel que déclaré via hypertarget)}{le texte à afficher}
% Merci de votre participation à Creare Mundum ;)

\section{Les âmes}
Les âmes sont des sortes de clés produites par le Démiurge, qui permettent d'appréhender son monde d'une autre manière. Grâce à cette nouvelle vision, le héros va acquérir de meilleures capacités physiques ou psychiques. Les âmes sont, sous leur formes physiques, de petites pierres d'un noir profond. Le commerce d'âmes se pratique que dans la ville de Transition.
\subsection{Exemples d’âmes}
Par exemple, les âmes peuvent conférer :
\begin{itemize}
\item Une rapidité accrue
\item Une vision accrue
\item Une plus grande force
\item Une plus grande intelligence
\item L'apprentissage de nouveaux sorts
\item La capacité de parler à certains animaux
\end{itemize}
Le maitre de jeu est libre d'inventer de nouveaux pouvoirs, la liste n'est proposée qu'a titre indicatif.
\subsection{Assimilation des âmes}
Pour que le propriétaire utilise réellement son âme, il doit fusionner avec elle. À l'aide d'une phrase d'incantation ( “Kaparushan palkinon Viha” ). La petite pierre noire va alors disparaître en une sorte de gaz de la même couleur va pénétrer dans le sujet (par les yeux, les oreilles, la bouche). Cette nouvelle vision du monde est dure à supporter, le héros va mettre du temps à s'en remettre et a totalement appréhender son âme. On fonctionnera ici tel l’expérience.
\subsection{Abandonner son âme}
Il est impossible d'abandonner une âme que l'on a assimilé. À la mort du propriétaire, son âme disparait avec lui.