%Caractéristiques types d'un paragraphe:
% \subsection{TitreDuParagrape}
% \hypertarget {titreduparagraphe} (en minuscule et sans accent)
% Votre texte
% Merci de retourner à la ligne à chaque phrase. Cela n'influencera pas la présentation de votre texte et permettra une meilleure lisibilité
%Un lien intra document se fait ainsi: \hyperlink {lenomdepuaragraphe (tel que déclaré via hypertarget)}{le texte à afficher}
% Merci de votre participation à Creare Mundum ;)

\section{Les guildes}
\subsection{La guilde des voleurs}
\hypertarget{laguildedesvoleurs}{}Cette guilde de l'ombre est très répandue dans le monde.
Son but officiel est de voler aux riches et de le redonner aux pauvres, mais le peuple en doute et se méfie de ses membres.
Elle est dirigée par 8 grands voleurs. Ce sont les voleurs les plus talentueux que l'on puisse trouver.
Ce poste est aussi à vie. Il est impossible d'en démissionner, sous peine de mort.
La guilde des voleurs possède aussi des salles de communication.
\subsection{La guilde des magiciens}
\hypertarget {laguildedesmagiciens}{}Cette guilde est la guilde officielles des magiciens, que tout bon sorcier se doit d’incorporer.
Leur université se trouve à \hyperlink {grahyrst} {Grahyrst}.
La guilde tolère tous les types de magie, bien que la Nécromancie soit interdite, peu des ses pratiquants sont réprimandés.
La guilde dispose de succursale dans toute les grandes villes.
C'est dans ces succursales qu'il faut aller pour toutes demandes de sort.
\subsection{La guilde des guerriers}
\hypertarget {laguildedesguerriers}{}Cette guilde regroupe tous les guerriers.
Pour y rentrer, il suffit de n'avoir commis aucun délit reconnu, et de se présenter dans une succursale.
Comme les autres guildes, elle est bien représentée dans tous le pays.
Il faut s'y rendre pour des demandes de protections, d'entrainement, etc.
\subsection{La guilde de l'ombre}
\hypertarget{laguildedelombre}{}Cette guilde regroupe toute les pratiques interdites.
Fortement réprimandé par l'empereur, elle est très discrète.
Personne n'en sait et ne veut en savoir plus.
