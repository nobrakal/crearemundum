\section{Les Humains}
\subsection{Caractéristiques physiques}
Les humains sont sont assez différents des ce côté-là. Il peuvent mesurer de 1m6o à 2m10. Les humains du nord de l’empire, du fait de croisement avec les nains, sont de plus petite taille. Par contre, les humains de sud de l’empire, qui connaissent mieux les orques et les elfes, sont relativement plus grand. La plupart des personnes ont les yeux et les cheveux d’un brun profond. Les humains ne tendent pas à avoir un corps athlétique naturellement, bien que la plupart s’entraine rigoureusement dans ce but. 
\subsection{Us et coutumes}
\subsubsection{Mariage}
Les hommes se marient vers l’âge de vingt ans avec des jeunes filles de quinze ans. Ils se font souvent par intérêt financier, la marié amenant sa fameuse dot. Les mariages ’amoureux’ sont assez mal vu, car synonyme la plupart du temps de pauvreté. 
\subsubsection{Héritage}
Les biens se transmettent de père en fils (ou fille s’y il n’y a pas de fils), par testament. La loi oblige le testamentaire à signer de son propre sang, pour assurer son consentement. 
\subsubsection{Repas}
Les humains prennent leur repas à heure fixe : 7h le matin, 11h le midi et 19h le soir. Ils mangent relativement copieusement, même les pauvres. C’est une chose sacré dans leur tradition. Mais il est vrai que les nordiques affectionnent encore plus les repas, étant indispensable à leur survis en extérieur. Ils y vouent un véritable culte et il serait impensable ne serait-ce que d'arriver en retard à un des diner. 
\subsubsection{Hiérarchie}
Les hommes apprécient beaucoup être classé selon leur place dans la société. Le plus haut rang est celui d’empereur, puis vient les titres de noblesses (dans l’ordre : Duc, Marquis, Comte, Vicomte et Baron). Ensuite vient l’armée, très appréciée et respectée dans l’empire. Puis apparait la bourgeoisie. Là, l’ordre se fait en fonction de la richesse, et des relations (les deux n’étant pas souvent séparés). Enfin, le gros de la masse, le peuple, où comme pour la bourgeoisie, la place se fait en fonction de la fortune. Certains se trouvent en dessous du peuple, comme les bourreaux, les repris de justice et les vagabonds. 
\subsubsection{Relations aux autres peuples}
Les humains sont très accueillant et ouvert aux autres peuples et coutumes. Tous les postes leurs sont accordées, sauf dans la noblesse (encore qu’il y ai quelques exceptions). 

\section{Les Nains}
\subsection{Caractéristiques physiques}
Les nains sont de petits êtres, leur taille varie entre 80 centimètre et 1 mètre 50. Ils portent tous la barbe, c'est une grande distinction. Les nains sont très musclés et fort. 

\section{Les Gobelins}
\subsection{Caractéristiques physiques}
Les gobelins sont des créatures pouvant mesurer de 95 cm à 1m45, bien que ceux des montagnes soit généralement plus petits et ceux des plaines (ces derniers usent aussi plus de leur bipédie, afin de voir plus loin). Ils possèdent une peau différente selon l’habitat d’origine: ocre pour le désert, vert, brun, marron pour les forêt et les plaines grise, noire pour les terrains rocheux ou blanche pour les lieux enneigés. Les gobelins s'adaptent étonnamment bien à ces différents milieux. Ils ont néanmoins tous des yeux d'un orange vif.
\subsection{Métissage}
Certains gobelins se sont mélangés avec d'autres races. Ce métissage est très mal vu et les “balzaf”, comme ils sont appelés, sont souvent expulsés des communautés goblines.
\subsection{Organisation}
Il existe deux types de sociétés goblines, très différentes, et il est fortement déconseillé de prendre les uns pour les autres.
\subsection{Les gobelins sauvages}
Les gobelins sauvages, qui ont un comportement tribal. Leurs sociétés varient, mais on retrouve généralement une forme clanique. Ces clans sont souvent isolés, mais ont la particularité d'être très agressif, et quelque soit la créature adverse. Ils se font souvent la guerre entre eux, mais il arrive qu'ils attaquent des petits villages humains isolés.
\subsection{Les gobelins civilisés}
Les gobelins civilisés vivent en grande communauté, en grande majorité à Vrag. Leur intelligence supérieur à celle de leurs confrères provient d'expériences magiques pratiquées par des sorcier aux abords de l'actuel faille gobline, il y a bien longtemps. Leur dirigeant est d'ailleurs par tradition un sorcier.
Ils souffrent cependant qu’on les y associe. Les gobelins civilisés possèdent des lois strictement établies.
\subsubsection{Hiérarchie}
Tout en haut de l'échelle se trouve le roi-sorcier, puis... Plus rien. Les gobelins civilisés ne reconnaissent officiellement que ce grand chef. Il s'agit après d'une hiérarchie officieuse, surtout basée sur la renommée marchande.
\subsubsection{Le commerce}
Quelque soit leurs origines, les gobelins vouent un grand culte à ce qui est pour eux un art: le commerce.

\section{Les elfes marins}
\subsection{Hiérarchie}
\subsubsection{Le président}
