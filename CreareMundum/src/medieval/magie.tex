% Caractéristiques types d'un paragraphe:
% \subsection{TitreDuParagrape}
% \subsubsection{Description}
% \hypertarget {titreduparagraphe} (en minuscule et sans accent)
% Votre texte
% \subsubsection{Personalités} (si besoin est)
% Merci de retourner à la ligne à chaque phrase. Cela n'influencera pas la présentation de votre texte et permettra une meilleure lisibilité
% Un lien intra document se fait ainsi: \hyperlink {lenomdepuaragraphe (tel que déclaré via hypertarget)}{le texte à afficher}
% Merci de votre participation à Creare Mundum ;)

\section{Magie}
\subsubsection{Introduction}
La Magie se présente sous trois grandes formes:
\begin{itemize}
\item L'Énergie, la magie originelle, forme pure de la puissance du démiurge.
\item Les 4 éléments, une évolution de la magie originelle, synthétisé dans le vent, la terre, l'eau et le feu. 
\item La lumière/ les ténèbres, utilise les règles de création du démiurge.
\end{itemize}
Chaque personnage devra choisir lors de sa création le type de magie qu'il souhaite pratiquer.
\subsection{L'Énergie}
L'Énergie est la forme de magie la plus pure. Elle se présente sous forme d'une très forte lumière. Son utilisation est assez basique, mais puissante. 
Une attaque se fait en concentrant sa pensé sur l'objet/l'ennemi que l'on veut atteindre.
L'énergie peut aussi servir à se protéger des attaques magiques. Il suffit alors de se concentrer sur l'objet qui menace. Nous parlerons alors de 'dés-énergie' car deux rayons énergétique s'annihilent. 
\subsubsection{Les sorts}
Les sorts peuvent s'acheter auprès de marchand ou être trouver dans des donjons. Ils se présentent sous forme de parchemin, que le joueur peut apprendre. Concrètement, ils apprennent une nouvelle façon de concentrer son esprit.
\subsubsection{Pour le MJ}
Pour incorporer la magie énergétique dans votre campagne, rien de plus simple. Son utilisation est totalement similaire à celle des armes de distance (arc/arbalète). Utilisez les mêmes règles pour les armes de distances que pour la magie énergétique. Les munitions sont quant à elles fixés par rapport au nombre de flèches disponible dans un carquois classique. 
Pour la dés-énergie, le sort adverse est tout simplement annulé (sauf en cas de très gros écart de niveau. Dans ce cas là, les dégâts seront justes réduit). Le joueur doit par contre pouvoir voir l'attaque, et dire qu'il s'en prémunit.
\subsection{Les 4 éléments}
Après avoir créé l'énergie brute, le démiurge l'a fait évolué vers autre chose : les 4 éléments. C'est le type de magie le plus pratiqué. Elle utilise la puissance naturelle qui maintient le monde en place. Chaque mage choisi les éléments qu'il va apprendre, et peu donc se spécialiser plus ou moins.
\subsubsection{Les sorts}
Les sorts se répartissent par Éléments puis en Cercle. Chaque mage a la capacité, et le droit, d'apprendre des sorts de son Cercle. Son cercle se calcule très simplement : Cercle=Niveau+Points de Spécialité (voir si dessous)
\subsubsection{Pour le MJ}
Le mage élémentaire se conçoit de manière un peu plus complexe. Chaque joueur va choisir quel élément lui plait le plus. Pour ça, il dispose de 4 points de "spécialisation" à répartir entre les 4 éléments (eau, feu, terre et air). Chacun de ces point bonus ajoute un cercle aux sorts maitrisés. 
Par exemple, un joueur voulant bien maitriser le feu et un peu l'eau aura 3 points de Spécialité dans le feu et 1 dans l'eau. Ce qui fait qu'il aura le droit d'apprendre des sorts de cercle 4 en feu (il est niveau 1, plus ses 3 points), et de cercle 2 en feu (Il est niveau 1, plus 1 points).
\subsection{Le cas de la lumière et des ténèbres}
Le lumière et les ténèbres sont deux types de magie particulières. En effet, elles détournent certaines règles établies par le démiurge pour contrôler sa créativité. Cette magie, si ce mot convient encore, a donc surtout des valeurs curatives pour la lumière, et maladives pour les ténèbres. Ces types de magies sont praticables par quiconque le veut, pour peu qu'il connaisse les sorts.
\subsubsection{Les sorts}
\subsubsection{Pour le MJ}
La magie lumière/ténèbres n'est pas un type de magie à proprement parler, et donc un joueur ne pourra pas être lumière ou ténèbres. Chacun (même les guerriers) est libre d’utiliser cette magie, pour le prix d'une action simple.
Les sorts s’apprennent quant à eux dans les livres ou de la bouche de quelqu'un. 