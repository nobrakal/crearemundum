%Caractéristiques types d'un paragraphe:
% \subsection{TitreDuParagrape}
% \hypertarget {titreduparagraphe} (en minuscule et sans accent)
% Votre texte
% Merci de retourner à la ligne à chaque phrase. Cela n'influencera pas la présentation de votre texte et permettra une meilleure lisibilité
% Un lien intra document se fait ainsi: \hyperlink {lenomdepuaragraphe (tel que déclaré via hypertarget)}{le texte à afficher}
% Merci de votre participation à Creare Mundum ;)

\section{Religion}
\subsection{Le Démiurge}
\hypertarget {demiurge}{}Le Démiurge est une entité créatrice. Il est à l'origine de toute chose dans le monde. Mais il ne distingue ni le mal, ni le bien. Le Démiurge a d'abord créé la terre, l'a formé à son aise. Puis, il a créé la vie dans son monde, ainsi que la mort, qui induisait une évolution, un renouvellement. Depuis, il s'amuse à le voire progresser et grandir. Le Démiurge résiderai dans une grande et belle maison qui domine la ville de \hyperlink{transition}{Transition}. Le peu de fois qu'il apparait, c'est, selon les rumeurs, sous la forme d'un grand homme en blanc, encapuchonné.
\subsection{Les règles de création}
Au bout d'un certain temps, le Démiurge s'est aperçu de sa toute puissance, et a pris peur. Il a donc créé des règles, inconnues pour la plupart, empêchant sa folie créatrice de ravager le monde. En effet, certaine personne faisant le bien autour d'elles vives plus vieilles, certaines organisations progressent plus vite que d'autre... La religion à proprement parler se base sur ces règles. Le peu qui en est connue est utilisé par différentes églises pour faire le bien (via la magie de lumière) ou le mal (via la magie de ténèbres).
\subsection{Les Églises}
Il existe beaucoup d'ordre différent qui utilisent les règles de créations. Beaucoup sont de petites églises locale, qui demandent une offrande pour de meilleure récolte.
Il existe cependant 3 grandes églises:
\begin{itemize}
\item Saar: Prône une philosophie de vie très simple, en l'échange d'une grande félicitée.
\item Athadaxot: Prône la sagesse, le savoir.
\item Tanpphir: Ordre du soin, fait office “d'hopital”.
\end{itemize} 