%Caractéristiques types d'un paragraphe:
% \subsection{TitreDuParagrape}
% \hypertarget {titreduparagraphe} (en minuscule et sans accent)
% Votre texte
% Merci de retourner à la ligne à chaque phrase. Cela n'influencera pas la présentation de votre texte et permettra une meilleure lisibilité
% Un lien intra document se fait ainsi: \hyperlink {lenomdepuaragraphe (tel que déclaré via hypertarget)}{le texte à afficher}
% Merci de votre participation à Creare Mundum ;)

\section{Religion}
\subsection{Le Démiurge}
\hypertarget {demiurge}{}Le monde a été formé par une entité créatrice que l'on nomme Démiurge (dans le langage humain tout du moins). Cette entité n'a qu'un but: écrire des histoires. C'est pour cette raison qu'il a créé le monde, et surtout la vie à l’intérieur. Cette source infinie de contes à écrire le ravit au plus haut point, il n'a qu'a saisir sa plume et écrire ce qu'il voit. On ne lui connait aucune apparence physique précise. Tout le monde sait qu'il réside (si une telle entité peut avoir un lieu de résidence) dans la grande maison blanche qui domine \hyperlink{transition}{Transition}.
\subsection{Les règles de création}
Le Démiurge a créé des règles pour régir la vie dans son monde, et favoriser les récits épiques. La plupart des gens ont une envie de justice, de "bien", et ces derniers vivent plus vieux que la moyenne. Les grands héros, avec un certain idéal de la vie, sont donc vénéré comme des sorte de saints, les gens cherchant à les imiter et à atteindre leur longévités.
\subsection{Vénération}
Les gens ne vénèrent ni ne haïssent le Démiurge: Il est la source de leur existence ainsi que de leurs malheurs, la balance s'équilibre, et ne penche d'aucun côté. Personne ne le vénère donc directement (on ne vénère pas les écrivains). Quelques fous s'y essayent cependant, et les groupes qu'ils forment sont réputés très dangereux, ce qui ravit encore plus leur Dieu. Les humains communs y préfèrent les églises traditionnelles.
\subsection{Les Églises}
Il existe beaucoup d'ordre différent qui utilisent les règles de créations. Beaucoup sont de petites églises locale, qui demandent une offrande ou une simple hygiène de vie pour de meilleures récoltes.
Il existe cependant 3 grandes églises:
\begin{itemize}
\item Saar: Prêche pour une vie d'érudition.
\item Athumar: Prône une philosophie de vie très simple, en l'échange d'une grande félicitée.
\item Tanpphir: Ordre du soin, fait office “d'hôpital”.
\end{itemize} 
\subsection{La grande Bibliothèque}
\hypertarget{bibliotheque}{}
La grande Bibliothèque est un bâtiment qui domine Transition. Elle renferme l'ensemble des écrits de Milan de Garabe, ainsi que de nombreux livres de sages de toutes races et de toutes les époques.
Nombre d'érudits de toutes races et nationalités viennent y consulter les grands récits.
\newline
Ces érudits se constituent gardes de la bibliothèque, et ils se partagent la charge de surveiller qu'aucun livre ne sorte, ni qu'aucun conflit n'éclate. On les appelle, justement, les \textit{conservatoris}. Le fanatisme est de rigueur entre eux. Une grande partie des \textit{conservatoris} est issue de l'ordre de Saar. 
\newline
Régulièrement, de nouvelles pièces sont ouvertes, contenant un nombre incroyable de livres.
\subsubsection{Conflits}
La rumeur court que le Démiurge aurait pris de l'avance dans l'écriture de ses notes, et donc que certains livres renfermeraient l'avenir... Malheureusement, personne ne sait si cela est vrai ou pas, et régulièrement, des personnes fouillent la bibliothèque et les livres non répertoriés à la recherche du si précieux livre.