\section{Thorneye}
\hypertarget{thorneye}Thorneye est le berceau de la race des Elfes. A l'époque de leur apogée, cette forêt s'étend sur une superficie extraordinaire, englobant une bonne partie des régions alentours, et s'étendant d'Est en Ouest du continent. En son centre, il est dit que les palais des Elfes rivalisent de beauté et de splendeur, et que les notables de la société Elfe se font un devoir d'entretenir une ou plusieurs demeures les plus splendides possibles.
\section{Kazadren}
\hypertarget{kazadren}Kazadren, le pays des mauriacs, est une forteresse géante, bastion imprenable perché sur les pics de la chaîne de montagnes éponyme. Les tours et les remparts semblent se fondre avec le roc environnant, et donnent l'impression d'être à l'abri de tout. Cependant, l'impression que dégage la forteresse vue de loin n'est qu'un pâle aperçu de son aspect intérieur. 
\newline
Les demeures souterraines des mauriacs sont en effet réputées pour leur somptuosité. Les réseaux de couloirs longs de kilomètres, les maisons titanesques décorées avec ostentation, tout à Kazadren renvoie une impression de luxe et de faste, couplé avec une efficacité et un souci du détail sans pareil. Les forêts pétrifiées souterraines, les cavernes dont les murs scintillent de pierres précieuses, les ponts entre les niveaux, tantôt titanesques, tantôt léger et aériens, ne sont que quelques exemples des merveilles du royaume de Kazadren. Cependant, s'il existe une chose qui peut se mesurer au luxe des galeries de Kazadren, c'est son armée. Des soldats innombrables, bien équipés, bien entraînés, qui ne craignent aucun ennemi forment la plus efficace ligne de défense jamais vue dans ce monde. Les Arodefs de combat confèrent à cette armée une mobilité et une réactivité sans faille, ainsi qu'une puissance de feu impressionnante.
\subsection{Dren}
La capitale de Kazadren est aussi sa plus merveilleuse gemme. Ses portes, monstres de bois millénaires incrusté de granit, s'ouvrent vers les plaines au sud des montagnes. Les galeries de Dren ont été creusées sur plusieurs dizaines de niveaux, reliées entre elles par des centaines d'escaliers, de passages et d'accès dérobés. Ces galeries, toutes somptueuses et grandioses, lient des différents palais de la ville. De nombreuses salles sont décorées avec faste, mais nulle n'égale la salle du trône. Ses dimensions exactes sont connues de peu, mais elles est réputée pour son gigantisme. Les colonnes, sur deux rangés parallèles, mènent jusqu'au trône, monolithe de marbre mesurant plusieurs mètres de haut.
\subsubsection{Les grottes de Kétédren}
Ces grottes sont connues pour être magnifiques et somptueuses, même selon les critères des mauriacs. Sur plusieurs centaines de mètre de haut, cette trouée est comme une balafre dans la montagne de Kazadren. La roche, couverte d'une végétation luminescente, éclaire l'endroit d'une pâle lueur. Cette lueur éclaire les environs et permet de discerner la splendeur du lieu. La grotte abrite en effet une chute d'eau d'une beauté incommensurable. L'eau ruisselle et dévale les pentes en de multiples cascades, rugissant et glougloutant jusqu'à se perdre dans les tréfonds de la montagne.
\section{Le désert Orque}
Aride et cruel, le Désert Orque est impitoyable envers ceux qui y pénètrent, et il ne fait pas d'exception pour les autochtones. De ce fait, les Orques souffrent de mauvaises conditions de vie. Ils sont réduits à un mode de vie nomade, vivant de rapines occasionnelles sur les clans humains.
\section{La faille}
La faille est une balafre dans les plaines. Ses rochers ont une bien mauvaise réputation, et les voyageurs l’évitent, certains n'en étant jamais revenus.
\section{Les clans humains}
Répartis sur les côtes de la Mer Noble, et pour quelques un empiétant sur les Monts Mineurs, les hommes vivent pauvrement, à l'ombre des autres races de ce monde. Ils vivent de chasse et d'une agriculture timide, mais leurs conditions de vie sont mauvaises. L'espérance de vie est faible, le niveau technologique bas, et les rapines des Nalfein et des Orques les maintiennent à l'échelon le plus bas de la civilisation. Les hommes maîtrisent à peine la métallurgie, et leurs armes sont de mauvaise qualité. Cependant, toutes ces privations ont eu un effet inattendu : elles ont forgé une race d'hommes forts et endurants, qui résistent bien aux privations et qui sont assez robustes pour ne pas disparaître en quelques années.
\section{Les Terres inhabités}
Les montagnes de Thargelon, la forêt d'Aegnord et l'île de Rinam sont encore à l'état sauvage, inhospitalières et désertes. Les montagnes ont un climat froid et austère. Quant à elle, Aegnord est sauvage et dangereuse. Son climat est chaud et humide, la végétation exubérante. 
\newline
Rinam est une île quasi-déserte, sèche et aride. Quelques buissons y poussent, ainsi qu'une herbe sèche et rase. Les côtes nord et ouest de l'île sont couvertes de falaises abruptes, surplombant des plages fines et entrecoupées de sentiers abrupts. Les tempêtes agitent souvent cette partie de l'île, et des vagues de plusieurs mètres accompagnées de vents violents frappent souvent les falaises avec force. A l'inverse, les côtes sud et est sont en pente douce, le climat y est plus calme.
\section{Les territoires Nalfein}
Au sud de Thargelon, la race des Nalfein prospère et s'étend de plus en plus, envahissant les territoires des hommes et ravageant leurs terres. Les Nalfein, agressifs par nature, vivent en société régie par une sorte d'esprit de communauté, sans chef apparent. Ils agissent d'un commun accord pour le bien de leur race, et la vie d'un individu ne vaut rien si son sacrifice permet d'en sauver deux. Les Nalfein vivent dans des abris primitifs, organisés en cercles concentriques autour d'une place, ce qui fait ressembler leurs campements à des parodies de villages organisés. Les abris sont construits avec les matériaux disponibles sur place, et quand une région manque de ressources, les Nalfein se déplacent. Ils vivent ainsi dans un état semi-sédentaire, et détruisent tout ce qui se met sur le passage de leur migration.
\section{Transition}
Sur la mer de Tarek Eredron (l'Étendue Glacée en langage mauriac) se trouve une cité. Cette cité, perchée sur un pic rocheux, a pour nom Elleanol (Le Joyau en elfique). Transition est habitée en majorité par des Elfes. Ces derniers sont des nobles, des penseurs et des artistes en majorité. Ils restent liés à Thorneye grâce à un portail magique, reliant le palais royal à un grand hall à Transition. Cette communauté est assez indépendante du pouvoir royal, et la vie suit son cours là-bas.
\newline
Au sommet du pic, orienté vers le sud-ouest, se trouve un manoir blanc. Ce dernier constitue une petite bibliothèque regroupant les savoirs elfiques et mauriacs, écrit par leurs sages respectifs. Les autres peuples préfèrent l'oralité, ou n'ont tout simplement pas la connaissance de l'écriture.