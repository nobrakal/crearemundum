\section{Les Humains}
Durant l'ère raciale, les humains sont une minorité. Faibles et désorganisés, les quelques clans sédentaires vivent sur les bords de la Mer Noble, vivant de chasse et de pêche. Selon les lieux, la hiérarchie et les traditions varient beaucoup, mais toutes sont axées sur la survie à court terme et sans fioriture. Le niveau technologique humain est faible, les rares d'entre eux qui maîtrisent la métallurgie doivent se contenter d'objets de bronze de piètre qualité. L'espérance de vie est courte, mais les hommes de l'époque sont forts et endurants, et supportent bien les privations.

\section{Les Elfes}
Les Elfes de l'ère raciale bénéficient de conditions de vie suffisamment bonnes pour vivre sans se préoccuper sans cesse du lendemain, et ont alors quelque temps à consacrer à l'art et aux sciences. Ils forment une société structurée, et leur maîtrise des alliages leur permet d'entretenir une armée puissante, soutenue par des mages d'un rang supérieur et par des stratèges supérieurs. Leur empire s'étend de Thorneye aux Monts Mineurs, et englobe les plaines de sang. Les Elfes sont la race dominante de ce monde, et ils semblent appelés à régner sur les décennies à venir.

\section{Les Nains}
Les Nains, race déjà ancienne qui a connu son heure de gloire, forment un peuple plus réservé qui ne se mêle pas des affaires extérieures. Repliés dans les montagnes de Dren, ils vivent leur vie à l'écart, et cela sied très bien aux Elfes car il est dit que les nains possèdent toujours une armée impressionnante, composée de soldats hautement entraînés et équipés des meilleures armes jamais forgées, ce qui en fait un peuple à craindre. Il est dit que le jour où les nains sortiront de leurs demeures souterraines, le mode entier sera menacé par la guerre.

\section{Les Nalfein}
Les Nalfein sont une race aux coutumes étranges et belliqueuses. Humanoïdes, ils sont grands (presque deux mètres) et musculeux. Leur peau est brune, parcheminée, et d'une épaisseur telle qu'ils n'ont presque pas besoin de vêtements, un pagne leur convenant très bien. Ils sont peu armés, leurs griffes et leur poigne puissante leur suffisent pour dépecer la plupart de leurs adversaires. Leurs pieds sont protégés par une épaisse couche de corne, qui leur sert de chaussure naturelle. Leur tête est ovale, surmontée d'une crête en os, et leur bouche est protégée par une sorte de peigne de la même matière. Leurs yeux se résument à deux fentes noires, et leurs oreilles ne sont que deux trous sur les côtés de leur tête. Les Nalfein vivent principalement au sud du continent, les plus avancés d'entre eux ayant établi des colonies au nord de Thargelon, au sud du désert Orque, et même en remontant la côte de la mer noble. Ces colons vivent entre autre du pillage, et ne rechignent pas à organiser un raid sur une communauté humaine peu protégée de temps à autre. Cette situation préoccupe peu les Elfes, mais les hommes, de plus en plus opprimés, ont commencé à fuir leur terres natales et les attaques des Nalfein doivent bien trouver une nouvelle cible...


\section{Les Orques}
Les Orques sont une peuplade mineure, des barbares dégénérés qui sont tolérés par les Elfes seulement dans la mesure où ils restent tranquilles. Ils se terrent donc au fin fond de leur désert, pillant de temps en temps un village humain quand la nourriture se fait rare. La concurrence croissante des Nalfein les gène, mais ils ne sont pas en mesure d'agir seuls ou de déclarer une guerre ouverte à ces derniers.

\section{Les Gobelins}
A l'abri de la faille Gobeline, ces petits êtres vivent et commercent entre eux, sans d'occuper des autres races et de la course du monde extérieur. Quelques champs aux alentours de leur capitale leur fournissent largement de quoi se nourrir, et même un surplus que les plus courageux d'entre eux essaient d'exporter.