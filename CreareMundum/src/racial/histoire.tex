\section{Les invasions Nalfein}
Au fur et à mesure qu'ils épuisent les ressources environnantes, les Nalfein se déplacent et cherchent de nouvelles zones à piller. C'est ainsi que leurs déprédations les ont conduits jusqu'aux limites des colonies humaines, et même aux frontières de l'empire Elfe. Les Nalfein ont ensuite commencé à faire ce qu'ils faisaient de mieux : piller. Leurs rapines ont commencé à agacer les Elfes, et ces derniers ont lancé quelques campagnes de représailles, rasant des campements Nalfein les plus proches. Cependant, au lieu de se laisser dissuader, les Nalfein ont augmenté l'intensité de leurs raids. S'ensuivit rapidement une guerre ouverte. Une force punitive d'Elfes fut montée, menée par le général Eden-Atil. 
\newline
Eden-Atil était un Elfe grand, altier et aimé par ses pairs. Il était vu comme un bon stratège, quoi qu'un peu borné parfois. Il ne manquait pas non plus de courage au combat, et n'hésitait pas à mener ses hommes depuis la première ligne. Levant une force de milliers de soldats, Eden-Atil se lança dans ce qui devait être un massacre rapide et facile. Les premiers campements Nalfein ne résistèrent pas longtemps, submergés sous le nombre. Cependant, une fois qu'elle eut dépassé le désert Orque, l'armée se heurta à un plus grand nombre d'ennemis. Des groupes entiers de Nalfein, mus par un instinct mystérieux, commencèrent à converger vers la zone des combats. Individuellement, les Nalfein étaient dangereux. En groupe, ils étaient terrifiants. Malgré leur manque de coordination, ils se battaient farouchement, sans pitié. Les Elfes, tenant bon au départ, commencèrent rapidement à ployer sous le nombre croissant d'ennemis. Au bout de deux mois d'affrontements acharnés, les Nalfein avaient clairement pris l'avantage. Une fois que cette information fut portée aux oreilles des nobles de Thorneye, une délégation fut envoyée dans le palais royal de Dren pour demander des renforts. Dran Arkh'Akkor, le roi nain de l'époque, était conscient du danger représenté par les Nalfein et fut prompt à réagir. Les portes de la montagne s'ouvrirent en grand pour la première fois depuis des éons et la terre trembla sous les pas de l'armée la plus nombreuse qui fut jamais levée. Sans perdre de temps, et soutenues par l'important réseau de ravitaillement des Elfes, les phalanges naniques traversèrent le continent sans encombre. Elles se joignirent rapidement aux force Elfes déjà affaiblies par des semaines de combats éreintants. Les nains se précipitèrent au combat sans tarder, avec une bravoure et une efficacité sans pareilles. La progression de l'armée coalisée reprit rapidement, exterminant toujours plus de Nalfein, subissant toujours plus de pertes. Les steppes arides du sud du Désert Orque se couvrirent rapidement des corps enchevêtrés des trois races.