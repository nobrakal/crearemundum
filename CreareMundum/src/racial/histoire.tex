\section{Les guerres Nalfein}
Au fur et à mesure qu'ils épuisent les ressources environnantes, les Nalfein se déplacent et cherchent de nouvelles zones à piller. C'est ainsi que leurs déprédations les ont conduits jusqu'aux limites des colonies humaines, et même aux frontières de l'empire Elfe. Les Nalfein ont ensuite commencé à faire ce qu'ils faisaient de mieux : piller. Leurs rapines ont commencé à agacer les Elfes, et ces derniers ont lancé quelques campagnes de représailles, rasant des campements Nalfein les plus proches. Cependant, au lieu de se laisser dissuader, les Nalfein ont augmenté l'intensité de leurs raids. S'ensuivit rapidement une guerre ouverte. Une force punitive d'Elfes fut montée, menée par le général Eden-Atil. 
\newline
Eden-Atil était un Elfe grand, altier et aimé par ses pairs. Il était vu comme un bon stratège, quoi qu'un peu borné parfois. Il ne manquait pas non plus de courage au combat, et n'hésitait pas à mener ses hommes depuis la première ligne. Levant une force de milliers de soldats, Eden-Atil se lança dans ce qui devait être un massacre rapide et facile. Les premiers campements Nalfein ne résistèrent pas longtemps, submergés sous le nombre. Cependant, une fois qu'elle eut dépassé le désert Orque, l'armée se heurta à un plus grand nombre d'ennemis. Des groupes entiers de Nalfein, mus par un instinct mystérieux, commencèrent à converger vers la zone des combats. Individuellement, les Nalfein étaient dangereux. En groupe, ils étaient terrifiants. Malgré leur manque de coordination, ils se battaient farouchement, sans pitié. Les Elfes, tenant bon au départ, commencèrent rapidement à ployer sous le nombre croissant d'ennemis. Un régiment de quelques centaines de soldat Elfes fut coupé du reste de l'armée par un groupe de Nalfein. Ces soldats, mené par le capitaine Edra Ellarann, une Elfe réputée pour son indépendance, lutta pendant des jours contre les Nalfein pour rejoindre le corps d'armée principal. Cependant, il devint vite évident que les ennemis étaient trop nombreux pour réussir un tel exploit. Le régiment se borna donc à quelques opérations d'arrière-garde pour tuer un maximum de créatures. Au bout de plusieurs dizaines d'opérations discrètes, Ellarann décida de lever le camp. Ses soldats étaient portés pour morts et ils commençaient à manquer de vivres et d'équipement en bon état. Ils longèrent donc la côte de la Noble Mer, espérant trouver un refuge où attendre des jours plus propices.
\newline
Au bout de deux mois d'affrontements acharnés, les Nalfein avaient clairement pris l'avantage. Une fois que cette information fut portée aux oreilles des nobles de Thorneye, une délégation fut envoyée dans le palais royal de Dren pour demander des renforts. Dran Arkh'Akkor, le roi nain de l'époque, était conscient du danger représenté par les Nalfein et fut prompt à réagir. Les portes de la montagne s'ouvrirent en grand pour la première fois depuis des éons et la terre trembla sous les pas de l'armée la plus nombreuse qui fut jamais levée. Sans perdre de temps, et soutenues par l'important réseau de ravitaillement des Elfes, les phalanges naniques traversèrent le continent sans encombre. Menées par le général Edren Arkh'Akkor, le fils du roi, elles se joignirent rapidement aux force Elfes déjà affaiblies par des semaines de combats éreintants. Les nains se précipitèrent au combat sans tarder, avec une bravoure et une efficacité sans pareilles. La progression de l'armée coalisée reprit rapidement, exterminant toujours plus de Nalfein, subissant toujours plus de pertes. Les steppes arides du sud du Désert Orque se couvrirent rapidement des corps enchevêtrés des trois races.
\newline
Après deux mois de combats acharnés, le front s'enlisa. La technique et l'armement des Elfes et des Nains se heurtait à la force brute et à l'instinct des Nalfein. De nombreux campements avaient été détruits, mais les forces s'épuisaient, et le ravitaillement se faisait de plus en plus difficile, les Orques s'enhardissant et organisant de plus en plus de raids. La guerre n'était qu'une succession de batailles sanglantes, ne s'interrompant que quand un des deux partis était exterminé. aucune stratégie n'était de mise, aucun prisonnier n'était épargné. Le moral était très bas chez les coalisés, les soldats ayant l'impression de participer à une boucherie sans fin. Quant à eux, les Nalfein ignoraient jusqu'au concept de moral, ils combattaient car leur instinct le leur dictait et ils y mettaient toutes leurs forces, participant parfois à des combats jusqu'à tomber d'épuisement.
\newline
Les semaines qui suivirent furent aussi violentes que les précédentes. Les trois armées s'amenuisaient petit à petit. Un tournant majeur du conflit fur la mort du général Eden-Atil, le crâne défoncé par le puissant coup de gourdin d'un Nalfein. La chaîne de commandement des Elfes étant stable et bien définie, un nouveau général fut promptement nommé, du nom de Tal'Angmar. Il était également compétent, mais contrairement à son prédécesseur il préférait les réunions stratégiques aux champs de bataille. Cependant, les Elfes et les Nains continuaient à périr, en emportant bien sûr leur comptant de Nalfein dans la tombe. Cependant, alors que les généraux commençaient à douter de leur capacité à poursuivre le conflit, les monstres commencèrent à refluer. En effet, ces derniers avaient également perdu "espoir", et la préservation de leur race leur dictait de se replier.
\newline
Les armée naine et elfe purent alors se replier, après des mois de campagne éreintante. Cependant, Tal'Angmar décida de profiter de sa présence au sud du continent pour organiser une série de frappes de représailles sur les pillards Orques qui avaient commencé à cibler de petites communautés Elfes. Malgré la fatigue accumulée lors des dernières semaines, les troupes de Tal'Angmar n'eurent aucune difficulté à exterminer les bandits qui rôdaient dans la région. Exsangues, le Elfes se replièrent dans leur capitale à Thorneye et à ses environs, abandonnant une vaste partie de leur domaine. Les nains, quant à eux, se replièrent à nouveau sous leur montagne, léchant leurs blessures dans leur domaine souterrain.
\section{Expansions humaines et colonisations elfiques}
Avec les forêts Elfes soudainement libérées, les humains purent s'étendre et prospérer. Ils envahirent les abords de la forêt, et commencèrent à gagner du terrain sur les bois. Réunis sous la coupe d'un certain Ermudor, les humains devinrent assez forts pour résister aux raids Orques. Ils formèrent rapidement une nation, croissant en influence, et Ermudor fut nommé premier Empereur des hommes.
Quant à eux, les Elfes d'Ellarann étaient toujours sur la côte de la Noble Mer. Ils avaient établi un campement quasi-permanent et s'étaient habitués à leur mode de vie. Certains se sentaient abandonnés par Thorneye, d'autres avaient toujours douté du système monarchique. Finalement, il fut décidé qu'ils ne rejoindraient pas leurs compatriotes. Ils commercèrent avec les clans humains, échangeant du bois pour construire des navires contre des armes et de l'équipement. Ils prirent finalement la direction de l'île de Rinam, posant là-bas les bases d'une société démocratique. Ils choisirent leur capitaine comme dirigeante de cette nation naissante, et commencèrent à apprivoiser la mer et ses caprices.