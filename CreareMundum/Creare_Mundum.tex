\documentclass{book}
\usepackage[utf8]{inputenc}
\usepackage[T1]{fontenc}
\usepackage[francais]{babel}
\usepackage[linkcolor=blue, colorlinks=true,]{hyperref}
\usepackage{graphicx}
\usepackage{fancyhdr}
\pagestyle{fancy}
\fancyhead[C]{\rightmark}
\fancyhead[L]{}
\fancyhead[R]{}
\fancyfoot[LO]{\hyperlink {participation} {Envie de participer?}}
\fancyfoot[RE]{\hyperlink {participation} {Envie de participer?}}

\makeatletter
\let\insertdate\@date
\makeatother

% Creare Mundum est sous licence CC-BY-SA, présente dans le dossier d'origine, merci de la respecter!
%
% Page mère regroupant les autres pour une meilleure coordination des participants (communiquez uniquement le fichier modifié)
% Pour tout nouveau participant, il est conseillé d'aller voir le tutoriel sur le LaTex du site du zero.

\title{Creare Mundum \\ Devel}
\author{The Creare Mundum Project}
\date{\oldstylenums{\insertdate}}
\begin{document}
\maketitle
\setcounter{tocdepth}{2} %Génération du Sommaire.
\renewcommand{\contentsname}{Sommaire} 
\tableofcontents

%%%%%%%%%
\part{L'ère raciale}
\section{Les Humains}
\subsection{Caractéristiques physiques}
Les humains sont sont assez différents des ce côté-là. Il peuvent mesurer de 1m6o à 2m10. Les humains du nord, du fait de croisement avec les nains, sont de plus petite taille. Par contre, les humains de sud, qui connaissent mieux les orques et les alvénis, sont relativement plus grand. La plupart des personnes ont les yeux et les cheveux d’un brun profond. Les humains ne tendent pas à avoir un corps athlétique naturellement, bien que la plupart s’entraine rigoureusement dans ce but. 
\subsection{Us et coutumes}
\subsubsection{Mariage}
Les hommes se marient vers l’âge de vingt ans avec des jeunes filles de quinze ans. Ils se font souvent par intérêt financier, la marié amenant sa fameuse dot. Les mariages ’amoureux’ sont assez mal vu, car synonyme la plupart du temps de pauvreté. 
\subsubsection{Héritage}
Les biens se transmettent de père en fils (ou fille s’y il n’y a pas de fils), par testament. La loi oblige le testamentaire à signer de son propre sang, pour assurer son consentement. 
\subsubsection{Repas}
Les humains prennent leur repas à heure fixe : 7h le matin, 11h le midi et 19h le soir. Ils mangent relativement copieusement, même les pauvres. C’est une chose sacré dans leur tradition. Mais il est vrai que les nordiques affectionnent encore plus les repas, étant indispensable à leur survis en extérieur. Ils y vouent un véritable culte et il serait impensable ne serait-ce que d'arriver en retard à un des diner. 
\subsubsection{Hiérarchie}
Les hommes apprécient beaucoup être classé selon leur place dans la société. Le plus haut rang est celui d’empereur, puis vient les titres de noblesses (dans l’ordre : Duc, Marquis, Comte, Vicomte et Baron). Ensuite vient l’armée, très appréciée et respectée chez les humains. Puis apparait la bourgeoisie. Là, l’ordre se fait en fonction de la richesse, et des relations (les deux n’étant pas souvent séparés). Enfin, le gros de la masse, le peuple, où comme pour la bourgeoisie, la place se fait en fonction de la fortune. Certains se trouvent en dessous du peuple, comme les bourreaux, les repris de justice et les vagabonds. 
\subsubsection{Relations aux autres peuples}
Les humains sont très accueillant et ouvert aux autres peuples et coutumes. Tous les postes leurs sont accordées, sauf dans la noblesse (encore qu’il y ai quelques exceptions). 

\section{Les Mauriacs}
\subsection{Caractéristiques physiques}
Les mauriacs sont de petits êtres, leur taille varie entre 80 centimètres et 1 mètre 50. Ils vivent exclusivement dans les montagnes (notamment celles de Kazadren et de Thargelon). C'est un peuple semi-troglodytique.

\section{Les Skhos}
\subsection{Caractéristiques physiques}
Les skhos sont des créatures pouvant mesurer de 95 cm à 1m45, bien que ceux des montagnes soit généralement plus petits et ceux des plaines (ces derniers usent aussi plus de leur bipédie, afin de voir plus loin). Ils possèdent une peau différente selon l’habitat d’origine: ocre pour le désert, vert, brun, marron pour les forêt et les plaines grise, noire pour les terrains rocheux ou blanche pour les lieux enneigés. Les skhos s'adaptent étonnamment bien à ces différents milieux. Ils ont néanmoins tous des yeux d'un orange vif.
\subsection{Métissage}
Certains skhos se sont mélangés avec d'autres races. Ce métissage est très mal vu et les “balzaf”, comme ils sont appelés, sont souvent expulsés des communautés skhosiennes.
\subsection{Organisation}
Il existe deux types de sociétés skhos, très différentes, et il est fortement déconseillé de prendre les uns pour les autres.
\subsection{Les skhos sauvages}
Les gobelins sauvages, qui ont un comportement tribal. Leurs sociétés varient, mais on retrouve généralement une forme clanique. Ces clans sont souvent isolés, mais ont la particularité d'être très agressif, et quelque soit la créature adverse. Ils se font souvent la guerre entre eux, mais il arrive qu'ils attaquent des petits villages humains isolés.
\subsection{Les grands skhos}
Les grands skhos vivent en grande communauté, en grande majorité à Vrag. Leur intelligence supérieure à celle de leurs confrères provient d'expériences magiques pratiquées par des sorcier aux abords de l'actuel faille de Skhos, il y a bien longtemps. Leur dirigeant est d'ailleurs par tradition un sorcier.
Ils souffrent cependant qu’on les y associe. Les grands skhos possèdent des lois strictement établies.
\subsubsection{Hiérarchie}
Tout en haut de l'échelle se trouve le roi-sorcier, puis... Plus rien. Les grands skhos ne reconnaissent officiellement que ce grand chef. Il s'agit après d'une hiérarchie officieuse, surtout basée sur la renommée marchande.
\subsubsection{Le commerce}
Quelque soit leurs origines, les grands skhos vouent un grand culte à ce qui est pour eux un art: le commerce.

\section{Les Dalcés}
\subsection{Caractéristiques physiques}
Les dalcés sont plutôt de haute stature, environ 1 mètre 80. Ils ont la peau d'un bleu très pale.
\subsection{Origines}
Les dalcés sont issus d'une très vielle migration d'alvénis.
\subsection{Hiérarchie}
\subsubsection{Le président}
Les dalcés ont un système démocratique, et donc élisent un président. Tous les 10 ans, de grandes élections ont lieux sur l'ile, et c'est un moment de grande réjouissance.

\section{Les Alvénis}
\subsection{Caractéristiques physiques}
Les Alvénis sont d'une taille moyenne, environ celle d'un humain. Ils sont particulièrement proche de la nature.

\chapter{Géographie de l'Ère Raciale}
\section{Les Terres Humaines}
\subsection{Sombre-Cime}
\subsubsection{Description}
La population de la cité s'est de plus en plus rapprochée des mauriacs, la plupart des humains ayant choisi l'exode face au racisme croissant exprimé par les habitants du royaume voisin. La technologie de la cité est donc très proche de celle des mauriacs. La cité s'éloigne de plus en plus des autres cités humaines, et a maintenant des relations plus que froide avec Anksfall. On peut par exemple citer le conflit des 30 jours, durant lequel Sombre-Cime fut sous blocus de la part de l'imposante cité humaine d'Anksfall.
\subsection{Anksfall}
\subsubsection{Description}
La ville est assez avancée technologiquement, principalement dans le domaine des armes et de la production d'énergie. La découverte d'explosifs bon marché et faciles à produire a permis de créer des armes à feu très modernes et puissantes. C'est le fleuron de la technologie humaine.
\subsubsection{Direction politique}
Anksfall se réclame encore et toujours capitale de l'empire humain, et un empereur y vis toujours. Ce dernier na cependant plus aucun pouvoir en dehors des frontières de la cité et des plaines avoisinantes. Seuls les cités de Grahyrst en en moinde mesure Leheath conservent des liens puissants avec Anksfall, notamment économiques.
\subsection{Grahysrt}
\subsubsection{Description}
Les luttes de pouvoir incessantes qui ont pris place dans la cité ont lourdement freiné son économie, mais cette dernière commence à se relever et ses chantiers navals se remettent en marche, l'industrie de la cité étant de nouveau entièrement tournée vers la production de navires. On murmure que les chantiers navals tournent à plein régime et que des navires de guerre dotés des dernières innovations techniques commencent à sortir de la cité. Cette dernière est très proche du pouvoir central d'Anksfall, et la reconnaissance de l'empereur d'Anksfall par la cité-état revient régulièrement sur la table des dirigeants de la ville.
\subsubsection{Direction politique}
La direction de la ville est assurée par un conseil de marchands, qui reignent ici en rois. À tour de rôle, ces derniers se répartissent l'exécutif.
\subsection{Leheath}
\subsubsection{Description}
La cité est restée relativement en marge du developpement scientifique, et n'a pour ainsi dire fait aucune découverte majeure récemment. Cependant, elle reste la capitale des plaisirs, et ne se prive pas de profiter des innovations techniques de ses voisins... À sa manière.
\subsubsection{Direction politique}
Ici, c'est le prince qui dirige. Dans une vie de débauche et de luxures délicieuses, ce dernier assure l'exécutif de la cité-état. Il est heureusement secondé par nombre de conseillers, qui assurent le reste de la politique de l'état.
\subsubsection{Le solarium}
Il s'agit du joyau de la cité. Une immense baie vitrée, où reigne en permanence 35\degre. Transats, bar et autres piscines se succèdent sur quelques kilomètres, assurant des vacances inégallées sur tout le monde humain.
\section{Thargelon}
\subsection{Ketelundr}
\subsubsection{Description}
Progressivement, les mauriacs de ce royaume se sont ouverts au monde et leurs relations avec les humains se sont améliorées. Ce regain de l'entente locale a avantagé les mauriacs de Thargelon puisque la cité est devenue plus puissante que jamais et connaît son âge d'or.
\subsubsection{Technologie}
La technologie de ce royaume nain est assez orthodoxe, bien que très avancée. Les ingénieurs de Thargelon sont passés maîtres dans l'art de la miniaturisation, si bien que la plupart des mauriacs du royaume délaissent les taches manuelles au profit d'une grande compétence en ingénierie. Cela ne se fait pas sans problème, et les nains ont parfois recours à l'utilisation de procédés douteux pour subvenir à leur besoin en main d’œuvre. L'armée, quant à elle, préfère les affrontements à distance utilisant drones et missiles à un combat rapproché. Les générateurs de la cité, merveilles technologiques, utilisent les forces naturelles et les convertissent en énergie utilisable au quotidien.
\section{Kazadren}
\subsection{Dren}
\subsubsection{Description}
Au fur et à mesure du temps, les mauriacs de Kazadren se sont repliés sur eux-mêmes, et un racisme croissant envers toutes les autres races les a entraînés à haïr les mauriacs de Thargelon qui acceptent de commercer avec le reste du monde.
\subsubsection{Le géranium}
Un nouveau minerai découvert dans les montagnes voisines de la cité est désormais utilisé par les ingénieurs mauriacs: ils s'agit du géranium (Certains lui préfèreront sa dénomination scientifique de "Môr-ladron"). Ce métal est utilisé pour produire des armes dévastatrices mais à la manipulation très risquée. Ce Môr-ladron peut également servir de source d'énergie car il dégage une chaleur intense. Cette énergie peut être utilisée pour alimenter des véhicules par exemple. Des décennies d'expérimentation, ainsi que de nombreux décès, ont été nécessaires afin de mettre au point ces armes, et de nombreux mauriacs ont trouvé la mort en testant ce genre d'engins. Récemment, les mines de géranium ont commencé à fourmiller d'activité, et les usines tournent en permanence à produire un débit ahurissant de technologie létale.
\section{Le désert Orque}
\subsection{Mauhagr}
\subsubsection{Description}
Les Orques possèdent peu de technologie, se contentant de piller celle des humains lorsque l'occasion se présente. Ils sont depuis longtemps repliés sur eux-mêmes, organisant des raids de temps en temps.
\section{La faille Skhos}
\subsection{Vrag}
\subsubsection{Description}
Les Skhos étant de grands marchands, Vrag a connu un important développement et est progressivement devenue une plaque tournante du commerce mondial. Cette situation satisfait grandement la mafia locale, car elle n'hésite pas à taxer à volonté le commerce international qui transite par Vrag. Correctement avancés technologiquement, les skhos privilégient le commerce pour se fournir ce dont ils ont besoin. Ils sont peu orientés vers les armes, et seront sans doute les premières victimes collatérales si une guerre éclate.
\subsubsection{Direction politique}
Vrag, la capitale skhossienne, est dirigée par sa mafia, du moins officieusement. Il existe bien un grand roi skhos, mais ce dernier ne fait pas un pas sans l'assentiment des barons de la mafias locale.
\section{L'eau et la terre}
\subsection{Rinam}
\subsubsection{Description}
Les Dalcés, de plus en plus tournés vers la piraterie et la violence, augmentent la fréquence et l'intensité de leurs attaques pour alimenter leurs velléités expansionnistes. Ils méprisent le combat à distance, privilégiant une approche furtive suivie d'un assaut foudroyant. C'est pour cette raison que leurs véhicules légers, leurs armes de poing et de corps-à-corps sont aussi réputées que convoitées.
\subsubsection{Direction politique}
Rinam demeure une démocratie, et en est très fière ! Le président ne cesse d'ailleurs de mettre en oeuvre un "soft-power" redoutable, et on peut observer dans toutes les villes possédant un régime autortaire, notamment les villes humaines, la naissance de cellules républicaines.
\subsection{Thorneye}
\subsubsection{Description}
Les Alvénis bruns renient et méprisent toute technologie. Un prétendu guide spirituel affermit son emprise sur la société et la race entière périclite, s'accrochant désespérément aux derniers restes de sa civilisation.
\subsection{Aegnord} 
\subsubsection{Description}
Les Alvénis Gris ont vite compris qu'ils pourraient tirer profit de la technologie. Alors que certains d'entre eux restent méfiants par nature, d'autre se vouent corps et âme à la recherche sur les Intelligences Artificielles (IA). Certains d'entre eux devinrent même addict à l'utilisation de gadgets technologiques et basèrent leur vie sur leur utilisation. Leur civilisation commença à dépérir lentement mais sûrement et la technologie sur laquelle ils se reposaient signa leur perte, après des années de prospérité et de leadership dans le domaine de la technologie et de l'IA.

\chapter{Histoire de l'Ère Raciale, ou la chute des grands empires}
% Caractéristiques types d'un paragraphe:
% \subsection{TitreDuParagrape}
% \subsubsection{Description}
% \hypertarget {titreduparagraphe} (en minuscule et sans accent)
% Votre texte
% \subsubsection{Personalités} (si besoin est)
% Merci de retourner à la ligne à chaque phrase. Cela n'influencera pas la présentation de votre texte et permettra une meilleure lisibilité
% Un lien intra document se fait ainsi: \hyperlink {lenomdepuaragraphe (tel que déclaré via hypertarget)}{le texte à afficher}
% Merci de votre participation à Creare Mundum ;)

\section{Ermudor le Grand}
\subsection{De l'unification des clans humains}
\subsubsection{Naissance et enfance d'Ermudor}
Ermudor est né en 472 sur la côte de la Noble Mer, dans un taudis dans la bourgade qui deviendra Anksfall, capitale des hommes, d'un père bûcheron et d'une mère tisserande. Il apprend très vite l'art de la chasse auprès des hommes de sa tribu. Il se passionne particulièrement dans le pistage des animaux, et devient rapidement expert en ce domaine. Son tableau de chasse devient vite impressionnant, ce qui lui vaut le respect des autres hommes. Ermudor se fait rapidement connaître aux alentours comme un jeune homme talentueux et généreux, qui n'hésite pas à dispenser conseils et aide à ceux qui en ont besoin. Cependant, il lui arrive régulièrement de s'isoler en forêt pendant des périodes de plus en plus longues.
\subsubsection{L'âge de raison}
Lorsqu'il atteint les quinze ans, Ermudor est un homme. 

\part{L'ère médiévale}
\begin{figure}
\begin{center}
\hypertarget{cartedumonde}{}
\includegraphics[scale=0.13]{./Ressources/medieval/Carte_monde.png}
\caption{La carte du monde en l'an 500}
\end{center}
\end{figure}
\chapter{Histoire médiévale}
\section{Introduction}
Nous sommes proches de l'an 500. La civilisation humaine est en pleine construction et est appelée à diriger.
Toutes les races cohabitent sur ce monde relativement bien, quoique avec quelques tensions tout de même.
La magie est présente et beaucoup utilisée.
% Caractéristiques types d'un paragraphe:
% \subsection{TitreDuParagrape}
% \subsubsection{Description}
% \hypertarget {titreduparagraphe} (en minuscule et sans accent)
% Votre texte
% \subsubsection{Personalités} (si besoin est)
% Merci de retourner à la ligne à chaque phrase. Cela n'influencera pas la présentation de votre texte et permettra une meilleure lisibilité
% Un lien intra document se fait ainsi: \hyperlink {lenomdepuaragraphe (tel que déclaré via hypertarget)}{le texte à afficher}
% Merci de votre participation à Creare Mundum ;)

\section{Ermudor le Grand}
\subsection{De l'unification des clans humains}
\subsubsection{Naissance et enfance d'Ermudor}
Ermudor est né en 472 sur la côte de la Noble Mer, dans un taudis dans la bourgade qui deviendra Anksfall, capitale des hommes, d'un père bûcheron et d'une mère tisserande. Il apprend très vite l'art de la chasse auprès des hommes de sa tribu. Il se passionne particulièrement dans le pistage des animaux, et devient rapidement expert en ce domaine. Son tableau de chasse devient vite impressionnant, ce qui lui vaut le respect des autres hommes. Ermudor se fait rapidement connaître aux alentours comme un jeune homme talentueux et généreux, qui n'hésite pas à dispenser conseils et aide à ceux qui en ont besoin. Cependant, il lui arrive régulièrement de s'isoler en forêt pendant des périodes de plus en plus longues.
\subsubsection{L'âge de raison}
Lorsqu'il atteint les quinze ans, Ermudor est un homme. 

\chapter{Géographie médiévale}
\section{Les Terres Humaines}
\subsection{Sombre-Cime}
\subsubsection{Description}
La population de la cité s'est de plus en plus rapprochée des mauriacs, la plupart des humains ayant choisi l'exode face au racisme croissant exprimé par les habitants du royaume voisin. La technologie de la cité est donc très proche de celle des mauriacs. La cité s'éloigne de plus en plus des autres cités humaines, et a maintenant des relations plus que froide avec Anksfall. On peut par exemple citer le conflit des 30 jours, durant lequel Sombre-Cime fut sous blocus de la part de l'imposante cité humaine d'Anksfall.
\subsection{Anksfall}
\subsubsection{Description}
La ville est assez avancée technologiquement, principalement dans le domaine des armes et de la production d'énergie. La découverte d'explosifs bon marché et faciles à produire a permis de créer des armes à feu très modernes et puissantes. C'est le fleuron de la technologie humaine.
\subsubsection{Direction politique}
Anksfall se réclame encore et toujours capitale de l'empire humain, et un empereur y vis toujours. Ce dernier na cependant plus aucun pouvoir en dehors des frontières de la cité et des plaines avoisinantes. Seuls les cités de Grahyrst en en moinde mesure Leheath conservent des liens puissants avec Anksfall, notamment économiques.
\subsection{Grahysrt}
\subsubsection{Description}
Les luttes de pouvoir incessantes qui ont pris place dans la cité ont lourdement freiné son économie, mais cette dernière commence à se relever et ses chantiers navals se remettent en marche, l'industrie de la cité étant de nouveau entièrement tournée vers la production de navires. On murmure que les chantiers navals tournent à plein régime et que des navires de guerre dotés des dernières innovations techniques commencent à sortir de la cité. Cette dernière est très proche du pouvoir central d'Anksfall, et la reconnaissance de l'empereur d'Anksfall par la cité-état revient régulièrement sur la table des dirigeants de la ville.
\subsubsection{Direction politique}
La direction de la ville est assurée par un conseil de marchands, qui reignent ici en rois. À tour de rôle, ces derniers se répartissent l'exécutif.
\subsection{Leheath}
\subsubsection{Description}
La cité est restée relativement en marge du developpement scientifique, et n'a pour ainsi dire fait aucune découverte majeure récemment. Cependant, elle reste la capitale des plaisirs, et ne se prive pas de profiter des innovations techniques de ses voisins... À sa manière.
\subsubsection{Direction politique}
Ici, c'est le prince qui dirige. Dans une vie de débauche et de luxures délicieuses, ce dernier assure l'exécutif de la cité-état. Il est heureusement secondé par nombre de conseillers, qui assurent le reste de la politique de l'état.
\subsubsection{Le solarium}
Il s'agit du joyau de la cité. Une immense baie vitrée, où reigne en permanence 35\degre. Transats, bar et autres piscines se succèdent sur quelques kilomètres, assurant des vacances inégallées sur tout le monde humain.
\section{Thargelon}
\subsection{Ketelundr}
\subsubsection{Description}
Progressivement, les mauriacs de ce royaume se sont ouverts au monde et leurs relations avec les humains se sont améliorées. Ce regain de l'entente locale a avantagé les mauriacs de Thargelon puisque la cité est devenue plus puissante que jamais et connaît son âge d'or.
\subsubsection{Technologie}
La technologie de ce royaume nain est assez orthodoxe, bien que très avancée. Les ingénieurs de Thargelon sont passés maîtres dans l'art de la miniaturisation, si bien que la plupart des mauriacs du royaume délaissent les taches manuelles au profit d'une grande compétence en ingénierie. Cela ne se fait pas sans problème, et les nains ont parfois recours à l'utilisation de procédés douteux pour subvenir à leur besoin en main d’œuvre. L'armée, quant à elle, préfère les affrontements à distance utilisant drones et missiles à un combat rapproché. Les générateurs de la cité, merveilles technologiques, utilisent les forces naturelles et les convertissent en énergie utilisable au quotidien.
\section{Kazadren}
\subsection{Dren}
\subsubsection{Description}
Au fur et à mesure du temps, les mauriacs de Kazadren se sont repliés sur eux-mêmes, et un racisme croissant envers toutes les autres races les a entraînés à haïr les mauriacs de Thargelon qui acceptent de commercer avec le reste du monde.
\subsubsection{Le géranium}
Un nouveau minerai découvert dans les montagnes voisines de la cité est désormais utilisé par les ingénieurs mauriacs: ils s'agit du géranium (Certains lui préfèreront sa dénomination scientifique de "Môr-ladron"). Ce métal est utilisé pour produire des armes dévastatrices mais à la manipulation très risquée. Ce Môr-ladron peut également servir de source d'énergie car il dégage une chaleur intense. Cette énergie peut être utilisée pour alimenter des véhicules par exemple. Des décennies d'expérimentation, ainsi que de nombreux décès, ont été nécessaires afin de mettre au point ces armes, et de nombreux mauriacs ont trouvé la mort en testant ce genre d'engins. Récemment, les mines de géranium ont commencé à fourmiller d'activité, et les usines tournent en permanence à produire un débit ahurissant de technologie létale.
\section{Le désert Orque}
\subsection{Mauhagr}
\subsubsection{Description}
Les Orques possèdent peu de technologie, se contentant de piller celle des humains lorsque l'occasion se présente. Ils sont depuis longtemps repliés sur eux-mêmes, organisant des raids de temps en temps.
\section{La faille Skhos}
\subsection{Vrag}
\subsubsection{Description}
Les Skhos étant de grands marchands, Vrag a connu un important développement et est progressivement devenue une plaque tournante du commerce mondial. Cette situation satisfait grandement la mafia locale, car elle n'hésite pas à taxer à volonté le commerce international qui transite par Vrag. Correctement avancés technologiquement, les skhos privilégient le commerce pour se fournir ce dont ils ont besoin. Ils sont peu orientés vers les armes, et seront sans doute les premières victimes collatérales si une guerre éclate.
\subsubsection{Direction politique}
Vrag, la capitale skhossienne, est dirigée par sa mafia, du moins officieusement. Il existe bien un grand roi skhos, mais ce dernier ne fait pas un pas sans l'assentiment des barons de la mafias locale.
\section{L'eau et la terre}
\subsection{Rinam}
\subsubsection{Description}
Les Dalcés, de plus en plus tournés vers la piraterie et la violence, augmentent la fréquence et l'intensité de leurs attaques pour alimenter leurs velléités expansionnistes. Ils méprisent le combat à distance, privilégiant une approche furtive suivie d'un assaut foudroyant. C'est pour cette raison que leurs véhicules légers, leurs armes de poing et de corps-à-corps sont aussi réputées que convoitées.
\subsubsection{Direction politique}
Rinam demeure une démocratie, et en est très fière ! Le président ne cesse d'ailleurs de mettre en oeuvre un "soft-power" redoutable, et on peut observer dans toutes les villes possédant un régime autortaire, notamment les villes humaines, la naissance de cellules républicaines.
\subsection{Thorneye}
\subsubsection{Description}
Les Alvénis bruns renient et méprisent toute technologie. Un prétendu guide spirituel affermit son emprise sur la société et la race entière périclite, s'accrochant désespérément aux derniers restes de sa civilisation.
\subsection{Aegnord} 
\subsubsection{Description}
Les Alvénis Gris ont vite compris qu'ils pourraient tirer profit de la technologie. Alors que certains d'entre eux restent méfiants par nature, d'autre se vouent corps et âme à la recherche sur les Intelligences Artificielles (IA). Certains d'entre eux devinrent même addict à l'utilisation de gadgets technologiques et basèrent leur vie sur leur utilisation. Leur civilisation commença à dépérir lentement mais sûrement et la technologie sur laquelle ils se reposaient signa leur perte, après des années de prospérité et de leadership dans le domaine de la technologie et de l'IA.

%Caractéristiques types d'un paragraphe:
% \subsection{TitreDuParagrape}
% \hypertarget {titreduparagraphe} (en minuscule et sans accent)
% Votre texte
% Merci de retourner à la ligne à chaque phrase. Cela n'influencera pas la présentation de votre texte et permettra une meilleure lisibilité
%Un lien intra document se fait ainsi: \hyperlink {lenomdepuaragraphe (tel que déclaré via hypertarget)}{le texte à afficher}
% Merci de votre participation à Creare Mundum ;)

\section{Les guildes}
\subsection{La guilde des voleurs}
\hypertarget{laguildedesvoleurs}{}Cette guilde de l'ombre est très répandue dans le monde.
Son but officiel est de voler aux riches et de le redonner aux pauvres, mais le peuple en doute et se méfie de ses membres.
Elle est dirigée par 8 grands voleurs. Ce sont les voleurs les plus talentueux que l'on puisse trouver.
Ce poste est aussi à vie. Il est impossible d'en démissionner, sous peine de mort.
La guilde des voleurs possède aussi des salles de communication.
\subsection{La guilde des magiciens}
\hypertarget {laguildedesmagiciens}{}Cette guilde est la guilde officielles des magiciens, que tout bon sorcier se doit d’incorporer.
Leur université se trouve à \hyperlink {grahyrst} {Grahyrst}.
La guilde tolère tous les types de magie, bien que la Nécromancie soit interdite, peu des ses pratiquants sont réprimandés.
La guilde dispose de succursale dans toute les grandes villes.
C'est dans ces succursales qu'il faut aller pour toutes demandes de sort.
\subsection{La guilde des guerriers}
\hypertarget {laguildedesguerriers}{}Cette guilde regroupe tous les guerriers.
Pour y rentrer, il suffit de n'avoir commis aucun délit reconnu, et de se présenter dans une succursale.
Comme les autres guildes, elle est bien représentée dans tous le pays.
Il faut s'y rendre pour des demandes de protections, d'entrainement, etc.
\subsection{La guilde de l'ombre}
\hypertarget{laguildedelombre}{}Cette guilde regroupe toute les pratiques interdites.
Fortement réprimandé par l'empereur, elle est très discrète.
Personne n'en sait et ne veut en savoir plus.
 

\chapter{Au delà de la conscience}
%Caractéristiques types d'un paragraphe:
% \subsection{TitreDuParagrape}
% \hypertarget {titreduparagraphe} (en minuscule et sans accent)
% Votre texte
% Merci de retourner à la ligne à chaque phrase. Cela n'influencera pas la présentation de votre texte et permettra une meilleure lisibilité
% Un lien intra document se fait ainsi: \hyperlink {lenomdepuaragraphe (tel que déclaré via hypertarget)}{le texte à afficher}
% Merci de votre participation à Creare Mundum ;)

\section{Religion}
\subsection{Le Démiurge}
\hypertarget {demiurge}{}Le Démiurge est une entité créatrice. Il est à l'origine de toute chose dans le monde. Mais il ne distingue ni le mal, ni le bien. Le Démiurge a d'abord créé la terre, l'a formé à son aise. Puis, il a créé la vie dans son monde, ainsi que la mort, qui induisait une évolution, un renouvellement. Depuis, il s'amuse à le voire progresser et grandir. Le Démiurge résiderai dans une grande et belle maison qui domine la ville de \hyperlink{transition}{Transition}. Le peu de fois qu'il apparait, c'est, selon les rumeurs, sous la forme d'un grand homme en blanc, encapuchonné.
\subsection{Les règles de création}
Au bout d'un certain temps, le Démiurge s'est aperçu de sa toute puissance, et a pris peur. Il a donc créé des règles, inconnues pour la plupart, empêchant sa folie créatrice de ravager le monde. En effet, certaine personne faisant le bien autour d'elles vives plus vieilles, certaines organisations progressent plus vite que d'autre... La religion à proprement parler se base sur ces règles. Le peu qui en est connue est utilisé par différentes églises pour faire le bien (via la magie de lumière) ou le mal (via la magie de ténèbres).
\subsection{Les Églises}
Il existe beaucoup d'ordre différent qui utilisent les règles de créations. Beaucoup sont de petites églises locale, qui demandent une offrande pour de meilleure récolte.
Il existe cependant 3 grandes églises:
\begin{itemize}
\item Saar: Prône une philosophie de vie très simple, en l'échange d'une grande félicitée.
\item Athadaxot: Prône la sagesse, le savoir.
\item Tanpphir: Ordre du soin, fait office “d'hopital”.
\end{itemize} 
%Caractéristiques types d'un paragraphe:
% \subsection{TitreDuParagrape}
% \hypertarget {titreduparagraphe} (en minuscule et sans accent)
% Votre texte
% Merci de retourner à la ligne à chaque phrase. Cela n'influencera pas la présentation de votre texte et permettra une meilleure lisibilité
%Un lien intra document se fait ainsi: \hyperlink {lenomdepuaragraphe (tel que déclaré via hypertarget)}{le texte à afficher}
% Merci de votre participation à Creare Mundum ;)

\section{Les âmes}
Les âmes sont des sortes de clés produites par le Démiurge, qui permettent d'appréhender son monde d'une autre manière. Grâce à cette nouvelle vision, le héros va acquérir de meilleures capacités physiques ou psychiques. Les âmes sont, sous leur formes physiques, de petites pierres d'un noir profond. Le commerce d'âmes se pratique que dans la ville de Transition.
\subsection{Exemples d’âmes}
Par exemple, les âmes peuvent conférer :
\begin{itemize}
\item Une rapidité accrue
\item Une vision accrue
\item Une plus grande force
\item Une plus grande intelligence
\item L'apprentissage de nouveaux sorts
\item La capacité de parler à certains animaux
\end{itemize}
Le maitre de jeu est libre d'inventer de nouveaux pouvoirs, la liste n'est proposée qu'a titre indicatif.
\subsection{Assimilation des âmes}
Pour que le propriétaire utilise réellement son âme, il doit fusionner avec elle. À l'aide d'une phrase d'incantation ( “Kaparushan palkinon Viha” ). La petite pierre noire va alors disparaître en une sorte de gaz de la même couleur va pénétrer dans le sujet (par les yeux, les oreilles, la bouche). Cette nouvelle vision du monde est dure à supporter, le héros va mettre du temps à s'en remettre et a totalement appréhender son âme. On fonctionnera ici tel l’expérience.
\subsection{Abandonner son âme}
Il est impossible d'abandonner une âme que l'on a assimilé. À la mort du propriétaire, son âme disparait avec lui.
\section{La magie}
La magie est une partie intégrante de nombreux jeu de rôle, et il convient de la traiter d'une façon appropriée.
Nous allons nous baser sur la magie présente dans CreareMundum, mais il est très simple de changer de "base magique".

\subsection{La magie élémentaire}
La magie la plus primaire, et qui conviendra à tout apprenti magicien. C'est dur, c'est brutal, et ça fait mal.
\subsubsection{L'attaque}
Globalement, la magie élémentaire suit les règles de dommage des armes à distance. Voici un tableau d'équivalence entre les deux.
\newline
\newline
\begin{tabular}{l|l|l}
   Rang & Sort & Flèche correspondante \\
   \hline
   4 & Boule d'énergie & Flèche de fer \\
   3 & Flash énergétique & Flèche d'acier \\
   2 & Rayon énergétique & Flèche orc \\
   1 & Supra-rayon énergétique & Flèche elfique \\
\end{tabular}
\newline
\subsubsection{La défense}
La magie élémentaire sert aussi à se défendre. Il existe 4 rang de défense, annulant les attaques de même rang.
\newline
\newline
\begin{tabular}{l|l|l}
   Rang & Sort & Sort annulé \\
   \hline
   4 & Protection énergétique  & Boule d'énergie \\
   3 & Bouclier énergétique  & Flash énergétique \\
   2 & Mur d'énergie & Rayon énergétique \\
   1 & Rempart d'énergie & Supra-rayon énergétique \\
\end{tabular}
\newline
\newline
Les sorts de rangs supérieurs annulent les dégâts des sorts inférieurs. La réciproque est par contre totalement fausse. Se protéger avec un sort de rang inférieur à l'attaque est simplement inutile. La protection vole en éclat sans enlever le moindre dégât. 

\chapter{Les races médiévales}
\section{Les Humains}
\subsection{Caractéristiques physiques}
Les humains sont sont assez différents des ce côté-là. Il peuvent mesurer de 1m6o à 2m10. Les humains du nord, du fait de croisement avec les nains, sont de plus petite taille. Par contre, les humains de sud, qui connaissent mieux les orques et les alvénis, sont relativement plus grand. La plupart des personnes ont les yeux et les cheveux d’un brun profond. Les humains ne tendent pas à avoir un corps athlétique naturellement, bien que la plupart s’entraine rigoureusement dans ce but. 
\subsection{Us et coutumes}
\subsubsection{Mariage}
Les hommes se marient vers l’âge de vingt ans avec des jeunes filles de quinze ans. Ils se font souvent par intérêt financier, la marié amenant sa fameuse dot. Les mariages ’amoureux’ sont assez mal vu, car synonyme la plupart du temps de pauvreté. 
\subsubsection{Héritage}
Les biens se transmettent de père en fils (ou fille s’y il n’y a pas de fils), par testament. La loi oblige le testamentaire à signer de son propre sang, pour assurer son consentement. 
\subsubsection{Repas}
Les humains prennent leur repas à heure fixe : 7h le matin, 11h le midi et 19h le soir. Ils mangent relativement copieusement, même les pauvres. C’est une chose sacré dans leur tradition. Mais il est vrai que les nordiques affectionnent encore plus les repas, étant indispensable à leur survis en extérieur. Ils y vouent un véritable culte et il serait impensable ne serait-ce que d'arriver en retard à un des diner. 
\subsubsection{Hiérarchie}
Les hommes apprécient beaucoup être classé selon leur place dans la société. Le plus haut rang est celui d’empereur, puis vient les titres de noblesses (dans l’ordre : Duc, Marquis, Comte, Vicomte et Baron). Ensuite vient l’armée, très appréciée et respectée chez les humains. Puis apparait la bourgeoisie. Là, l’ordre se fait en fonction de la richesse, et des relations (les deux n’étant pas souvent séparés). Enfin, le gros de la masse, le peuple, où comme pour la bourgeoisie, la place se fait en fonction de la fortune. Certains se trouvent en dessous du peuple, comme les bourreaux, les repris de justice et les vagabonds. 
\subsubsection{Relations aux autres peuples}
Les humains sont très accueillant et ouvert aux autres peuples et coutumes. Tous les postes leurs sont accordées, sauf dans la noblesse (encore qu’il y ai quelques exceptions). 

\section{Les Mauriacs}
\subsection{Caractéristiques physiques}
Les mauriacs sont de petits êtres, leur taille varie entre 80 centimètres et 1 mètre 50. Ils vivent exclusivement dans les montagnes (notamment celles de Kazadren et de Thargelon). C'est un peuple semi-troglodytique.

\section{Les Skhos}
\subsection{Caractéristiques physiques}
Les skhos sont des créatures pouvant mesurer de 95 cm à 1m45, bien que ceux des montagnes soit généralement plus petits et ceux des plaines (ces derniers usent aussi plus de leur bipédie, afin de voir plus loin). Ils possèdent une peau différente selon l’habitat d’origine: ocre pour le désert, vert, brun, marron pour les forêt et les plaines grise, noire pour les terrains rocheux ou blanche pour les lieux enneigés. Les skhos s'adaptent étonnamment bien à ces différents milieux. Ils ont néanmoins tous des yeux d'un orange vif.
\subsection{Métissage}
Certains skhos se sont mélangés avec d'autres races. Ce métissage est très mal vu et les “balzaf”, comme ils sont appelés, sont souvent expulsés des communautés skhosiennes.
\subsection{Organisation}
Il existe deux types de sociétés skhos, très différentes, et il est fortement déconseillé de prendre les uns pour les autres.
\subsection{Les skhos sauvages}
Les gobelins sauvages, qui ont un comportement tribal. Leurs sociétés varient, mais on retrouve généralement une forme clanique. Ces clans sont souvent isolés, mais ont la particularité d'être très agressif, et quelque soit la créature adverse. Ils se font souvent la guerre entre eux, mais il arrive qu'ils attaquent des petits villages humains isolés.
\subsection{Les grands skhos}
Les grands skhos vivent en grande communauté, en grande majorité à Vrag. Leur intelligence supérieure à celle de leurs confrères provient d'expériences magiques pratiquées par des sorcier aux abords de l'actuel faille de Skhos, il y a bien longtemps. Leur dirigeant est d'ailleurs par tradition un sorcier.
Ils souffrent cependant qu’on les y associe. Les grands skhos possèdent des lois strictement établies.
\subsubsection{Hiérarchie}
Tout en haut de l'échelle se trouve le roi-sorcier, puis... Plus rien. Les grands skhos ne reconnaissent officiellement que ce grand chef. Il s'agit après d'une hiérarchie officieuse, surtout basée sur la renommée marchande.
\subsubsection{Le commerce}
Quelque soit leurs origines, les grands skhos vouent un grand culte à ce qui est pour eux un art: le commerce.

\section{Les Dalcés}
\subsection{Caractéristiques physiques}
Les dalcés sont plutôt de haute stature, environ 1 mètre 80. Ils ont la peau d'un bleu très pale.
\subsection{Origines}
Les dalcés sont issus d'une très vielle migration d'alvénis.
\subsection{Hiérarchie}
\subsubsection{Le président}
Les dalcés ont un système démocratique, et donc élisent un président. Tous les 10 ans, de grandes élections ont lieux sur l'ile, et c'est un moment de grande réjouissance.

\section{Les Alvénis}
\subsection{Caractéristiques physiques}
Les Alvénis sont d'une taille moyenne, environ celle d'un humain. Ils sont particulièrement proche de la nature.


\chapter{Ressources médiévales}
\section{Musique d'ambiance}
\subsection{Tonnerre}
Un orage gronde au loin, avec un peu de pluie, 3:30 min: \href{run:./Ressources/medieval/Tonnerre.ogg}{Lancer}
\subsection{Forêt}
Une forêt en automne, quelques corbeaux croassent, 11:38 min: \href{run:./Ressources/medieval/Foret.ogg}{Lancer}
\subsection{Vent}
Le vent souffle, 3:39 min: \href{run:./Ressources/medieval/Vent.ogg}{Lancer}
\subsection{Pluie}
Une pluie torrentielle, 30 sec: \href{run:./Ressources/medieval/Pluie.ogg}{Lancer}
\subsection{Loups}
Les loups se préparent à attaquer, 6:24 min: \href{run:./Ressources/medieval/Loups.ogg}{Lancer}
\subsection{Ville}
\subsubsection{Brouhaha}
La vie d'une ville, 1:45 min: \href{run:./Ressources/medieval/Brouhaha_ville.ogg}{Lancer}
\subsubsection{Attroupement}
Les choses étranges ameutent la plèbe, 30 sec: \href{run:./Ressources/medieval/Attroupement.ogg}{Lancer}
\section{L'Assistant}
\subsubsection{Linux}
Voici un petit programme comprenant un lanceur de dés ainsi qu'un petit éditeur de texte, afin de vous aider à mener vos parties. (il nécessite gtk): \href{run:./Ressources/CreareMundum_Assistant/CreareMundum_Assistant_linux.py}{Lancer}
\subsubsection{Windows}
ToDo

%%%%%%%%%
\part{L'ère technologique}
\chapter{Histoire et Géographie technologique}
\section{Introduction}
An 2500. Les différents peuples sont arrivés à l'apogée de leurs savoirs-faire. Le monde est globalement en paix, sous l'égide humaine.
\section{Les Terres Humaines}
\subsection{Sombre-Cime}
\subsubsection{Description}
La population de la cité s'est de plus en plus rapprochée des mauriacs, la plupart des humains ayant choisi l'exode face au racisme croissant exprimé par les habitants du royaume voisin. La technologie de la cité est donc très proche de celle des mauriacs. La cité s'éloigne de plus en plus des autres cités humaines, et a maintenant des relations plus que froide avec Anksfall. On peut par exemple citer le conflit des 30 jours, durant lequel Sombre-Cime fut sous blocus de la part de l'imposante cité humaine d'Anksfall.
\subsection{Anksfall}
\subsubsection{Description}
La ville est assez avancée technologiquement, principalement dans le domaine des armes et de la production d'énergie. La découverte d'explosifs bon marché et faciles à produire a permis de créer des armes à feu très modernes et puissantes. C'est le fleuron de la technologie humaine.
\subsubsection{Direction politique}
Anksfall se réclame encore et toujours capitale de l'empire humain, et un empereur y vis toujours. Ce dernier na cependant plus aucun pouvoir en dehors des frontières de la cité et des plaines avoisinantes. Seuls les cités de Grahyrst en en moinde mesure Leheath conservent des liens puissants avec Anksfall, notamment économiques.
\subsection{Grahysrt}
\subsubsection{Description}
Les luttes de pouvoir incessantes qui ont pris place dans la cité ont lourdement freiné son économie, mais cette dernière commence à se relever et ses chantiers navals se remettent en marche, l'industrie de la cité étant de nouveau entièrement tournée vers la production de navires. On murmure que les chantiers navals tournent à plein régime et que des navires de guerre dotés des dernières innovations techniques commencent à sortir de la cité. Cette dernière est très proche du pouvoir central d'Anksfall, et la reconnaissance de l'empereur d'Anksfall par la cité-état revient régulièrement sur la table des dirigeants de la ville.
\subsubsection{Direction politique}
La direction de la ville est assurée par un conseil de marchands, qui reignent ici en rois. À tour de rôle, ces derniers se répartissent l'exécutif.
\subsection{Leheath}
\subsubsection{Description}
La cité est restée relativement en marge du developpement scientifique, et n'a pour ainsi dire fait aucune découverte majeure récemment. Cependant, elle reste la capitale des plaisirs, et ne se prive pas de profiter des innovations techniques de ses voisins... À sa manière.
\subsubsection{Direction politique}
Ici, c'est le prince qui dirige. Dans une vie de débauche et de luxures délicieuses, ce dernier assure l'exécutif de la cité-état. Il est heureusement secondé par nombre de conseillers, qui assurent le reste de la politique de l'état.
\subsubsection{Le solarium}
Il s'agit du joyau de la cité. Une immense baie vitrée, où reigne en permanence 35\degre. Transats, bar et autres piscines se succèdent sur quelques kilomètres, assurant des vacances inégallées sur tout le monde humain.
\section{Thargelon}
\subsection{Ketelundr}
\subsubsection{Description}
Progressivement, les mauriacs de ce royaume se sont ouverts au monde et leurs relations avec les humains se sont améliorées. Ce regain de l'entente locale a avantagé les mauriacs de Thargelon puisque la cité est devenue plus puissante que jamais et connaît son âge d'or.
\subsubsection{Technologie}
La technologie de ce royaume nain est assez orthodoxe, bien que très avancée. Les ingénieurs de Thargelon sont passés maîtres dans l'art de la miniaturisation, si bien que la plupart des mauriacs du royaume délaissent les taches manuelles au profit d'une grande compétence en ingénierie. Cela ne se fait pas sans problème, et les nains ont parfois recours à l'utilisation de procédés douteux pour subvenir à leur besoin en main d’œuvre. L'armée, quant à elle, préfère les affrontements à distance utilisant drones et missiles à un combat rapproché. Les générateurs de la cité, merveilles technologiques, utilisent les forces naturelles et les convertissent en énergie utilisable au quotidien.
\section{Kazadren}
\subsection{Dren}
\subsubsection{Description}
Au fur et à mesure du temps, les mauriacs de Kazadren se sont repliés sur eux-mêmes, et un racisme croissant envers toutes les autres races les a entraînés à haïr les mauriacs de Thargelon qui acceptent de commercer avec le reste du monde.
\subsubsection{Le géranium}
Un nouveau minerai découvert dans les montagnes voisines de la cité est désormais utilisé par les ingénieurs mauriacs: ils s'agit du géranium (Certains lui préfèreront sa dénomination scientifique de "Môr-ladron"). Ce métal est utilisé pour produire des armes dévastatrices mais à la manipulation très risquée. Ce Môr-ladron peut également servir de source d'énergie car il dégage une chaleur intense. Cette énergie peut être utilisée pour alimenter des véhicules par exemple. Des décennies d'expérimentation, ainsi que de nombreux décès, ont été nécessaires afin de mettre au point ces armes, et de nombreux mauriacs ont trouvé la mort en testant ce genre d'engins. Récemment, les mines de géranium ont commencé à fourmiller d'activité, et les usines tournent en permanence à produire un débit ahurissant de technologie létale.
\section{Le désert Orque}
\subsection{Mauhagr}
\subsubsection{Description}
Les Orques possèdent peu de technologie, se contentant de piller celle des humains lorsque l'occasion se présente. Ils sont depuis longtemps repliés sur eux-mêmes, organisant des raids de temps en temps.
\section{La faille Skhos}
\subsection{Vrag}
\subsubsection{Description}
Les Skhos étant de grands marchands, Vrag a connu un important développement et est progressivement devenue une plaque tournante du commerce mondial. Cette situation satisfait grandement la mafia locale, car elle n'hésite pas à taxer à volonté le commerce international qui transite par Vrag. Correctement avancés technologiquement, les skhos privilégient le commerce pour se fournir ce dont ils ont besoin. Ils sont peu orientés vers les armes, et seront sans doute les premières victimes collatérales si une guerre éclate.
\subsubsection{Direction politique}
Vrag, la capitale skhossienne, est dirigée par sa mafia, du moins officieusement. Il existe bien un grand roi skhos, mais ce dernier ne fait pas un pas sans l'assentiment des barons de la mafias locale.
\section{L'eau et la terre}
\subsection{Rinam}
\subsubsection{Description}
Les Dalcés, de plus en plus tournés vers la piraterie et la violence, augmentent la fréquence et l'intensité de leurs attaques pour alimenter leurs velléités expansionnistes. Ils méprisent le combat à distance, privilégiant une approche furtive suivie d'un assaut foudroyant. C'est pour cette raison que leurs véhicules légers, leurs armes de poing et de corps-à-corps sont aussi réputées que convoitées.
\subsubsection{Direction politique}
Rinam demeure une démocratie, et en est très fière ! Le président ne cesse d'ailleurs de mettre en oeuvre un "soft-power" redoutable, et on peut observer dans toutes les villes possédant un régime autortaire, notamment les villes humaines, la naissance de cellules républicaines.
\subsection{Thorneye}
\subsubsection{Description}
Les Alvénis bruns renient et méprisent toute technologie. Un prétendu guide spirituel affermit son emprise sur la société et la race entière périclite, s'accrochant désespérément aux derniers restes de sa civilisation.
\subsection{Aegnord} 
\subsubsection{Description}
Les Alvénis Gris ont vite compris qu'ils pourraient tirer profit de la technologie. Alors que certains d'entre eux restent méfiants par nature, d'autre se vouent corps et âme à la recherche sur les Intelligences Artificielles (IA). Certains d'entre eux devinrent même addict à l'utilisation de gadgets technologiques et basèrent leur vie sur leur utilisation. Leur civilisation commença à dépérir lentement mais sûrement et la technologie sur laquelle ils se reposaient signa leur perte, après des années de prospérité et de leadership dans le domaine de la technologie et de l'IA.


%%%%%%%%%
\part{L'Apocalypse}
\chapter{Now !}
\section{Introduction}
An 2863. Le monde s'enflamme. Les peuples combattent. Les pleurs et le sang remplacent les sourires et le vin. C'est la guerre, la véritable guerre.
\section{Récit par Milan de Garabe}
\subsection{À propos de Milan de Garabe}
Milan de Garabe fut un humain très érudit, même pour son époque où toutes les sciences semblaient acquises. Il a laissé un des rares témoignages de l'apocalypse, à travers ses \underline{Notes}. Voir sa note n\degre1375.

%%%%%%%%%
\part{L'ère post-apocalyptique}
\chapter{Histoire et Géographie post-apocalyptique}
\section{Introduction}
Nous sommes en l'an 3000. Les civilisations se sont détruites après des siècles de combat. 
Il ne reste que quelques débris de l'ancien monde. 
La magie a disparu depuis des lustres, mais quelques signaux anormaux commencent à être perçus.
\section{Les Terres Humaines}
\subsection{Sombre-Cime}
\subsubsection{Description}
La population de la cité s'est de plus en plus rapprochée des mauriacs, la plupart des humains ayant choisi l'exode face au racisme croissant exprimé par les habitants du royaume voisin. La technologie de la cité est donc très proche de celle des mauriacs. La cité s'éloigne de plus en plus des autres cités humaines, et a maintenant des relations plus que froide avec Anksfall. On peut par exemple citer le conflit des 30 jours, durant lequel Sombre-Cime fut sous blocus de la part de l'imposante cité humaine d'Anksfall.
\subsection{Anksfall}
\subsubsection{Description}
La ville est assez avancée technologiquement, principalement dans le domaine des armes et de la production d'énergie. La découverte d'explosifs bon marché et faciles à produire a permis de créer des armes à feu très modernes et puissantes. C'est le fleuron de la technologie humaine.
\subsubsection{Direction politique}
Anksfall se réclame encore et toujours capitale de l'empire humain, et un empereur y vis toujours. Ce dernier na cependant plus aucun pouvoir en dehors des frontières de la cité et des plaines avoisinantes. Seuls les cités de Grahyrst en en moinde mesure Leheath conservent des liens puissants avec Anksfall, notamment économiques.
\subsection{Grahysrt}
\subsubsection{Description}
Les luttes de pouvoir incessantes qui ont pris place dans la cité ont lourdement freiné son économie, mais cette dernière commence à se relever et ses chantiers navals se remettent en marche, l'industrie de la cité étant de nouveau entièrement tournée vers la production de navires. On murmure que les chantiers navals tournent à plein régime et que des navires de guerre dotés des dernières innovations techniques commencent à sortir de la cité. Cette dernière est très proche du pouvoir central d'Anksfall, et la reconnaissance de l'empereur d'Anksfall par la cité-état revient régulièrement sur la table des dirigeants de la ville.
\subsubsection{Direction politique}
La direction de la ville est assurée par un conseil de marchands, qui reignent ici en rois. À tour de rôle, ces derniers se répartissent l'exécutif.
\subsection{Leheath}
\subsubsection{Description}
La cité est restée relativement en marge du developpement scientifique, et n'a pour ainsi dire fait aucune découverte majeure récemment. Cependant, elle reste la capitale des plaisirs, et ne se prive pas de profiter des innovations techniques de ses voisins... À sa manière.
\subsubsection{Direction politique}
Ici, c'est le prince qui dirige. Dans une vie de débauche et de luxures délicieuses, ce dernier assure l'exécutif de la cité-état. Il est heureusement secondé par nombre de conseillers, qui assurent le reste de la politique de l'état.
\subsubsection{Le solarium}
Il s'agit du joyau de la cité. Une immense baie vitrée, où reigne en permanence 35\degre. Transats, bar et autres piscines se succèdent sur quelques kilomètres, assurant des vacances inégallées sur tout le monde humain.
\section{Thargelon}
\subsection{Ketelundr}
\subsubsection{Description}
Progressivement, les mauriacs de ce royaume se sont ouverts au monde et leurs relations avec les humains se sont améliorées. Ce regain de l'entente locale a avantagé les mauriacs de Thargelon puisque la cité est devenue plus puissante que jamais et connaît son âge d'or.
\subsubsection{Technologie}
La technologie de ce royaume nain est assez orthodoxe, bien que très avancée. Les ingénieurs de Thargelon sont passés maîtres dans l'art de la miniaturisation, si bien que la plupart des mauriacs du royaume délaissent les taches manuelles au profit d'une grande compétence en ingénierie. Cela ne se fait pas sans problème, et les nains ont parfois recours à l'utilisation de procédés douteux pour subvenir à leur besoin en main d’œuvre. L'armée, quant à elle, préfère les affrontements à distance utilisant drones et missiles à un combat rapproché. Les générateurs de la cité, merveilles technologiques, utilisent les forces naturelles et les convertissent en énergie utilisable au quotidien.
\section{Kazadren}
\subsection{Dren}
\subsubsection{Description}
Au fur et à mesure du temps, les mauriacs de Kazadren se sont repliés sur eux-mêmes, et un racisme croissant envers toutes les autres races les a entraînés à haïr les mauriacs de Thargelon qui acceptent de commercer avec le reste du monde.
\subsubsection{Le géranium}
Un nouveau minerai découvert dans les montagnes voisines de la cité est désormais utilisé par les ingénieurs mauriacs: ils s'agit du géranium (Certains lui préfèreront sa dénomination scientifique de "Môr-ladron"). Ce métal est utilisé pour produire des armes dévastatrices mais à la manipulation très risquée. Ce Môr-ladron peut également servir de source d'énergie car il dégage une chaleur intense. Cette énergie peut être utilisée pour alimenter des véhicules par exemple. Des décennies d'expérimentation, ainsi que de nombreux décès, ont été nécessaires afin de mettre au point ces armes, et de nombreux mauriacs ont trouvé la mort en testant ce genre d'engins. Récemment, les mines de géranium ont commencé à fourmiller d'activité, et les usines tournent en permanence à produire un débit ahurissant de technologie létale.
\section{Le désert Orque}
\subsection{Mauhagr}
\subsubsection{Description}
Les Orques possèdent peu de technologie, se contentant de piller celle des humains lorsque l'occasion se présente. Ils sont depuis longtemps repliés sur eux-mêmes, organisant des raids de temps en temps.
\section{La faille Skhos}
\subsection{Vrag}
\subsubsection{Description}
Les Skhos étant de grands marchands, Vrag a connu un important développement et est progressivement devenue une plaque tournante du commerce mondial. Cette situation satisfait grandement la mafia locale, car elle n'hésite pas à taxer à volonté le commerce international qui transite par Vrag. Correctement avancés technologiquement, les skhos privilégient le commerce pour se fournir ce dont ils ont besoin. Ils sont peu orientés vers les armes, et seront sans doute les premières victimes collatérales si une guerre éclate.
\subsubsection{Direction politique}
Vrag, la capitale skhossienne, est dirigée par sa mafia, du moins officieusement. Il existe bien un grand roi skhos, mais ce dernier ne fait pas un pas sans l'assentiment des barons de la mafias locale.
\section{L'eau et la terre}
\subsection{Rinam}
\subsubsection{Description}
Les Dalcés, de plus en plus tournés vers la piraterie et la violence, augmentent la fréquence et l'intensité de leurs attaques pour alimenter leurs velléités expansionnistes. Ils méprisent le combat à distance, privilégiant une approche furtive suivie d'un assaut foudroyant. C'est pour cette raison que leurs véhicules légers, leurs armes de poing et de corps-à-corps sont aussi réputées que convoitées.
\subsubsection{Direction politique}
Rinam demeure une démocratie, et en est très fière ! Le président ne cesse d'ailleurs de mettre en oeuvre un "soft-power" redoutable, et on peut observer dans toutes les villes possédant un régime autortaire, notamment les villes humaines, la naissance de cellules républicaines.
\subsection{Thorneye}
\subsubsection{Description}
Les Alvénis bruns renient et méprisent toute technologie. Un prétendu guide spirituel affermit son emprise sur la société et la race entière périclite, s'accrochant désespérément aux derniers restes de sa civilisation.
\subsection{Aegnord} 
\subsubsection{Description}
Les Alvénis Gris ont vite compris qu'ils pourraient tirer profit de la technologie. Alors que certains d'entre eux restent méfiants par nature, d'autre se vouent corps et âme à la recherche sur les Intelligences Artificielles (IA). Certains d'entre eux devinrent même addict à l'utilisation de gadgets technologiques et basèrent leur vie sur leur utilisation. Leur civilisation commença à dépérir lentement mais sûrement et la technologie sur laquelle ils se reposaient signa leur perte, après des années de prospérité et de leadership dans le domaine de la technologie et de l'IA.


\chapter{Les races post-apocalyptiques}
\section{Les Humains}
\subsection{Caractéristiques physiques}
Les humains sont sont assez différents des ce côté-là. Il peuvent mesurer de 1m6o à 2m10. Les humains du nord, du fait de croisement avec les nains, sont de plus petite taille. Par contre, les humains de sud, qui connaissent mieux les orques et les alvénis, sont relativement plus grand. La plupart des personnes ont les yeux et les cheveux d’un brun profond. Les humains ne tendent pas à avoir un corps athlétique naturellement, bien que la plupart s’entraine rigoureusement dans ce but. 
\subsection{Us et coutumes}
\subsubsection{Mariage}
Les hommes se marient vers l’âge de vingt ans avec des jeunes filles de quinze ans. Ils se font souvent par intérêt financier, la marié amenant sa fameuse dot. Les mariages ’amoureux’ sont assez mal vu, car synonyme la plupart du temps de pauvreté. 
\subsubsection{Héritage}
Les biens se transmettent de père en fils (ou fille s’y il n’y a pas de fils), par testament. La loi oblige le testamentaire à signer de son propre sang, pour assurer son consentement. 
\subsubsection{Repas}
Les humains prennent leur repas à heure fixe : 7h le matin, 11h le midi et 19h le soir. Ils mangent relativement copieusement, même les pauvres. C’est une chose sacré dans leur tradition. Mais il est vrai que les nordiques affectionnent encore plus les repas, étant indispensable à leur survis en extérieur. Ils y vouent un véritable culte et il serait impensable ne serait-ce que d'arriver en retard à un des diner. 
\subsubsection{Hiérarchie}
Les hommes apprécient beaucoup être classé selon leur place dans la société. Le plus haut rang est celui d’empereur, puis vient les titres de noblesses (dans l’ordre : Duc, Marquis, Comte, Vicomte et Baron). Ensuite vient l’armée, très appréciée et respectée chez les humains. Puis apparait la bourgeoisie. Là, l’ordre se fait en fonction de la richesse, et des relations (les deux n’étant pas souvent séparés). Enfin, le gros de la masse, le peuple, où comme pour la bourgeoisie, la place se fait en fonction de la fortune. Certains se trouvent en dessous du peuple, comme les bourreaux, les repris de justice et les vagabonds. 
\subsubsection{Relations aux autres peuples}
Les humains sont très accueillant et ouvert aux autres peuples et coutumes. Tous les postes leurs sont accordées, sauf dans la noblesse (encore qu’il y ai quelques exceptions). 

\section{Les Mauriacs}
\subsection{Caractéristiques physiques}
Les mauriacs sont de petits êtres, leur taille varie entre 80 centimètres et 1 mètre 50. Ils vivent exclusivement dans les montagnes (notamment celles de Kazadren et de Thargelon). C'est un peuple semi-troglodytique.

\section{Les Skhos}
\subsection{Caractéristiques physiques}
Les skhos sont des créatures pouvant mesurer de 95 cm à 1m45, bien que ceux des montagnes soit généralement plus petits et ceux des plaines (ces derniers usent aussi plus de leur bipédie, afin de voir plus loin). Ils possèdent une peau différente selon l’habitat d’origine: ocre pour le désert, vert, brun, marron pour les forêt et les plaines grise, noire pour les terrains rocheux ou blanche pour les lieux enneigés. Les skhos s'adaptent étonnamment bien à ces différents milieux. Ils ont néanmoins tous des yeux d'un orange vif.
\subsection{Métissage}
Certains skhos se sont mélangés avec d'autres races. Ce métissage est très mal vu et les “balzaf”, comme ils sont appelés, sont souvent expulsés des communautés skhosiennes.
\subsection{Organisation}
Il existe deux types de sociétés skhos, très différentes, et il est fortement déconseillé de prendre les uns pour les autres.
\subsection{Les skhos sauvages}
Les gobelins sauvages, qui ont un comportement tribal. Leurs sociétés varient, mais on retrouve généralement une forme clanique. Ces clans sont souvent isolés, mais ont la particularité d'être très agressif, et quelque soit la créature adverse. Ils se font souvent la guerre entre eux, mais il arrive qu'ils attaquent des petits villages humains isolés.
\subsection{Les grands skhos}
Les grands skhos vivent en grande communauté, en grande majorité à Vrag. Leur intelligence supérieure à celle de leurs confrères provient d'expériences magiques pratiquées par des sorcier aux abords de l'actuel faille de Skhos, il y a bien longtemps. Leur dirigeant est d'ailleurs par tradition un sorcier.
Ils souffrent cependant qu’on les y associe. Les grands skhos possèdent des lois strictement établies.
\subsubsection{Hiérarchie}
Tout en haut de l'échelle se trouve le roi-sorcier, puis... Plus rien. Les grands skhos ne reconnaissent officiellement que ce grand chef. Il s'agit après d'une hiérarchie officieuse, surtout basée sur la renommée marchande.
\subsubsection{Le commerce}
Quelque soit leurs origines, les grands skhos vouent un grand culte à ce qui est pour eux un art: le commerce.

\section{Les Dalcés}
\subsection{Caractéristiques physiques}
Les dalcés sont plutôt de haute stature, environ 1 mètre 80. Ils ont la peau d'un bleu très pale.
\subsection{Origines}
Les dalcés sont issus d'une très vielle migration d'alvénis.
\subsection{Hiérarchie}
\subsubsection{Le président}
Les dalcés ont un système démocratique, et donc élisent un président. Tous les 10 ans, de grandes élections ont lieux sur l'ile, et c'est un moment de grande réjouissance.

\section{Les Alvénis}
\subsection{Caractéristiques physiques}
Les Alvénis sont d'une taille moyenne, environ celle d'un humain. Ils sont particulièrement proche de la nature.

\section{Faune}
\subsection{Agont}
L'Agont est un petit animal, proche d'une souris, mais contrairement à cette dernière, il est ovipare.
Il se nourrit principalement de baies sauvages.
Il se cache dans de petits terriers, qu'il a lui même creusé dans les basses plaines.  
\section{Flore}
\subsection{Enylà}
L'Enylà est une plante très rustique.
Elle affectionne particulièrement les zones humides et dépourvues de toutes autres plantes.
Sa couleur terre la dissimule parfaitement aux yeux des herbivores.
Par contre son petit fruit rond est d'un bleu profond ce qui attire les animaux.
Elle recouvre des collines entières notamment dans la région de Natskhaz.


\chapter{Ressources post-apocalyptiques}
% Ce fichier est vide, remplissez-le !


%%%%%%%%%
\part{Ressources générales}
\chapter*{Ressources générales}
\section{Chronologie}
\subsection{Frise chronologique}
\renewcommand{\arraystretch}{1.8}

Voici une frise chronologique des différentes ères, qui résume les évènements historiques de Creare Mundum.
\begin{figure}[ht]
\begin{tabular}{r | l}
198 & Mort de Dolgarur (cf Note n\degre236)\\
500 & Les hommes dominent le monde connu, fin de l'ère raciale
\end{tabular}
\caption{L'ère raciale}
\end{figure}

\begin{figure}[ht]
\begin{tabular}{r | l}

\end{tabular}
\caption{L'ère médiévale}
\end{figure}

\begin{figure}[ht]
\begin{tabular}{r | l}
2500 & Découverte de l'éléctricité
\end{tabular}
\caption{L'ère moderne}
\end{figure}

\begin{figure}[ht]
\begin{tabular}{r | l}
2800 & Premières recherches sur l'énergie nucléaire \\
2825 & Unification sacrée des peuples humains craignant leur anihilation
\end{tabular}
\caption{L'ère technologique}
\end{figure}

\begin{figure}[ht]
\begin{tabular}{r | l}
3010 & Unification des clans pirates\\
3021 & Début de l'Apocalypse, l'ensemble des peuples entre en guerre\\
3025 & Chute d'Anksfall\\
3030 & Fin de l'Apocalypse, chute de l'empire humain
\end{tabular}
\caption{L'Apocalypse}
\end{figure}

\begin{figure}[ht]
\begin{tabular}{r | l}
3030 & Fin du conflit armé\\
3044 & Arym arrive au pouvoir\\
\end{tabular}
\caption{L'ère post-apocalyptique}
\end{figure}
\newpage
\section{Participation au projet}
\subsection{Comment?}
\hypertarget{participation}{}
Creare Mundum vous a plu? 
Envie de partager vos scénarios, vos suggestions ou vos idées?
\newline
Contactez-nous sur la mailing list: \href {mailto:crearemundum@lists.tuxfamily.org}{crearemundum@lists.tuxfamily.org}
\newline
Ou rendez vous sur notre site: \href {http://creare-mundum.tuxfamily.org/} {http://creare-mundum.tuxfamily.org/}
\subsection{Les créateurs}
Voici la liste de ceux qui ont participé au projet Creare Mundum. Leur aide fut, est et sera toujours très précieuse au projet. Merci encore!  
\begin{itemize}
\item Alexandre ’Nobrakal’ Moine 
\item Philippe ’Tymophil’ Aubé 
\item Eliott ’Sulf’ Filippi
\item Émiland ’Derec’ Garrabé
\item Luc H.
\end{itemize}
\subsection{Remerciements}
Voilà maintenant la liste de ceux qui ont contribué à plus petite mesure, et souvent sans le savoir. 
\begin{itemize}
\item Dan Scott, pour l’image d’arodef.
\item Les artistes du MMORPG Ryzom, pour l'habitation de Thorneye ainsi que la dague d'Aegnord (modifiée). 
\end{itemize}
\subsection{Licence}
Creare Mundum est un projet libre de droit, publié sous la licence Creative Commons BY-SA. C'est à dire que quiconque a la possibilité d'utiliser ce document (ainsi que toute les autres parties du projet), de le redistribuer et de le modifier. La seule obligation est de redistribuer le contenu (modifié ou non) sous les mêmes conditions.
\end{document}

